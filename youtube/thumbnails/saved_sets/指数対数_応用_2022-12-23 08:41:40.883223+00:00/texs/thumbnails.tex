
\documentclass[10pt,
% a4paper,
%twocolumn,
fleqn,
%landscape, 
% papersize,
dvipdfmx,
uplatex
]{jsarticle}



\def\maru#1{\textcircled{\scriptsize#1}}%丸囲み番号

% \RequirePackage[2020/09/30]{platexrelease}

%太字設定
\usepackage[deluxe]{otf}

\usepackage{emathEy}

\usepackage[g]{esvect}

%大きな文字
\usepackage{fix-cm}

%定理環境
\usepackage{emathThm}
%\theoremstyle{boxed}
\theorembodyfont{\normalfont}
\newtheorem{Question}{問題}[subsection]
\newtheorem{Q}{}[subsection]
\newtheorem{question}[Question]{}
\newtheorem{quuestion}{}[subsection]

%セクション,大問番号のデザイン
\renewcommand{\labelenumi}{(\arabic{enumi})}
\renewcommand{\theenumii}{\alph{enumii})}
\renewcommand{\thesection}{第\arabic{section}章}

%用紙サイズの詳細設定
\usepackage{bxpapersize}
\papersizesetup{size={80mm,45mm}}
\usepackage[top=0.7zw,bottom=0truemm,left=3truemm,right=133truemm]{geometry}
\usepackage[dvipdfmx]{graphicx}

%余白など
\usepackage{setspace} % 行間
\setlength{\mathindent}{1zw}
\setlength\parindent{0pt}


%色カラーに関する設定
\usepackage{color}
\definecolor{shiro}{rgb}{0.95703125,0.87109375,0.7421875}
\definecolor{kin}{rgb}{0.95703125,0.87109375,0.7421875}
\definecolor{orange}{rgb}{1,0.7,0.2}
\definecolor{bradorange}{rgb}{1,0.5,0}
\definecolor{pink}{rgb}{0.9176,0.5686,0.5960}
\definecolor{mizu}{rgb}{0.6156,0.8,0.9955}
\color{kin}
% \pagecolor{hukamido}

\usepackage{at}%図の配置
% \usepackage{wallpaper}

\begin{document}

\bf\boldmath



\at(0cm,0cm){\includegraphics[width=8cm,bb=0 0 1920 1080]{./youtube/thumbnails/templates/smart_background/指数対数.jpeg}}
\at(7.0cm,0.2cm){\small\color{bradorange}$\overset{\text{指数対数}}{\text{応用}}$}
{\color{orange}\Large\underline{$3$乗根の無理数性$〜$阪大$〜$}}\vspace{0.3zw}

\normalsize 
問.以下の問いに答えよ.\\
(1)  $\sqrt 2,\;\sqrt[3]3$が無理数であることを示せ.\\
(2)  $p,\;q,\;\sqrt 2p+\sqrt[3]3q$がすべて有理数であることする.このとき,$p=q=0$であることを示せ.\\



\newpage



\at(0cm,0cm){\includegraphics[width=8cm,bb=0 0 1920 1080]{./youtube/thumbnails/templates/smart_background/指数対数.jpeg}}
\at(7.0cm,0.2cm){\small\color{bradorange}$\overset{\text{指数対数}}{\text{応用}}$}
{\color{orange}\normalsize\underline{無理数の無理数乗は有理数になりえるか}}\vspace{0.3zw}

\normalsize 
問.次の各問いに答えよ.\\
(1)  $\log _34$は無理数であることを示せ.\\
(2)  $a,\;b$がともに無理数で,$a^b$は有理数であるような
数$a,\;b$の組を$1$組求めよ.\\



\newpage



\at(0cm,0cm){\includegraphics[width=8cm,bb=0 0 1920 1080]{./youtube/thumbnails/templates/smart_background/指数対数.jpeg}}
\at(7.0cm,0.2cm){\small\color{bradorange}$\overset{\text{指数対数}}{\text{応用}}$}
{\color{orange}\Large\underline{底に文字を含む対数不等式}}\vspace{0.3zw}

\normalsize 
問.次の$x$についての不等式を解け.ただし,$a$は$1$ではない正の実数とする.\\
(1)  $\log _a\left(2x+{13}\right)>\log _a\left(4-x\right)$\\
(2)  $\log _a\left(x-a\right)\geqq \log _{a^2}\left(x-a\right)$\\
(3)  $\log _ax\leqq \log _xa$\\



\newpage



\at(0cm,0cm){\includegraphics[width=8cm,bb=0 0 1920 1080]{./youtube/thumbnails/templates/smart_background/指数対数.jpeg}}
\at(7.0cm,0.2cm){\small\color{bradorange}$\overset{\text{指数対数}}{\text{応用}}$}
{\color{orange}\Large\underline{対数方程式が実数解を持つ条件}}\vspace{0.3zw}

\large 
問.$x$についての方程式

\vspace{0.3zw}
\hspace{0.5zw}$\log _3\left(x-3\right)=\log _9\left(kx-6\right)\vspace{0.3zw}$


が相異なる$2$つの解をもつように,実数$k$の範囲を求めよ.


\newpage



\at(0cm,0cm){\includegraphics[width=8cm,bb=0 0 1920 1080]{./youtube/thumbnails/templates/smart_background/指数対数.jpeg}}
\at(7.0cm,0.2cm){\small\color{bradorange}$\overset{\text{指数対数}}{\text{応用}}$}
{\color{orange}\huge\underline{対数方程式}}\vspace{0.3zw}

\normalsize 
問.次の方程式を解け.\\
(1)  $\log _2\left(x^2-2x\right)=\log _2\left(3x-4\right)$\\
(2)  $\log _2\left(x+2\right)+\log _2\left(x-5\right)=3$\\
(3)  $\log _{\frac{1}{3}}\left(6-x\right)+2\log _3x=0$\\



\newpage



\at(0cm,0cm){\includegraphics[width=8cm,bb=0 0 1920 1080]{./youtube/thumbnails/templates/smart_background/指数対数.jpeg}}
\at(7.0cm,0.2cm){\small\color{bradorange}$\overset{\text{指数対数}}{\text{応用}}$}
{\color{orange}\LARGE\underline{$100^{99}$と$99^{100}$の大小比較}}\vspace{0.3zw}

\Large 
問.${100}^{99}$と${99}^{100}$の大小を判定せよ.ただし,必要なら近似値$\log _{{10}}2=0.{3010},\;\log _{{10}}3=0.{4771}$を用いて良い.


\newpage



\at(0cm,0cm){\includegraphics[width=8cm,bb=0 0 1920 1080]{./youtube/thumbnails/templates/smart_background/指数対数.jpeg}}
\at(7.0cm,0.2cm){\small\color{bradorange}$\overset{\text{指数対数}}{\text{応用}}$}
{\color{orange}\Large\underline{常用対数の近似値$〜$津田塾大$〜$}}\vspace{0.3zw}

\large 
問.次の値を小数第$1$位まで求めよ.小数第$2$以下は切り捨てよ.\\
(1)  $\log _{10}2$\\
(2)  $\log _{10}5$\\
(3)  $\log _{10}3$\\



\newpage



\at(0cm,0cm){\includegraphics[width=8cm,bb=0 0 1920 1080]{./youtube/thumbnails/templates/smart_background/指数対数.jpeg}}
\at(7.0cm,0.2cm){\small\color{bradorange}$\overset{\text{指数対数}}{\text{応用}}$}
{\color{orange}\Large\underline{対数不等式が表す領域$〜$京大$〜$}}\vspace{0.3zw}

\normalsize 
問.不等式

\vspace{0.3zw}
\hspace{0.5zw}$\log _xy+\log _yx>2+\left(\log _x2\right)\left(\log _y2\right)\vspace{0.3zw}$


を満たす$x,\;y$の組$\left(x,\;y\right)$の範囲を座標平面上に図示せよ.


\newpage



\at(0cm,0cm){\includegraphics[width=8cm,bb=0 0 1920 1080]{./youtube/thumbnails/templates/smart_background/指数対数.jpeg}}
\at(7.0cm,0.2cm){\small\color{bradorange}$\overset{\text{指数対数}}{\text{応用}}$}
{\color{orange}\LARGE\underline{対数不等式が表す領域}}\vspace{0.3zw}

\Large 
問.不等式

\vspace{0.3zw}
\hspace{0.5zw}$1<\log _xy<2\vspace{0.3zw}$


を満たす点$\left(x,\;y\right)$の領域を図示せよ.


\newpage



\at(0cm,0cm){\includegraphics[width=8cm,bb=0 0 1920 1080]{./youtube/thumbnails/templates/smart_background/指数対数.jpeg}}
\at(7.0cm,0.2cm){\small\color{bradorange}$\overset{\text{指数対数}}{\text{応用}}$}
{\color{orange}\LARGE\underline{指数方程式 Lv.2 }}\vspace{0.3zw}

\normalsize 
問.次の各々の等式を満たす実数$x$の値を求めよ.\\
(1)  $\left(2^x\right)^2-5\cdot 2^x+4=0$\\
(2)  $9^x-2\cdot 3^x-3=0$\\
(3)  $4^{x+1}+2\cdot 2^x-2=0$\\



\newpage



\at(0cm,0cm){\includegraphics[width=8cm,bb=0 0 1920 1080]{./youtube/thumbnails/templates/smart_background/指数対数.jpeg}}
\at(7.0cm,0.2cm){\small\color{bradorange}$\overset{\text{指数対数}}{\text{応用}}$}
{\color{orange}\Large\underline{指数方程式が実数解を持つ条件}}\vspace{0.3zw}

\Large 
問.方程式

\vspace{0.3zw}
\hspace{0.5zw}$4^x-a\cdot 2^{x+1}+a+2=0\vspace{0.3zw}$


を満たす実数$x$が存在するような実数$a$の値を求めよ.


\newpage



\at(0cm,0cm){\includegraphics[width=8cm,bb=0 0 1920 1080]{./youtube/thumbnails/templates/smart_background/指数対数.jpeg}}
\at(7.0cm,0.2cm){\small\color{bradorange}$\overset{\text{指数対数}}{\text{応用}}$}
{\color{orange}\LARGE\underline{指数方程式の解の配置}}\vspace{0.3zw}

\large 
問.方程式

\vspace{0.3zw}
\hspace{0.5zw}$9^x+2a\cdot 3^x+2a^2+a-6=0\vspace{0.3zw}$


を満たす$x$の正の解,負の解が$1$つずつ存在するような,定数$a$の値のとる範囲を求めよ.


\newpage



\at(0cm,0cm){\includegraphics[width=8cm,bb=0 0 1920 1080]{./youtube/thumbnails/templates/smart_background/指数対数.jpeg}}
\at(7.0cm,0.2cm){\small\color{bradorange}$\overset{\text{指数対数}}{\text{応用}}$}
{\color{orange}\LARGE\underline{小数首位とその数字}}\vspace{0.3zw}

\Large 
問.$\left(\bunsuu{2}{5}\right)^{50}$は小数第何位に初めて$0$でない数字が現れるか.また,その数字を求めよ.ただし,必要ならば$\log _{10}2=0.{3010}$を用いて良い.


\newpage



\at(0cm,0cm){\includegraphics[width=8cm,bb=0 0 1920 1080]{./youtube/thumbnails/templates/smart_background/指数対数.jpeg}}
\at(7.0cm,0.2cm){\small\color{bradorange}$\overset{\text{指数対数}}{\text{応用}}$}
{\color{orange}\Large\underline{$3^{3^{3^3}}$の桁数の桁数を求めよ}}\vspace{0.3zw}

\small 
問.(1)  $\log _3x=3$を満たす整数$x$を求めよ.\\
(2)  $\log _3\left(\log _3x\right)=3$を満たす整数$x$は何桁か.また,最高位の数字を求めよ.\\
(3)  $\log _3\left(\log _3\left(\log _3x\right)\right)=3$を満たす整数$x$の桁数を$n$とするとき,$n$は何桁か.
必要ならば$\log _{10}2=0.{3010},\;\log _{10}3=0.{4771}$を用いて良い.\\



\newpage



\at(0cm,0cm){\includegraphics[width=8cm,bb=0 0 1920 1080]{./youtube/thumbnails/templates/smart_background/指数対数.jpeg}}
\at(7.0cm,0.2cm){\small\color{bradorange}$\overset{\text{指数対数}}{\text{応用}}$}
{\color{orange}\LARGE\underline{桁数を不等式で表す}}\vspace{0.3zw}

\normalsize 
問.(1)  ${29}^{100}$は${147}$桁である.${29}^{23}$は何桁の数となるか.\\
(2)  $\left(1.{25}\right)^n$の整数部分が$3$桁となる自然数$n$はどんな範囲の数か.
ただし,必要ならば$\log _{10}2=0.{3010},\;\log _{10}3=0.{4771}$を用いて良い.\\



\newpage



\at(0cm,0cm){\includegraphics[width=8cm,bb=0 0 1920 1080]{./youtube/thumbnails/templates/smart_background/指数対数.jpeg}}
\at(7.0cm,0.2cm){\small\color{bradorange}$\overset{\text{指数対数}}{\text{応用}}$}
{\color{orange}\LARGE\underline{桁数,最高位,最高次位}}\vspace{0.3zw}

\large 
問.次の問いに答えよ.\\
(1)  $2^{{2019}}$は何桁か.\\
(2)  $2^{{2019}}$の最高位の数は何か.\\
(3)  $2^{{2019}}$の最高次位の数は何か.\\



\end{document}

