\documentclass[10pt,
% a4paper,
%twocolumn,
fleqn,
%landscape, 
% papersize,
dvipdfmx,
uplatex
]{jsarticle}



\def\maru#1{\textcircled{\scriptsize#1}}%丸囲み番号

% \RequirePackage[2020/09/30]{platexrelease}

%太字設定
\usepackage[deluxe]{otf}

\usepackage{emathEy}

\usepackage[g]{esvect}

%大きな文字
\usepackage{fix-cm}

%定理環境
\usepackage{emathThm}
%\theoremstyle{boxed}
\theorembodyfont{\normalfont}
\newtheorem{Question}{問題}[subsection]
\newtheorem{Q}{}[subsection]
\newtheorem{question}[Question]{}
\newtheorem{quuestion}{}[subsection]

%セクション,大問番号のデザイン
\renewcommand{\labelenumi}{(\arabic{enumi})}
\renewcommand{\theenumii}{\alph{enumii})}
\renewcommand{\thesection}{第\arabic{section}章}

%用紙サイズの詳細設定
\usepackage{bxpapersize}
\papersizesetup{size={80mm,45mm}}
\usepackage[top=0.7zw,bottom=0truemm,left=3truemm,right=133truemm]{geometry}
\usepackage[dvipdfmx]{graphicx}

%余白など
\usepackage{setspace} % 行間
\setlength{\mathindent}{1zw}
\setlength\parindent{0pt}


%色カラーに関する設定
\usepackage{color}
\definecolor{shiro}{rgb}{0.95703125,0.87109375,0.7421875}
\definecolor{kin}{rgb}{0.95703125,0.87109375,0.7421875}
\definecolor{orange}{rgb}{1,0.7,0.2}
\definecolor{bradorange}{rgb}{1,0.5,0}
\definecolor{pink}{rgb}{0.9176,0.5686,0.5960}
\definecolor{mizu}{rgb}{0.6156,0.8,0.9955}
\color{kin}
% \pagecolor{hukamido}

\usepackage{at}%図の配置
% \usepackage{wallpaper}

\begin{document}



\at(0cm,0cm){\includegraphics[width=8cm,bb=0 0 1920 1080]{./youtube/thumbnails/templates/smart_background/数II微積.jpeg}}
{\color{orange}\bf\boldmath\normalsize\underline{放物線と$2$本の接線で囲まれた部分の面積}}\vspace{0.3zw}

\LARGE 
\bf\boldmath 問.放物線$y=x^2-4x+3$と、この放物線上の点$\left(0,\;3\right),\;\left(6,\;{15}\right)$に置ける接線で囲まれた図形の面積$S$を求めよ.
\at(7.0cm,0.2cm){\small\color{bradorange}$\overset{\text{数Ⅱ微積}}{\text{典型}}$}


\newpage



\at(0cm,0cm){\includegraphics[width=8cm,bb=0 0 1920 1080]{./youtube/thumbnails/templates/smart_background/数II微積.jpeg}}
{\color{orange}\bf\boldmath\Large\underline{放物線で囲まれた面積の最小}}\vspace{0.3zw}

\Large 
\bf\boldmath 問.放物線$y=x^2$と点$\left(1,\;2\right)$を通る直線とで囲まれた図形の面積$S$が最小になるとき、その直線の方程式を求めよ.
\at(7.0cm,0.2cm){\small\color{bradorange}$\overset{\text{数Ⅱ微積}}{\text{典型}}$}


\newpage



\at(0cm,0cm){\includegraphics[width=8cm,bb=0 0 1920 1080]{./youtube/thumbnails/templates/smart_background/数II微積.jpeg}}
{\color{orange}\bf\boldmath\Large\underline{放物線で囲まれた面積の等分}}\vspace{0.3zw}

\LARGE 
\bf\boldmath 問.放物線$y=2x-x^2$と$x$軸で囲まれた図形の面積を直線$y=kx$が$2$等分するように、定数$k$の値を定めよ.
\at(7.0cm,0.2cm){\small\color{bradorange}$\overset{\text{数Ⅱ微積}}{\text{典型}}$}


\newpage



\at(0cm,0cm){\includegraphics[width=8cm,bb=0 0 1920 1080]{./youtube/thumbnails/templates/smart_background/数II微積.jpeg}}
{\color{orange}\bf\boldmath\LARGE\underline{$\displaystyle\frac{1}{6}$公式の利用}}\vspace{0.3zw}

\Large 
\bf\boldmath 問.次の曲線や直線で囲まれた図形の面積$S$を求めよ.\\
(1)  $y=x^2-3x+5,\;y=2x-1$\\
(2)  $y=2x^2-6x+4,\;y=-3x^2+9x-6$\\

\at(7.0cm,0.2cm){\small\color{bradorange}$\overset{\text{数Ⅱ微積}}{\text{典型}}$}


\newpage



\at(0cm,0cm){\includegraphics[width=8cm,bb=0 0 1920 1080]{./youtube/thumbnails/templates/smart_background/数II微積.jpeg}}
{\color{orange}\bf\boldmath\normalsize\underline{積分方程式$〜$定積分で表された関数$〜$}}\vspace{0.3zw}

\Large 
\bf\boldmath 問.次の等式を満たす関数$f\left(x\right)$を求めよ.\\
(1)  $f\left(x\right)=3x^2-x\displaystyle\int_0^2f\left(t\right)dt+2$\\
(2)  $f\left(x\right)=1+\displaystyle\int_0^1\left(x-t\right)f\left(t\right)dt$\\

\at(7.0cm,0.2cm){\small\color{bradorange}$\overset{\text{数Ⅱ微積}}{\text{典型}}$}


\newpage



\at(0cm,0cm){\includegraphics[width=8cm,bb=0 0 1920 1080]{./youtube/thumbnails/templates/smart_background/数II微積.jpeg}}
{\color{orange}\bf\boldmath\Large\underline{積分方程式$〜$定積分の微分$〜$}}\vspace{0.3zw}

\Large 
\bf\boldmath 問.等式

\vspace{0.3zw}
\hspace{0.5zw}$\displaystyle\int_a^xf\left(t\right)dt=x^3-3x^2+x+a\vspace{0.3zw}$


を満たす関数$f\left(x\right)$と定数$a$の値の範囲を求めよ.
\at(7.0cm,0.2cm){\small\color{bradorange}$\overset{\text{数Ⅱ微積}}{\text{典型}}$}


\newpage



\at(0cm,0cm){\includegraphics[width=8cm,bb=0 0 1920 1080]{./youtube/thumbnails/templates/smart_background/数II微積.jpeg}}
{\color{orange}\bf\boldmath\huge\underline{接線の本数}}\vspace{0.3zw}

\Large 
\bf\boldmath 問.点$\left(0,\;k\right)$から曲線

\vspace{0.3zw}
\hspace{0.5zw}$y=x^3+2x^2-4x\vspace{0.3zw}$


に引くことのできる接線の本数を求めよ.
\at(7.0cm,0.2cm){\small\color{bradorange}$\overset{\text{数Ⅱ微積}}{\text{典型}}$}


\newpage



\at(0cm,0cm){\includegraphics[width=8cm,bb=0 0 1920 1080]{./youtube/thumbnails/templates/smart_background/数II微積.jpeg}}
{\color{orange}\bf\boldmath\LARGE\underline{$3$次方程式の実数解の個数}}\vspace{0.3zw}

\normalsize 
\bf\boldmath 問.$3$次方程式$2x^3+3x^2-{12}x+a=0$が次の解をもつとき、
定数$a$の値の範囲を求めよ.\\
(1)  異なる$3$つの実数解\\
(2)  ただ一つの実数解\\
(3)  異なる$2$つの正の解と負の解\\

\at(7.0cm,0.2cm){\small\color{bradorange}$\overset{\text{数Ⅱ微積}}{\text{典型}}$}


\newpage



\at(0cm,0cm){\includegraphics[width=8cm,bb=0 0 1920 1080]{./youtube/thumbnails/templates/smart_background/数II微積.jpeg}}
{\color{orange}\bf\boldmath\LARGE\underline{$4$次方程式の実数解の個数}}\vspace{0.3zw}

\LARGE 
\bf\boldmath 問.次の$4$次方程式の異なる実数解の個数を求めよ.

\vspace{0.3zw}
\hspace{0.5zw}$x^4-4x^3+4x^2-2=0\vspace{0.3zw}$


\at(7.0cm,0.2cm){\small\color{bradorange}$\overset{\text{数Ⅱ微積}}{\text{典型}}$}


\newpage



\at(0cm,0cm){\includegraphics[width=8cm,bb=0 0 1920 1080]{./youtube/thumbnails/templates/smart_background/数II微積.jpeg}}
{\color{orange}\bf\boldmath\large\underline{係数に文字を含む$3$次関数の最大最小}}\vspace{0.3zw}

\large 
\bf\boldmath 問.$a>0$とする.関数

\vspace{0.3zw}
\hspace{0.5zw}$f\left(x\right)=x^3-3a^2x\left(0\leqq x\leqq 1\right)\vspace{0.3zw}$


について、\\
(1)  最小値を求めよ.\\
(2)  最大値を求めよ.\\

\at(7.0cm,0.2cm){\small\color{bradorange}$\overset{\text{数Ⅱ微積}}{\text{典型}}$}


\newpage



\at(0cm,0cm){\includegraphics[width=8cm,bb=0 0 1920 1080]{./youtube/thumbnails/templates/smart_background/数II微積.jpeg}}
{\color{orange}\bf\boldmath\large\underline{区間に文字を含む$3$次関数の最大最小}}\vspace{0.3zw}

\large 
\bf\boldmath 問.$a>0$とする.関数

\vspace{0.3zw}
\hspace{0.5zw}$f\left(x\right)=x^3-3x^2+1\left(0\leqq x\leqq a\right)\vspace{0.3zw}$


について、\\
(1)  最小値を求めよ.\\
(2)  最大値を求めよ.\\

\at(7.0cm,0.2cm){\small\color{bradorange}$\overset{\text{数Ⅱ微積}}{\text{典型}}$}


\newpage



\at(0cm,0cm){\includegraphics[width=8cm,bb=0 0 1920 1080]{./youtube/thumbnails/templates/smart_background/数II微積.jpeg}}
{\color{orange}\bf\boldmath\huge\underline{極値の計算工夫}}\vspace{0.3zw}

\LARGE 
\bf\boldmath 問.関数

\vspace{0.3zw}
\hspace{0.5zw}$f\left(x\right)=x^3-3x^2-6x+5\vspace{0.3zw}$


の極値を求めよ.
\at(7.0cm,0.2cm){\small\color{bradorange}$\overset{\text{数Ⅱ微積}}{\text{典型}}$}


\newpage



\at(0cm,0cm){\includegraphics[width=8cm,bb=0 0 1920 1080]{./youtube/thumbnails/templates/smart_background/数II微積.jpeg}}
{\color{orange}\bf\boldmath\huge\underline{共通接線の方程式}}\vspace{0.3zw}

\Large 
\bf\boldmath 問.$2$つの放物線

\vspace{0.3zw}
\hspace{0.5zw}$y=x^2,\;\vspace{0.3zw}$



\vspace{0.3zw}
\hspace{0.5zw}$y=-x^2+6x-5\vspace{0.3zw}$


の共通接線の方程式を求めよ.
\at(7.0cm,0.2cm){\small\color{bradorange}$\overset{\text{数Ⅱ微積}}{\text{典型}}$}


\newpage



\at(0cm,0cm){\includegraphics[width=8cm,bb=0 0 1920 1080]{./youtube/thumbnails/templates/smart_background/数II微積.jpeg}}
{\color{orange}\bf\boldmath\huge\underline{接線の方程式}}\vspace{0.3zw}

\large 
\bf\boldmath 問.次の接線の方程式を求めよ.\\
(1)  曲線$y=x^2+4x$上の点$\left(1,\;5\right)$における接線\\
(2)  曲線$y=x^3-3x^2-1$に点$\left(0,\;0\right)$から引いた接線\\

\at(7.0cm,0.2cm){\small\color{bradorange}$\overset{\text{数Ⅱ微積}}{\text{典型}}$}


\newpage



\at(0cm,0cm){\includegraphics[width=8cm,bb=0 0 1920 1080]{./youtube/thumbnails/templates/smart_background/数II微積.jpeg}}
{\color{orange}\bf\boldmath\LARGE\underline{常に単調増加する$3$次関数}}\vspace{0.3zw}

\Large 
\bf\boldmath 問.$x$の$3$次関数

\vspace{0.3zw}
\hspace{0.5zw}$f\left(x\right)=x^3+3kx^2-kx-1\vspace{0.3zw}$


が常に単調増加するような定数$k$の値の範囲を求めよ.
\at(7.0cm,0.2cm){\small\color{bradorange}$\overset{\text{数Ⅱ微積}}{\text{典型}}$}


\newpage



\at(0cm,0cm){\includegraphics[width=8cm,bb=0 0 1920 1080]{./youtube/thumbnails/templates/smart_background/数II微積.jpeg}}
{\color{orange}\bf\boldmath\huge\underline{極値から係数決定}}\vspace{0.3zw}

\Large 
\bf\boldmath 問.関数

\vspace{0.3zw}
\hspace{0.5zw}$f\left(x\right)=x^3+ax^2-bx+c\vspace{0.3zw}$


が、$x=-1$で極大値$5$をとり、$x=1$で極小となるとき、定数$a,\;b,\;c$の値を求めよ.
\at(7.0cm,0.2cm){\small\color{bradorange}$\overset{\text{数Ⅱ微積}}{\text{典型}}$}


\end{document}

