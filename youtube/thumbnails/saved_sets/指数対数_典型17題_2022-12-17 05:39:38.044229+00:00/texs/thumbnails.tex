\documentclass[10pt,
% a4paper,
%twocolumn,
fleqn,
%landscape, 
% papersize,
dvipdfmx,
uplatex
]{jsarticle}



\def\maru#1{\textcircled{\scriptsize#1}}%丸囲み番号

% \RequirePackage[2020/09/30]{platexrelease}

%太字設定
\usepackage[deluxe]{otf}

\usepackage{emathEy}

\usepackage[g]{esvect}

%大きな文字
\usepackage{fix-cm}

%定理環境
\usepackage{emathThm}
%\theoremstyle{boxed}
\theorembodyfont{\normalfont}
\newtheorem{Question}{問題}[subsection]
\newtheorem{Q}{}[subsection]
\newtheorem{question}[Question]{}
\newtheorem{quuestion}{}[subsection]

%セクション,大問番号のデザイン
\renewcommand{\labelenumi}{(\arabic{enumi})}
\renewcommand{\theenumii}{\alph{enumii})}
\renewcommand{\thesection}{第\arabic{section}章}

%用紙サイズの詳細設定
\usepackage{bxpapersize}
\papersizesetup{size={80mm,45mm}}
\usepackage[top=0.7zw,bottom=0truemm,left=3truemm,right=133truemm]{geometry}
\usepackage[dvipdfmx]{graphicx}

%余白など
\usepackage{setspace} % 行間
\setlength{\mathindent}{1zw}
\setlength\parindent{0pt}


%色カラーに関する設定
\usepackage{color}
\definecolor{shiro}{rgb}{0.95703125,0.87109375,0.7421875}
\definecolor{kin}{rgb}{0.95703125,0.87109375,0.7421875}
\definecolor{orange}{rgb}{1,0.7,0.2}
\definecolor{bradorange}{rgb}{1,0.5,0}
\definecolor{pink}{rgb}{0.9176,0.5686,0.5960}
\definecolor{mizu}{rgb}{0.6156,0.8,0.9955}
\color{kin}
% \pagecolor{hukamido}

\usepackage{at}%図の配置
% \usepackage{wallpaper}

\begin{document}



\at(0cm,0cm){\includegraphics[width=8cm,bb=0 0 1920 1080]{./youtube/thumbnails/templates/smart_background/指数対数.jpeg}}
{\color{orange}\bf\boldmath\large\underline{指数関数の最大値$〜$逆数の対称式$〜$}}\vspace{0.3zw}

\LARGE 
\bf\boldmath 問.関数

\vspace{0.3zw}
\hspace{0.5zw}$y=\left(2^x+2^{-x}\right)-2\left(4^x+4^{-x}\right)\vspace{0.3zw}$


の最大値を求めよ.
\at(7.0cm,0.2cm){\small\color{bradorange}$\overset{\text{指数対数}}{\text{典型}}$}


\newpage



\at(0cm,0cm){\includegraphics[width=8cm,bb=0 0 1920 1080]{./youtube/thumbnails/templates/smart_background/指数対数.jpeg}}
{\color{orange}\bf\boldmath\large\underline{対数関数の最大値$〜$対数の$2$次式$〜$}}\vspace{0.3zw}

\Large 
\bf\boldmath 問.次の関数の最大値と最小値を求めよ.

\vspace{0.3zw}
\hspace{0.5zw}$y=\left(\log _2x\right)^2-\log _2x^2-3\left(1\leqq x\leqq {16}\right)\vspace{0.3zw}$


\at(7.0cm,0.2cm){\small\color{bradorange}$\overset{\text{指数対数}}{\text{典型}}$}


\newpage



\at(0cm,0cm){\includegraphics[width=8cm,bb=0 0 1920 1080]{./youtube/thumbnails/templates/smart_background/指数対数.jpeg}}
{\color{orange}\bf\boldmath\large\underline{対数関数の最大値$〜$真数が$2$次式$〜$}}\vspace{0.3zw}

\LARGE 
\bf\boldmath 問.関数

\vspace{0.3zw}
\hspace{0.5zw}$y=\log _2x+\log _2\left({16}-x\right)\vspace{0.3zw}$


の最大値を求めよ.
\at(7.0cm,0.2cm){\small\color{bradorange}$\overset{\text{指数対数}}{\text{典型}}$}


\newpage



\at(0cm,0cm){\includegraphics[width=8cm,bb=0 0 1920 1080]{./youtube/thumbnails/templates/smart_background/指数対数.jpeg}}
{\color{orange}\bf\boldmath\Large\underline{対数方程式$\cdot$不等式 Lv.2 }}\vspace{0.3zw}

\Large 
\bf\boldmath 問.次の方程式$\cdot$不等式を解け.\\
(1)  $\log _2x+\log _2\left(x-7\right)=3$\\
(2)  $2\log _2\left(2-x\right)\leqq \log _2x$\\

\at(7.0cm,0.2cm){\small\color{bradorange}$\overset{\text{指数対数}}{\text{典型}}$}


\newpage



\at(0cm,0cm){\includegraphics[width=8cm,bb=0 0 1920 1080]{./youtube/thumbnails/templates/smart_background/指数対数.jpeg}}
{\color{orange}\bf\boldmath\Large\underline{小数首位$〜$常用対数の利用$〜$}}\vspace{0.3zw}

\Large 
\bf\boldmath 問.$\left(\bunsuu{1}{{}30}\right)^{20}$を小数で表したとき,小数第何位に初めて$0$でない数字が現れるか.ただし,
$\log _{10}3=0.{4771}$とする.
\at(7.0cm,0.2cm){\small\color{bradorange}$\overset{\text{指数対数}}{\text{典型}}$}


\newpage



\at(0cm,0cm){\includegraphics[width=8cm,bb=0 0 1920 1080]{./youtube/thumbnails/templates/smart_background/指数対数.jpeg}}
{\color{orange}\bf\boldmath\large\underline{桁数と最高位の数字$〜$常用対数$〜$}}\vspace{0.3zw}

\Large 
\bf\boldmath 問.$\log _{10}2=0.{3010},\;\log _{10}3=0.{4771}$とする.\\
(1)  ${12}^{80}$は何桁の整数か.\\
(2)  ${12}^{80}$の最高位の数字を求めよ.\\

\at(7.0cm,0.2cm){\small\color{bradorange}$\overset{\text{指数対数}}{\text{典型}}$}


\newpage



\at(0cm,0cm){\includegraphics[width=8cm,bb=0 0 1920 1080]{./youtube/thumbnails/templates/smart_background/指数対数.jpeg}}
{\color{orange}\bf\boldmath\Large\underline{対数方程式$\cdot$不等式 Lv.1 }}\vspace{0.3zw}

\large 
\bf\boldmath 問.次の方程式$\cdot$不等式を解け.\\
(1)  $\log _2x=3$\\
(2)  $\log _2x<3$\\
(3)  $\log _{\frac{1}{3}}\left(x-1\right)\leqq 2$\\

\at(7.0cm,0.2cm){\small\color{bradorange}$\overset{\text{指数対数}}{\text{典型}}$}


\newpage



\at(0cm,0cm){\includegraphics[width=8cm,bb=0 0 1920 1080]{./youtube/thumbnails/templates/smart_background/指数対数.jpeg}}
{\color{orange}\bf\boldmath\LARGE\underline{指数に対数を含む数}}\vspace{0.3zw}

\Large 
\bf\boldmath 問.次の式の値を求めよ.\\
(1)  ${10}^{\log _{{10}}3}$\\
(2)  ${81}^{\log _3{10}}$\\

\at(7.0cm,0.2cm){\small\color{bradorange}$\overset{\text{指数対数}}{\text{典型}}$}


\newpage



\at(0cm,0cm){\includegraphics[width=8cm,bb=0 0 1920 1080]{./youtube/thumbnails/templates/smart_background/指数対数.jpeg}}
{\color{orange}\bf\boldmath\Large\underline{対数を利用した等式の証明}}\vspace{0.3zw}

\Large 
\bf\boldmath 問.$xyz\neq 0,\;2^x=3^y=6^z$のとき,次の等式が成り立つことを証明せよ.

\vspace{0.3zw}
\hspace{0.5zw}$\bunsuu{1}{x}+\bunsuu{1}{y}=\bunsuu{1}{z}\vspace{0.3zw}$


\at(7.0cm,0.2cm){\small\color{bradorange}$\overset{\text{指数対数}}{\text{典型}}$}


\newpage



\at(0cm,0cm){\includegraphics[width=8cm,bb=0 0 1920 1080]{./youtube/thumbnails/templates/smart_background/指数対数.jpeg}}
{\color{orange}\bf\boldmath\LARGE\underline{対数を他の対数で表す}}\vspace{0.3zw}

\LARGE 
\bf\boldmath 問.$a=\log _23,\;b=\log _37$のとき,$\log _{42}{56}$を$a,\;b$を用いて表せ.
\at(7.0cm,0.2cm){\small\color{bradorange}$\overset{\text{指数対数}}{\text{典型}}$}


\newpage



\at(0cm,0cm){\includegraphics[width=8cm,bb=0 0 1920 1080]{./youtube/thumbnails/templates/smart_background/指数対数.jpeg}}
{\color{orange}\bf\boldmath\huge\underline{底の変換公式}}\vspace{0.3zw}

\Large 
\bf\boldmath 問.次の式を簡単にせよ.\\
(1)  $\log _48$\\
(2)  $\log _23\cdot \log _38$\\
(3)  $\left(\log _23+\log _49\right)\left(\log _34+\log _92\right)$\\

\at(7.0cm,0.2cm){\small\color{bradorange}$\overset{\text{指数対数}}{\text{典型}}$}


\newpage



\at(0cm,0cm){\includegraphics[width=8cm,bb=0 0 1920 1080]{./youtube/thumbnails/templates/smart_background/指数対数.jpeg}}
{\color{orange}\bf\boldmath\huge\underline{対数の基本性質}}\vspace{0.3zw}

\Large 
\bf\boldmath 問.次の式を簡単にせよ.\\
(1)  $\log _64+\log _69$\\
(2)  $4\log _2\sqrt 3-\log _2{18}$\\

\at(7.0cm,0.2cm){\small\color{bradorange}$\overset{\text{指数対数}}{\text{典型}}$}


\newpage



\at(0cm,0cm){\includegraphics[width=8cm,bb=0 0 1920 1080]{./youtube/thumbnails/templates/smart_background/指数対数.jpeg}}
{\color{orange}\bf\boldmath\huge\underline{対数の定義}}\vspace{0.3zw}

\normalsize 
\bf\boldmath 問.次の対数の値を求めよ.\\
(1)  $\log _7{49}$\\
(2)  $\log _2{64}$\\
(3)  $\log _55$\\
(4)  $\log _41$\\
(5)  $\log _2\bunsuu{1}{{}81}$\\
(6)  $\log _{\frac{1}{5}}\sqrt {125}$\\

\at(7.0cm,0.2cm){\small\color{bradorange}$\overset{\text{指数対数}}{\text{典型}}$}


\newpage



\at(0cm,0cm){\includegraphics[width=8cm,bb=0 0 1920 1080]{./youtube/thumbnails/templates/smart_background/指数対数.jpeg}}
{\color{orange}\bf\boldmath\normalsize\underline{指数関数の最大値$〜2$次関数に帰着$〜$}}\vspace{0.3zw}

\Large 
\bf\boldmath 問.次の関数の最大値と最小値を求めよ.また,そのときの$x$の値を求めよ.

\vspace{0.3zw}
\hspace{0.5zw}$y=4^x-2^{x+2}+1\left(-1\leqq x\leqq 2\right)\vspace{0.3zw}$


\at(7.0cm,0.2cm){\small\color{bradorange}$\overset{\text{指数対数}}{\text{典型}}$}


\newpage



\at(0cm,0cm){\includegraphics[width=8cm,bb=0 0 1920 1080]{./youtube/thumbnails/templates/smart_background/指数対数.jpeg}}
{\color{orange}\bf\boldmath\Large\underline{指数方程式$\cdot$不等式 Lv.2 }}\vspace{0.3zw}

\Large 
\bf\boldmath 問.次の方程式$\cdot$不等式を解け.\\
(1)  $5^{2x+1}+4\cdot 5^x-1=0$\\
(2)  $4^x+2^x-{20}>0$\\

\at(7.0cm,0.2cm){\small\color{bradorange}$\overset{\text{指数対数}}{\text{典型}}$}


\newpage



\at(0cm,0cm){\includegraphics[width=8cm,bb=0 0 1920 1080]{./youtube/thumbnails/templates/smart_background/指数対数.jpeg}}
{\color{orange}\bf\boldmath\Large\underline{指数方程式$\cdot$不等式 Lv.1 }}\vspace{0.3zw}

\large 
\bf\boldmath 問.次の方程式$\cdot$不等式を解け.\\
(1)  $\left(\bunsuu{1}{9}\right)^x=3$\\
(2)  $4^x<8^{x-1}$\\
(3)  $\left(\bunsuu{1}{5}\right)^x\leqq \bunsuu{1}{{{125}}}$\\

\at(7.0cm,0.2cm){\small\color{bradorange}$\overset{\text{指数対数}}{\text{典型}}$}


\newpage



\at(0cm,0cm){\includegraphics[width=8cm,bb=0 0 1920 1080]{./youtube/thumbnails/templates/smart_background/指数対数.jpeg}}
{\color{orange}\bf\boldmath\Large\underline{指数計算$〜$逆数の対称式$〜$}}\vspace{0.3zw}

\large 
\bf\boldmath 問.$a>0$のとする.$a^{\frac{1}{3}}+a^{\frac{1}{3}}=4$のとき,次の式の値を求めよ.\\
(1)  $a+a^{-1}$\\
(2)  $a^{\frac{1}{2}}+a^{\frac{1}{2}}$\\

\at(7.0cm,0.2cm){\small\color{bradorange}$\overset{\text{指数対数}}{\text{典型}}$}


\end{document}

