\documentclass[10pt,
% a4paper,
%twocolumn,
fleqn,
%landscape, 
% papersize,
dvipdfmx,
uplatex
]{jsarticle}



\def\maru#1{\textcircled{\scriptsize#1}}%丸囲み番号

% \RequirePackage[2020/09/30]{platexrelease}

%太字設定
\usepackage[deluxe]{otf}

\usepackage{emathEy}

\usepackage[g]{esvect}

%大きな文字
\usepackage{fix-cm}

%定理環境
\usepackage{emathThm}
%\theoremstyle{boxed}
\theorembodyfont{\normalfont}
\newtheorem{Question}{問題}[subsection]
\newtheorem{Q}{}[subsection]
\newtheorem{question}[Question]{}
\newtheorem{quuestion}{}[subsection]

%セクション,大問番号のデザイン
\renewcommand{\labelenumi}{(\arabic{enumi})}
\renewcommand{\theenumii}{\alph{enumii})}
\renewcommand{\thesection}{第\arabic{section}章}

%用紙サイズの詳細設定
\usepackage{bxpapersize}
\papersizesetup{size={80mm,45mm}}
\usepackage[top=0.7zw,bottom=0truemm,left=3truemm,right=133truemm]{geometry}
\usepackage[dvipdfmx]{graphicx}

%余白など
\usepackage{setspace} % 行間
\setlength{\mathindent}{1zw}
\setlength\parindent{0pt}


%色カラーに関する設定
\usepackage{color}
\definecolor{shiro}{rgb}{0.95703125,0.87109375,0.7421875}
\definecolor{kin}{rgb}{0.95703125,0.87109375,0.7421875}
\definecolor{orange}{rgb}{1,0.7,0.2}
\definecolor{bradorange}{rgb}{1,0.5,0}
\definecolor{pink}{rgb}{0.9176,0.5686,0.5960}
\definecolor{mizu}{rgb}{0.6156,0.8,0.9955}
\color{kin}
% \pagecolor{hukamido}

\usepackage{at}%図の配置
% \usepackage{wallpaper}

\begin{document}



\at(0cm,0cm){\includegraphics[width=8cm,bb=0 0 1920 1080]{./youtube/thumbnails/templates/smart_background/数B数列.jpeg}}
{\color{orange}\bf\boldmath\huge\underline{指数型の漸化式}}\vspace{0.3zw}

\Large 
\bf\boldmath 問.次の条件を満たす数列$\{a_n\}$の一般項を求めよ.

\vspace{0.3zw}
\hspace{0.5zw}$a_1={10},\;a_{n+1}=2a_n+3^n\vspace{0.3zw}$



\vspace{0.3zw}
\hspace{0.5zw}$\left(n=1,\;2,\;3,\;\cdots \right)\vspace{0.3zw}$


\at(7.0cm,0.2cm){\small\color{bradorange}$\overset{\text{数B数列}}{\text{典型}}$}


\newpage



\at(0cm,0cm){\includegraphics[width=8cm,bb=0 0 1920 1080]{./youtube/thumbnails/templates/smart_background/数B数列.jpeg}}
{\color{orange}\bf\boldmath\huge\underline{階差型の漸化式}}\vspace{0.3zw}

\Large 
\bf\boldmath 問.次の条件を満たす数列$\{a_n\}$の一般項を求めよ.

\vspace{0.3zw}
\hspace{0.5zw}$a_1=1,\;a_{n+1}=a_n+3n+1\vspace{0.3zw}$



\vspace{0.3zw}
\hspace{0.5zw}$\left(n=1,\;2,\;3,\;\cdots \right)\vspace{0.3zw}$


\at(7.0cm,0.2cm){\small\color{bradorange}$\overset{\text{数B数列}}{\text{典型}}$}


\newpage



\at(0cm,0cm){\includegraphics[width=8cm,bb=0 0 1920 1080]{./youtube/thumbnails/templates/smart_background/数B数列.jpeg}}
{\color{orange}\bf\boldmath\LARGE\underline{和と一般項の関係式}}\vspace{0.3zw}

\Large 
\bf\boldmath 問.数列$\{a_n\}$の初項から第$n$項までの和$S_n$が
$S_n=3n-2a_n$であるとき,数列$\{a_n\}$の一般項を求めよ.
\at(7.0cm,0.2cm){\small\color{bradorange}$\overset{\text{数B数列}}{\text{典型}}$}


\newpage



\at(0cm,0cm){\includegraphics[width=8cm,bb=0 0 1920 1080]{./youtube/thumbnails/templates/smart_background/数B数列.jpeg}}
{\color{orange}\bf\boldmath\Large\underline{$x^n+y^n$が整数であることの証明}}\vspace{0.3zw}

\LARGE 
\bf\boldmath 問.$n$は自然数とする.$2$数$x,\;y$の和と積が整数のとき,$x^n+y^n$は整数であることを、数学的帰納法を用いて証明せよ.
\at(7.0cm,0.2cm){\small\color{bradorange}$\overset{\text{数B数列}}{\text{典型}}$}


\newpage



\at(0cm,0cm){\includegraphics[width=8cm,bb=0 0 1920 1080]{./youtube/thumbnails/templates/smart_background/数B数列.jpeg}}
{\color{orange}\bf\boldmath\huge\underline{一般項の推測}}\vspace{0.3zw}

\Large 
\bf\boldmath 問.次の条件で定められる数列$\{a_n\}$の一般項を推測して、それが正しいことを数学的帰納法によって証明せよ.

\vspace{0.3zw}
\hspace{0.5zw}$a_1=3,\;\left(n+1\right)a_{n+1}=a_n^2-1\vspace{0.3zw}$


\at(7.0cm,0.2cm){\small\color{bradorange}$\overset{\text{数B数列}}{\text{典型}}$}


\newpage



\at(0cm,0cm){\includegraphics[width=8cm,bb=0 0 1920 1080]{./youtube/thumbnails/templates/smart_background/数B数列.jpeg}}
{\color{orange}\bf\boldmath\Large\underline{数学的帰納法による倍数証明}}\vspace{0.3zw}

\Large 
\bf\boldmath 問.$n$を自然数とするとき、

\vspace{0.3zw}
\hspace{0.5zw}$5^{n+1}+6^{2n-1}\vspace{0.3zw}$


は${31}$の倍数であることを証明せよ.
\at(7.0cm,0.2cm){\small\color{bradorange}$\overset{\text{数B数列}}{\text{典型}}$}


\newpage



\at(0cm,0cm){\includegraphics[width=8cm,bb=0 0 1920 1080]{./youtube/thumbnails/templates/smart_background/数B数列.jpeg}}
{\color{orange}\bf\boldmath\Large\underline{数学的帰納法による不等式証明}}\vspace{0.3zw}

\LARGE 
\bf\boldmath 問.$n$が$3$以上の自然数のとき,

\vspace{0.3zw}
\hspace{0.5zw}$3^n>5n+1\vspace{0.3zw}$


を証明せよ.
\at(7.0cm,0.2cm){\small\color{bradorange}$\overset{\text{数B数列}}{\text{典型}}$}


\newpage



\at(0cm,0cm){\includegraphics[width=8cm,bb=0 0 1920 1080]{./youtube/thumbnails/templates/smart_background/数B数列.jpeg}}
{\color{orange}\bf\boldmath\Large\underline{数学的帰納法による等式証明}}\vspace{0.3zw}

\Large 
\bf\boldmath 問.次の等式を数学的帰納法によって証明せよ.

\vspace{0.3zw}
\hspace{0.5zw}$1^3+2^3+3^3+\cdots +n^3=\bunsuu{1}{4}n^2\left(n+1\right)^2\vspace{0.3zw}$


\at(7.0cm,0.2cm){\small\color{bradorange}$\overset{\text{数B数列}}{\text{典型}}$}


\newpage



\at(0cm,0cm){\includegraphics[width=8cm,bb=0 0 1920 1080]{./youtube/thumbnails/templates/smart_background/数B数列.jpeg}}
{\color{orange}\bf\boldmath\huge\underline{数学的帰納法}}\vspace{0.3zw}

\Large 
\bf\boldmath 問.すべての自然数$n$について、次の等式が成立することを証明せよ.

\vspace{0.3zw}
\hspace{0.5zw}$1^3+2^3+\cdots +n^3=\left(1+2+\cdots +n\right)^2\vspace{0.3zw}$


\at(7.0cm,0.2cm){\small\color{bradorange}$\overset{\text{数B数列}}{\text{典型}}$}


\newpage



\at(0cm,0cm){\includegraphics[width=8cm,bb=0 0 1920 1080]{./youtube/thumbnails/templates/smart_background/数B数列.jpeg}}
{\color{orange}\bf\boldmath\huge\underline{隣接$3$項間漸化式}}\vspace{0.3zw}

\large 
\bf\boldmath 問.次の条件によって定められる数列$\{a_n\}$の一般項を求めよ.\\
(1)  $a_1=1,\;a_2=2,\;a_{n+2}=a_{n+1}+6a_n$\\
(2)  $a_1=0,\;a_2=2,\;a_{n+2}-4a_{n+1}+4a_n=0$\\

\at(7.0cm,0.2cm){\small\color{bradorange}$\overset{\text{数B数列}}{\text{典型}}$}


\newpage



\at(0cm,0cm){\includegraphics[width=8cm,bb=0 0 1920 1080]{./youtube/thumbnails/templates/smart_background/数B数列.jpeg}}
{\color{orange}\bf\boldmath\huge\underline{漸化式を解く}}\vspace{0.3zw}

\Large 
\bf\boldmath 問.次の条件によって定められる数列$\{a_n\}$の一般項を求めよ.

\vspace{0.3zw}
\hspace{0.5zw}$a_1=2,\;a_{n+1}=2a_n-3\vspace{0.3zw}$



\vspace{0.3zw}
\hspace{0.5zw}$\left(n=1,\;2,\;3,\;\cdots \right)\vspace{0.3zw}$


\at(7.0cm,0.2cm){\small\color{bradorange}$\overset{\text{数B数列}}{\text{典型}}$}


\newpage



\at(0cm,0cm){\includegraphics[width=8cm,bb=0 0 1920 1080]{./youtube/thumbnails/templates/smart_background/数B数列.jpeg}}
{\color{orange}\bf\boldmath\LARGE\underline{等差数列の和の最大}}\vspace{0.3zw}

\large 
\bf\boldmath 問.初項${79},\;$公差$-2$の等差数列$\{a_n\}$について、\\
(1)  第何項が初めて負となるか.\\
(2)  初項から第$n$項までの和が最大となるか.また、そのときの和を求めよ.\\

\at(7.0cm,0.2cm){\small\color{bradorange}$\overset{\text{数B数列}}{\text{典型}}$}


\newpage



\at(0cm,0cm){\includegraphics[width=8cm,bb=0 0 1920 1080]{./youtube/thumbnails/templates/smart_background/数B数列.jpeg}}
{\color{orange}\bf\boldmath\Large\underline{等差数列をなす$3$数の和と積から}}\vspace{0.3zw}

\LARGE 
\bf\boldmath 問.等差数列をなす$3$数があって,その和が${27},\;$積が${693}$である.この$3$数を求めよ.
\at(7.0cm,0.2cm){\small\color{bradorange}$\overset{\text{数B数列}}{\text{典型}}$}


\newpage



\at(0cm,0cm){\includegraphics[width=8cm,bb=0 0 1920 1080]{./youtube/thumbnails/templates/smart_background/数B数列.jpeg}}
{\color{orange}\bf\boldmath\Large\underline{等差数列であることの証明}}\vspace{0.3zw}

\normalsize 
\bf\boldmath 問.(1)  一般項が$a_n=3n-4$で表される数列$\{a_n\}$が等差数列であることを示し,初項と公差を求めよ.\\
(2)  $\left(1\right)$の数列$\{a_n\}$の項を一つおきに取り出して並べた数列$a_1,\;a_3,\;a_5,\;\cdots$が等差数列であることを示し,初項と公差を求めよ.\\

\at(7.0cm,0.2cm){\small\color{bradorange}$\overset{\text{数B数列}}{\text{典型}}$}


\newpage



\at(0cm,0cm){\includegraphics[width=8cm,bb=0 0 1920 1080]{./youtube/thumbnails/templates/smart_background/数B数列.jpeg}}
{\color{orange}\bf\boldmath\huge\underline{$n$を含む数列の和}}\vspace{0.3zw}

\Large 
\bf\boldmath 問.次の数列$\{a_n\}$の和$S$を求めよ.

\vspace{0.3zw}
\hspace{0.5zw}$1\cdot \left(2n-1\right),\;3\cdot \left(2n-3\right),\;5\cdot \left(2n-5\right),\;\cdots ,\;\left(2n-1\right)\cdot 1\vspace{0.3zw}$


\at(7.0cm,0.2cm){\small\color{bradorange}$\overset{\text{数B数列}}{\text{典型}}$}


\newpage



\at(0cm,0cm){\includegraphics[width=8cm,bb=0 0 1920 1080]{./youtube/thumbnails/templates/smart_background/数B数列.jpeg}}
{\color{orange}\bf\boldmath\huge\underline{数列の和の和}}\vspace{0.3zw}

\LARGE 
\bf\boldmath 問.次の数列$\{a_n\}$の和$S$を求めよ.

\vspace{0.3zw}
\hspace{0.5zw}$1,\;1+2,\;1+2+3,\;\cdots ,\;1+2+\cdots +n\vspace{0.3zw}$


\at(7.0cm,0.2cm){\small\color{bradorange}$\overset{\text{数B数列}}{\text{典型}}$}


\newpage



\at(0cm,0cm){\includegraphics[width=8cm,bb=0 0 1920 1080]{./youtube/thumbnails/templates/smart_background/数B数列.jpeg}}
{\color{orange}\bf\boldmath\LARGE\underline{等比数列の和の扱い}}\vspace{0.3zw}

\Large 
\bf\boldmath 問.初項から第${10}$項までの和が$6,\;$
初項から第${20}$項までの和が${24}$である等比数列$\{a_n\}$の,
初項から第${30}$項までの和$S$を求めよ.
\at(7.0cm,0.2cm){\small\color{bradorange}$\overset{\text{数B数列}}{\text{典型}}$}


\end{document}

