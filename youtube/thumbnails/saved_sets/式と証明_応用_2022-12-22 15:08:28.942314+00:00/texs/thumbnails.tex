\documentclass[10pt,
% a4paper,
%twocolumn,
fleqn,
%landscape, 
% papersize,
dvipdfmx,
uplatex
]{jsarticle}



\def\maru#1{\textcircled{\scriptsize#1}}%丸囲み番号

% \RequirePackage[2020/09/30]{platexrelease}

%太字設定
\usepackage[deluxe]{otf}

\usepackage{emathEy}

\usepackage[g]{esvect}

%大きな文字
\usepackage{fix-cm}

%定理環境
\usepackage{emathThm}
%\theoremstyle{boxed}
\theorembodyfont{\normalfont}
\newtheorem{Question}{問題}[subsection]
\newtheorem{Q}{}[subsection]
\newtheorem{question}[Question]{}
\newtheorem{quuestion}{}[subsection]

%セクション,大問番号のデザイン
\renewcommand{\labelenumi}{(\arabic{enumi})}
\renewcommand{\theenumii}{\alph{enumii})}
\renewcommand{\thesection}{第\arabic{section}章}

%用紙サイズの詳細設定
\usepackage{bxpapersize}
\papersizesetup{size={80mm,45mm}}
\usepackage[top=0.7zw,bottom=0truemm,left=3truemm,right=133truemm]{geometry}
\usepackage[dvipdfmx]{graphicx}

%余白など
\usepackage{setspace} % 行間
\setlength{\mathindent}{1zw}
\setlength\parindent{0pt}


%色カラーに関する設定
\usepackage{color}
\definecolor{shiro}{rgb}{0.95703125,0.87109375,0.7421875}
\definecolor{kin}{rgb}{0.95703125,0.87109375,0.7421875}
\definecolor{orange}{rgb}{1,0.7,0.2}
\definecolor{bradorange}{rgb}{1,0.5,0}
\definecolor{pink}{rgb}{0.9176,0.5686,0.5960}
\definecolor{mizu}{rgb}{0.6156,0.8,0.9955}
\color{kin}
% \pagecolor{hukamido}

\usepackage{at}%図の配置
% \usepackage{wallpaper}

\begin{document}



\at(0cm,0cm){\includegraphics[width=8cm,bb=0 0 1920 1080]{./youtube/thumbnails/templates/smart_background/数II式と証明.jpeg}}
{\color{orange}\bf\boldmath\LARGE\underline{コーシー 4つの証明}}\vspace{0.3zw}

\large 
\bf\boldmath 問.次の不等式を証明せよ.\\
\hfill また,等号成立条件を求めよ.

\Large
\vspace{0.6zw}
\hspace{0.2zw}$\left(a^2+b^2+c^2\right)\left(x^2+y^2+z^2\right)$\\
\hfill $\geqq \left(ax+by+cz\right)^2\vspace{0.3zw}$


\at(6.6cm,0.2cm){\small\color{bradorange}$\overset{\text{数Ⅱ式と証明}}{\text{応用}}$}


\newpage



\at(0cm,0cm){\includegraphics[width=8cm,bb=0 0 1920 1080]{./youtube/thumbnails/templates/smart_background/数II式と証明.jpeg}}
{\color{orange}\bf\boldmath\LARGE\underline{コーシー 4つの証明}}\vspace{0.3zw}

\large 
\bf\boldmath 問.次の不等式を証明せよ.\\
\hfill また,等号成立条件を求めよ.

\LARGE
\vspace{0.3zw}
\hspace{0.5zw}$\left(a^2+b^2\right)\left(x^2+y^2\right)$\\
\hfill $\geqq \left(ax+by\right)^2\vspace{0.3zw}$


\at(6.6cm,0.2cm){\small\color{bradorange}$\overset{\text{数Ⅱ式と証明}}{\text{応用}}$}


\newpage



\at(0cm,0cm){\includegraphics[width=8cm,bb=0 0 1920 1080]{./youtube/thumbnails/templates/smart_background/数II式と証明.jpeg}}
{\color{orange}\bf\boldmath\Large\underline{相加相乗不等式の等号成立}}\vspace{0.3zw}

\large
\bf\boldmath 問.$x>0,\;y>0$とする.

\huge 
\vspace{0.3zw}
\hspace{0.3zw}$\left(x+\bunsuu{1}{y}\right)\left(y+\bunsuu{4}{x}\right)$
\vspace{0.3zw}

\large
\hfill の最小値を求めよ.
\at(6.6cm,0.2cm){\small\color{bradorange}$\overset{\text{数Ⅱ式と証明}}{\text{応用}}$}


\newpage



\at(0cm,0cm){\includegraphics[width=8cm,bb=0 0 1920 1080]{./youtube/thumbnails/templates/smart_background/数II式と証明.jpeg}}
{\color{orange}\bf\boldmath\huge\underline{分数式の最大値}}\vspace{0.3zw}

\normalsize 
\bf\boldmath 問.$x>0$とする.次の最大値を求めよ.\vspace{0.1zw}

\large
(1)  \Large$\bunsuu{1}{{}x^2-x+1}$\hspace{0.2zw}
\large
(2)  \Large$\bunsuu{x}{{}x^2-x+1}$\vspace{-0.1zw}\\
\large
\hspace{4zw}(3)  \LARGE $\bunsuu{{x^2}}{{x^2-x+1}}$\\

\at(6.6cm,0.2cm){\small\color{bradorange}$\overset{\text{数Ⅱ式と証明}}{\text{応用}}$}


\newpage



\at(0cm,0cm){\includegraphics[width=8cm,bb=0 0 1920 1080]{./youtube/thumbnails/templates/smart_background/数II式と証明.jpeg}}
{\color{orange}\bf\boldmath\huge\underline{分数関数の最小値}}\vspace{0.3zw}

\normalsize
\bf\boldmath 問.$x>1$とする.関数

\Huge 
\vspace{0.2zw}
\hspace{0.3zw}$y=\bunsuu{{x^2-x+1}}{{x-1}}$
\vspace{0.1zw}

\normalsize
\hfill の最小値を求めよ.
\at(6.6cm,0.2cm){\small\color{bradorange}$\overset{\text{数Ⅱ式と証明}}{\text{応用}}$}


\newpage



\at(0cm,0cm){\includegraphics[width=8cm,bb=0 0 1920 1080]{./youtube/thumbnails/templates/smart_background/数II式と証明.jpeg}}
{\color{orange}\bf\boldmath\LARGE\underline{相加相乗不等式と最小値}}\vspace{0.3zw}

\normalsize
\bf\boldmath 問.$x>0$のとき,次の関数の最小値を求めよ.

\LARGE
\vspace{0.1zw}
(1)  $y=2x+\bunsuu{1}{x}$\vspace{-0.3zw}\\
\huge
(2)  $y=2x+\bunsuu{1}{{}x^2}$\\

\at(6.6cm,0.2cm){\small\color{bradorange}$\overset{\text{数Ⅱ式と証明}}{\text{応用}}$}


\newpage



\at(0cm,0cm){\includegraphics[width=8cm,bb=0 0 1920 1080]{./youtube/thumbnails/templates/smart_background/数II式と証明.jpeg}}
{\color{orange}\bf\boldmath\LARGE\underline{相加相乗不等式の証明}}\vspace{0.1zw}

\scriptsize 
\bf\boldmath 問.正の数$a,\;b,\;c,\;d$に対して,次の不等式を証明せよ.また,\\
\hfill 等号成立条件を求めよ.

\normalsize
\vspace{-1zw}
(1)  $\bunsuu{{a+b}}{2}\geqq \sqrt {ab}$\vspace{0.1zw}\\
\large
(2)  $\bunsuu{{a+b+c}}{3}\geqq \sqrt[3]{abc}$\vspace{0.1zw}\\
\Large
(3)  $\bunsuu{{a+b+c+d}}{4}\geqq \sqrt[4]{abcd}$\\

\at(6.6cm,0.2cm){\small\color{bradorange}$\overset{\text{数Ⅱ式と証明}}{\text{応用}}$}


\end{document}

