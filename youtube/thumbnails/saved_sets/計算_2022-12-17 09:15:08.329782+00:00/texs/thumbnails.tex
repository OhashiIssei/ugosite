\documentclass[10pt,
% a4paper,
%twocolumn,
fleqn,
%landscape, 
% papersize,
dvipdfmx,
uplatex
]{jsarticle}



\def\maru#1{\textcircled{\scriptsize#1}}%丸囲み番号

% \RequirePackage[2020/09/30]{platexrelease}

%太字設定
\usepackage[deluxe]{otf}

\usepackage{emathEy}

\usepackage[g]{esvect}

%大きな文字
\usepackage{fix-cm}

%定理環境
\usepackage{emathThm}
%\theoremstyle{boxed}
\theorembodyfont{\normalfont}
\newtheorem{Question}{問題}[subsection]
\newtheorem{Q}{}[subsection]
\newtheorem{question}[Question]{}
\newtheorem{quuestion}{}[subsection]

%セクション,大問番号のデザイン
\renewcommand{\labelenumi}{(\arabic{enumi})}
\renewcommand{\theenumii}{\alph{enumii})}
\renewcommand{\thesection}{第\arabic{section}章}

%用紙サイズの詳細設定
\usepackage{bxpapersize}
\papersizesetup{size={80mm,45mm}}
\usepackage[top=0.7zw,bottom=0truemm,left=3truemm,right=133truemm]{geometry}
\usepackage[dvipdfmx]{graphicx}

%余白など
\usepackage{setspace} % 行間
\setlength{\mathindent}{1zw}
\setlength\parindent{0pt}


%色カラーに関する設定
\usepackage{color}
\definecolor{shiro}{rgb}{0.95703125,0.87109375,0.7421875}
\definecolor{kin}{rgb}{0.95703125,0.87109375,0.7421875}
\definecolor{orange}{rgb}{1,0.7,0.2}
\definecolor{bradorange}{rgb}{1,0.5,0}
\definecolor{pink}{rgb}{0.9176,0.5686,0.5960}
\definecolor{mizu}{rgb}{0.6156,0.8,0.9955}
\color{kin}
% \pagecolor{hukamido}

\usepackage{at}%図の配置
% \usepackage{wallpaper}

\begin{document}



\at(0cm,0cm){\includegraphics[width=8cm,bb=0 0 1920 1080]{./youtube/thumbnails/templates/smart_background/数II微積.jpeg}}
{\color{orange}\bf\boldmath\LARGE\underline{絶対値の定積分}}\vspace{0.3zw}

\large 
\bf\boldmath 問.定積分

\huge 
\vspace{0.3zw}
\hspace{0.5zw}$\displaystyle\int_{-3}^3\zettaiti{x^2+x-2}\;dx\vspace{0.3zw}$

\large
\hfill 
の値を求めよ.
\at(7.0cm,0.2cm){\small\color{bradorange}$\overset{\text{数Ⅱ微積}}{\text{計算}}$}

\newpage

\at(0cm,0cm){\includegraphics[width=8cm,bb=0 0 1920 1080]{./youtube/thumbnails/templates/smart_background/数II微積.jpeg}}
{\color{orange}\bf\boldmath\huge\underline{$\sqrt x$の定積分}}\vspace{0.3zw}

\large 
\bf\boldmath 問.

\Huge 
\vspace{-0.2zw}
\hspace{1zw}$\displaystyle\int_0^3\sqrt x\;dx$
\vspace{-0.5zw}

\large
\hfill
を求めよ.
\at(7.0cm,0.2cm){\small\color{bradorange}$\overset{\text{数Ⅱ微積}}{\text{計算}}$}

\newpage

\at(0cm,0cm){\includegraphics[width=8cm,bb=0 0 1920 1080]{./youtube/thumbnails/templates/smart_background/数II微積.jpeg}}
{\color{orange}\bf\boldmath\LARGE\underline{$\sqrt {r^2-x^2}$の定積分}}\vspace{0.3zw}

\large
\bf\boldmath 問.

\Huge 
\vspace{-0.2zw}
\hspace{0.5zw}$\displaystyle\int_0^1\sqrt {4-x^2}\;dx$
\vspace{-0.2zw}

\large
\hfill 
を求めよ.
\at(7.0cm,0.2cm){\small\color{bradorange}$\overset{\text{数Ⅱ微積}}{\text{計算}}$}

\newpage

\at(0cm,0cm){\includegraphics[width=8cm,bb=0 0 1920 1080]{./youtube/thumbnails/templates/smart_background/数II微積.jpeg}}
{\color{orange}\bf\boldmath\LARGE\underline{$\left(ax+b\right)^n$の定積分}}\vspace{0.3zw}

\large 
\bf\boldmath 問.

\Huge 
\vspace{-0.2zw}
\hspace{0.5zw}$\displaystyle\int_0^1\left(2x-1\right)^5dx$
\vspace{-0.2zw}

\large 
\hfill を求めよ.
\at(7.0cm,0.2cm){\small\color{bradorange}$\overset{\text{数Ⅱ微積}}{\text{計算}}$}

\newpage

\at(0cm,0cm){\includegraphics[width=8cm,bb=0 0 1920 1080]{./youtube/thumbnails/templates/smart_background/数II微積.jpeg}}
{\color{orange}\bf\boldmath\LARGE\underline{定積分の基本計算}}\vspace{0.3zw}

\large 
\bf\boldmath 問.

\Huge 
\vspace{-0.2zw}
\hspace{0.2zw}$\displaystyle\int_1^5\left(x^2-3x\right)dx$
\vspace{-0.5zw}

\large
\hfill を求めよ.
\at(7.0cm,0.2cm){\small\color{bradorange}$\overset{\text{数Ⅱ微積}}{\text{計算}}$}

\newpage

\at(0cm,0cm){\includegraphics[width=8cm,bb=0 0 1920 1080]{./youtube/thumbnails/templates/smart_background/数II微積.jpeg}}
{\color{orange}\bf\boldmath\LARGE\underline{$3$次関数の極値の差}}\vspace{0.5zw}

\large 
\bf\boldmath 問.関数$f\left(x\right)=x^3+ax^2+x$の

\huge
\vspace{0.5zw}
\hspace{0.1zw}極大値と極小値の差が$4$

\vspace{0.7zw}
\large 
\hfill であるような$a$の値を求めよ.

\at(7.0cm,0.2cm){\small\color{bradorange}$\overset{\text{数Ⅱ微積}}{\text{計算}}$}

\newpage

\at(0cm,0cm){\includegraphics[width=8cm,bb=0 0 1920 1080]{./youtube/thumbnails/templates/smart_background/数II微積.jpeg}}
{\color{orange}\bf\boldmath\LARGE\underline{次数下げによる極値計算}}\vspace{0.3zw}

\Large 
\bf\boldmath 問.次の関数の極値を求めよ.

\Huge
\vspace{0.5zw}
\hspace{0.2zw}$y=x^3+x^2-2x\vspace{0.3zw}$

\at(7.0cm,0.2cm){\small\color{bradorange}$\overset{\text{数Ⅱ微積}}{\text{計算}}$}

\newpage

\at(0cm,0cm){\includegraphics[width=8cm,bb=0 0 1920 1080]{./youtube/thumbnails/templates/smart_background/数II微積.jpeg}}
{\color{orange}\bf\boldmath\LARGE\underline{$3$次関数の極値の和と差}}\vspace{0.3zw}

\large 
\bf\boldmath 問.関数$f\left(x\right)=x^3-3ax^2+3bx$の

\LARGE
極大値と極小値の和および差がそれぞれ$-{18},\;{32}$である

\large
\vspace{0.2zw}
\hfill とき,定数$a,\;b$の値を定めよ.
\at(7.0cm,0.2cm){\small\color{bradorange}$\overset{\text{数Ⅱ微積}}{\text{計算}}$}

\newpage

\at(0cm,0cm){\includegraphics[width=8cm,bb=0 0 1920 1080]{./youtube/thumbnails/templates/smart_background/数II微積.jpeg}}
{\color{orange}\bf\boldmath\huge\underline{積の微分公式}}\vspace{0.3zw}

\LARGE 
\bf\boldmath 問.次の関数を微分せよ.

\vspace{0.5zw}
\hspace{0.5zw}$y=\left(2x+1\right)^2\left(3x^2-2\right)\vspace{0.3zw}$

\at(7.0cm,0.2cm){\small\color{bradorange}$\overset{\text{数Ⅱ微積}}{\text{計算}}$}

\newpage

\at(0cm,0cm){\includegraphics[width=8cm,bb=0 0 1920 1080]{./youtube/thumbnails/templates/smart_background/数II微積.jpeg}}
{\color{orange}\bf\boldmath\huge\underline{合成関数の微分}}\vspace{0.3zw}

\LARGE 
\bf\boldmath 問.次の関数を微分せよ.

\HUGE
\vspace{0.1zw}
\hspace{0.5zw}$y=\left(2x+1\right)^3\vspace{0.3zw}$

\at(7.0cm,0.2cm){\small\color{bradorange}$\overset{\text{数Ⅱ微積}}{\text{計算}}$}

\newpage

\at(0cm,0cm){\includegraphics[width=8cm,bb=0 0 1920 1080]{./youtube/thumbnails/templates/smart_background/指数対数.jpeg}}
{\color{orange}\bf\boldmath\LARGE\underline{指数計算に対数を利用}}\vspace{0.3zw}

\Large 
\bf\boldmath 問.\LARGE$2^x=5^y={10}^z$\Large\;のとき,

\HUGE
\vspace{-0.3zw}
\hspace{0.5zw}$xy-yz-zx$\vspace{0.3zw}

\Large 
\hfill の値を求めよ.
\at(7.0cm,0.2cm){\small\color{bradorange}$\overset{\text{指数対数}}{\text{計算}}$}

\newpage

\at(0cm,0cm){\includegraphics[width=8cm,bb=0 0 1920 1080]{./youtube/thumbnails/templates/smart_background/指数対数.jpeg}}
{\color{orange}\bf\boldmath\huge\underline{底の変換公式}}\vspace{0.3zw}

\Large 
\bf\boldmath 問.次の式の値を求めよ.

\huge
\vspace{0.1zw}
\hspace{0.2zw}$\left(\log _29+\log _83\right)$\\
\hfill$\times\left(\log _32+\log _94\right)\vspace{0.3zw}$

\at(7.0cm,0.2cm){\small\color{bradorange}$\overset{\text{指数対数}}{\text{計算}}$}

\newpage

\at(0cm,0cm){\includegraphics[width=8cm,bb=0 0 1920 1080]{./youtube/thumbnails/templates/smart_background/指数対数.jpeg}}
{\color{orange}\bf\boldmath\LARGE\underline{無理数乗の大小比較}}\vspace{0.1zw}

\tiny
\bf\boldmath 問.次の$\fbox{\phantom{J}}$に$=,\;<,\;>$のいずれかを入れよ.\\
(1)  $\left(\sqrt 2\right)^2\fbox{\phantom{J}}\log _{\sqrt 2}2$

\scriptsize
(2)  $\left(\sqrt 2\right)^4\fbox{\phantom{J}}\log _{\sqrt 2}4$

\small
(3)  $\left(\sqrt 2\right)^8\fbox{\phantom{J}}\log _{\sqrt 2}8$

\vspace{-0.2zw}
\LARGE
(4)  $\left(\sqrt 2\right)^{\sqrt 8}\fbox{\phantom{J}}\log _{\sqrt 2}\sqrt 8$

\at(7.0cm,0.2cm){\small\color{bradorange}$\overset{\text{指数対数}}{\text{計算}}$}

\newpage

\at(0cm,0cm){\includegraphics[width=8cm,bb=0 0 1920 1080]{./youtube/thumbnails/templates/smart_background/指数対数.jpeg}}
{\color{orange}\bf\boldmath\huge\underline{対数計算}}\vspace{0.3zw}

\large 
\bf\boldmath 問.次の式の値を求めよ.

\LARGE
\vspace{0.2zw}
\hspace{0.5zw}$\log _5\sqrt 2-\bunsuu{1}{2}\log _5\bunsuu{1}{3}$\\
\hfill $-\bunsuu{3}{2}\log _5\sqrt[3]{{30}}\vspace{0.3zw}$

\at(7.0cm,0.2cm){\small\color{bradorange}$\overset{\text{指数対数}}{\text{計算}}$}

\newpage

\at(0cm,0cm){\includegraphics[width=8cm,bb=0 0 1920 1080]{./youtube/thumbnails/templates/smart_background/指数対数.jpeg}}
{\color{orange}\bf\boldmath\huge\underline{肩の上の対数}}\vspace{0.3zw}

\large 
\bf\boldmath 問.次の式の値を求めよ.

\fontsize{60}{0} \selectfont 
\vspace{0.2zw}
\hspace{0.5zw}$3^{\log _98}\vspace{0.3zw}$

\at(7.0cm,0.2cm){\small\color{bradorange}$\overset{\text{指数対数}}{\text{計算}}$}

\newpage

\at(0cm,0cm){\includegraphics[width=8cm,bb=0 0 1920 1080]{./youtube/thumbnails/templates/smart_background/指数対数.jpeg}}
{\color{orange}\bf\boldmath\huge\underline{$3$乗根の有理化}}\vspace{0.2zw}

\large
\bf\boldmath 問.

\fontsize{32}{0} \selectfont
\vspace{-0.6zw}
\hspace{0.8zw} $\bunsuu{5}{{}\sqrt[3]4+1}\vspace{0.3zw}$

\large
\vspace{-0.5zw}
\hfill
の分母を有理化せよ.

\at(7.0cm,0.2cm){\small\color{bradorange}$\overset{\text{指数対数}}{\text{計算}}$}

\newpage

\at(0cm,0cm){\includegraphics[width=8cm,bb=0 0 1920 1080]{./youtube/thumbnails/templates/smart_background/指数対数.jpeg}}
{\color{orange}\bf\boldmath\huge\underline{無理数乗の計算}}\vspace{0.3zw}

\large 
\bf\boldmath 問.次の式の値を求めよ.

\Huge
\vspace{0.5zw}
\hspace{0.5zw}$6^{\sqrt 6}×2^{\sqrt 6}÷3^{\sqrt 6}\vspace{0.3zw}$

\at(7.0cm,0.2cm){\small\color{bradorange}$\overset{\text{指数対数}}{\text{計算}}$}

\newpage

\at(0cm,0cm){\includegraphics[width=8cm,bb=0 0 1920 1080]{./youtube/thumbnails/templates/smart_background/指数対数.jpeg}}
{\color{orange}\bf\boldmath\huge\underline{累乗根の計算}}\vspace{0.3zw}

\large 
\bf\boldmath 問.次の式の値を求めよ.

\huge
\vspace{0.5zw}
\hspace{0.2zw}$\bunsuu{{\sqrt[3]4}}{{\sqrt {16}}}÷\bunsuu{{\sqrt {64}}}{{\sqrt[3]{64}}}×\bunsuu{{\sqrt {32}}}{{\sqrt[3]{32}}}\vspace{0.3zw}$

\at(7.0cm,0.2cm){\small\color{bradorange}$\overset{\text{指数対数}}{\text{計算}}$}

\newpage

\at(0cm,0cm){\includegraphics[width=8cm,bb=0 0 1920 1080]{./youtube/thumbnails/templates/smart_background/指数対数.jpeg}}
{\color{orange}\bf\boldmath\huge\underline{分数の分数乗}}\vspace{0.3zw}

\large 
\bf\boldmath 問.\vspace{-1zw}

\fontsize{35}{0} \selectfont
\vspace{-0.5zw}
\hspace{1.2zw}$\left(\bunsuu{{{27}}}{8}\right)^{\frac{2}{3}}\vspace{0.3zw}$

\large
\vspace{-1.7zw}
\hfill の値を求めよ.

\at(7.0cm,0.2cm){\small\color{bradorange}$\overset{\text{指数対数}}{\text{計算}}$}

\newpage

\at(0cm,0cm){\includegraphics[width=8cm,bb=0 0 1920 1080]{./youtube/thumbnails/templates/smart_background/数II式と証明.jpeg}}
{\color{orange}\bf\boldmath\huge\underline{連分数の計算}}\vspace{0.3zw}

\normalsize 
\bf\boldmath 問.

\LARGE
\vspace{-1.3zw}
\hspace{1zw}$\bunsuu{1}{1+\bunsuu{1}{1+\bunsuu{1}{1+\bunsuu{1}{x}}}}\vspace{-1zw}$

\normalsize 
\hfill を計算せよ.

\at(6.6cm,0.2cm){\small\color{bradorange}$\overset{\text{数Ⅱ式と証明}}{\text{計算}}$}

\newpage

\at(0cm,0cm){\includegraphics[width=8cm,bb=0 0 1920 1080]{./youtube/thumbnails/templates/smart_background/数II式と証明.jpeg}}
{\color{orange}\bf\boldmath\huge\underline{繁分数の計算}}\vspace{0.3zw}

\large 
\bf\boldmath 問.

\LARGE
\vspace{-0.5zw}
\hspace{1zw}$\bunsuu{{\bunsuu{{1+x}}{{1-x}}-\bunsuu{{1-x}}{{1+x}}}}{{\bunsuu{{1+x}}{{1-x}}+\bunsuu{{1-x}}{{1+x}}}}\vspace{0.3zw}$

\vspace{-1zw}
\normalsize 
\hfill を計算せよ.

\at(6.6cm,0.2cm){\small\color{bradorange}$\overset{\text{数Ⅱ式と証明}}{\text{計算}}$}

\newpage

\at(0cm,0cm){\includegraphics[width=8cm,bb=0 0 1920 1080]{./youtube/thumbnails/templates/smart_background/数II式と証明.jpeg}}
{\color{orange}\bf\boldmath\huge\underline{分数式の通分}}\vspace{0.3zw}

\large 
\bf\boldmath 問.次の式を計算せよ.

\Large 
\vspace{1.3zw}
\hspace{0.5zw}$\bunsuu{{2x-1}}{{x^2-3x+2}}-\bunsuu{{x-5}}{{x^2-5x+6}}\vspace{0.3zw}$

\at(6.6cm,0.2cm){\small\color{bradorange}$\overset{\text{数Ⅱ式と証明}}{\text{計算}}$}

\newpage

\at(0cm,0cm){\includegraphics[width=8cm,bb=0 0 1920 1080]{./youtube/thumbnails/templates/smart_background/数I数と式.jpeg}}
{\color{orange}\bf\boldmath\huge\underline{$3$元対称式の値}}\vspace{0.3zw}

\normalsize
\bf\boldmath 問.$x+y+z=2\sqrt 3+1$,\;\\
\hfill $xy+yz+zx=2\sqrt 3-1$,\;$xyz=-1$のとき,

\fontsize{35}{0} \selectfont
\vspace{0.1zw}
\hspace{0.1zw} $x^4+y^4+z^4$
\vspace{0.05zw}
   
\large
\hfill の式の値を求めよ.

\at(6.8cm,0.2cm){\small\color{bradorange}$\overset{\text{数I数と式}}{\text{計算}}$}

\newpage

\at(0cm,0cm){\includegraphics[width=8cm,bb=0 0 1920 1080]{./youtube/thumbnails/templates/smart_background/数I数と式.jpeg}}
{\color{orange}\bf\boldmath\huge\underline{$3$元対称式の値}}\vspace{0.3zw}

\normalsize
\bf\boldmath 問.$x+y+z=2\sqrt 3+1$,\;\\
\hfill $xy+yz+zx=2\sqrt 3-1$,\;$xyz=-1$のとき,

\fontsize{35}{0} \selectfont
\vspace{0.1zw}
\hspace{0.1zw} $x^3+y^3+z^3$
\vspace{0.05zw}

\large
\hfill の式の値を求めよ.

\at(6.8cm,0.2cm){\small\color{bradorange}$\overset{\text{数I数と式}}{\text{計算}}$}

\newpage

\at(0cm,0cm){\includegraphics[width=8cm,bb=0 0 1920 1080]{./youtube/thumbnails/templates/smart_background/数I数と式.jpeg}}
{\color{orange}\bf\boldmath\huge\underline{$3$元対称式の値}}\vspace{0.3zw}

\normalsize
\bf\boldmath 問.$x+y+z=2\sqrt 3+1$,\;\\
\hfill $xy+yz+zx=2\sqrt 3-1$,\;$xyz=-1$のとき,

\fontsize{35}{0} \selectfont
\vspace{0.1zw}
\hspace{0.1zw} $x^2+y^2+z^2$
\vspace{0.05zw}

\large
\hfill の式の値を求めよ.

\at(6.8cm,0.2cm){\small\color{bradorange}$\overset{\text{数I数と式}}{\text{計算}}$}

\newpage

\at(0cm,0cm){\includegraphics[width=8cm,bb=0 0 1920 1080]{./youtube/thumbnails/templates/smart_background/数I数と式.jpeg}}
{\color{orange}\bf\boldmath\huge\underline{因数分解$〜$上級$〜$}}\vspace{0.5zw}

\Large 
\bf\boldmath 

\LARGE
$a^4+b^4+c^4$\\
\hfill$-2a^2b^2-2b^2c^2-2c^2a^2$\vspace{0.5zw}

\Large
\hfill を因数分解せよ.
\at(6.8cm,0.2cm){\small\color{bradorange}$\overset{\text{数I数と式}}{\text{計算}}$}

\newpage

\at(0cm,0cm){\includegraphics[width=8cm,bb=0 0 1920 1080]{./youtube/thumbnails/templates/smart_background/数I数と式.jpeg}}
{\color{orange}\bf\boldmath\huge\underline{$x^n-y^n$の因数分解}}\vspace{0.3zw}

\Huge 
\bf\boldmath 
\hspace{0.2zw}$x^2-y^2,\;x^3-y^3,\;$\\
\hspace{0.2zw}$x^4-y^4,\;x^5-y^5$

\Large
\vspace{0.2zw}
\hfill を因数分解せよ.
\at(6.8cm,0.2cm){\small\color{bradorange}$\overset{\text{数I数と式}}{\text{計算}}$}

\newpage

\at(0cm,0cm){\includegraphics[width=8cm,bb=0 0 1920 1080]{./youtube/thumbnails/templates/smart_background/数I数と式.jpeg}}
{\color{orange}\bf\boldmath\huge\underline{$3$乗の展開}}\vspace{0.3zw}

\large 
\bf\boldmath 問.次の式を展開せよ.

\Huge
\vspace{-0.3zw}
(1)  $\left(a+b\right)^3$\vspace{-0.2zw}\\
(2)  \HUGE $\left(a+b+c\right)^3$\\

\at(6.8cm,0.2cm){\small\color{bradorange}$\overset{\text{数I数と式}}{\text{計算}}$}

\newpage

\at(0cm,0cm){\includegraphics[width=8cm,bb=0 0 1920 1080]{./youtube/thumbnails/templates/smart_background/数I数と式.jpeg}}
{\color{orange}\bf\boldmath\huge\underline{$2$乗の展開}}\vspace{0.1zw}

\large 
\bf\boldmath 問.次の式を展開せよ.

\LARGE 
\vspace{-0.3zw}
(1)  $\left(a+b\right)^2$\\
(2)  $\left(a+b+c\right)^2$\\
(3)  $\left(a+b+c+d\right)^2$\\

\at(6.8cm,0.2cm){\small\color{bradorange}$\overset{\text{数I数と式}}{\text{計算}}$}

\newpage

\at(0cm,0cm){\includegraphics[width=8cm,bb=0 0 1920 1080]{./youtube/thumbnails/templates/smart_background/数I数と式.jpeg}}
{\color{orange}\bf\boldmath\huge\underline{$3$次式の因数分解}}\vspace{0.5zw}

\LARGE 
\bf\boldmath $a^3+b^3+c^3+d^3$

\Large
\hfill $-3abc-3bcd-3cda-3dab$

\Large
\vspace{1zw}
\hfill を因数分解せよ.
\at(6.8cm,0.2cm){\small\color{bradorange}$\overset{\text{数I数と式}}{\text{計算}}$}

\newpage

\at(0cm,0cm){\includegraphics[width=8cm,bb=0 0 1920 1080]{./youtube/thumbnails/templates/smart_background/数I数と式.jpeg}}
{\color{orange}\bf\boldmath\huge\underline{因数分解〜中級〜}}\vspace{0.5zw}

\fontsize{50}{0} \selectfont
\bf\boldmath 
\hspace{0.5zw}$x^6-y^6$

\Large
\vspace{1zw}
\hfill を因数分解せよ.
\at(6.8cm,0.2cm){\small\color{bradorange}$\overset{\text{数I数と式}}{\text{計算}}$}

\newpage

\at(0cm,0cm){\includegraphics[width=8cm,bb=0 0 1920 1080]{./youtube/thumbnails/templates/smart_background/数I数と式.jpeg}}
{\color{orange}\bf\boldmath\huge\underline{因数分解〜中級〜}}\vspace{0.5zw}

\Huge 
\bf\boldmath $x^4-{13}x^2y^2+4y^4$

\Large
\vspace{1zw}
\hfill を因数分解せよ.
\at(6.8cm,0.2cm){\small\color{bradorange}$\overset{\text{数I数と式}}{\text{計算}}$}

\newpage

\at(0cm,0cm){\includegraphics[width=8cm,bb=0 0 1920 1080]{./youtube/thumbnails/templates/smart_background/数I数と式.jpeg}}
{\color{orange}\bf\boldmath\huge\underline{$3$次式の因数分解}}\vspace{0.5zw}

\Huge 
\bf\boldmath $a^3+b^3+c^3-3abc$

\Large
\vspace{1zw}
\hfill を因数分解せよ.
\at(6.8cm,0.2cm){\small\color{bradorange}$\overset{\text{数I数と式}}{\text{計算}}$}

\newpage

\at(0cm,0cm){\includegraphics[width=8cm,bb=0 0 1920 1080]{./youtube/thumbnails/templates/smart_background/数I数と式.jpeg}}
{\color{orange}\bf\boldmath\huge\underline{$3$次式の因数分解}}\vspace{0.5zw}

\fontsize{50}{0} \selectfont
\bf\boldmath 
\hspace{0.5zw}$a^3+b^3$

\Large
\vspace{1zw}
\hfill を因数分解せよ.
\at(6.8cm,0.2cm){\small\color{bradorange}$\overset{\text{数I数と式}}{\text{計算}}$}

\newpage

\at(0cm,0cm){\includegraphics[width=8cm,bb=0 0 1920 1080]{./youtube/thumbnails/templates/smart_background/数I数と式.jpeg}}
{\color{orange}\bf\boldmath\huge\underline{対称式の計算}}\vspace{0.3zw}

\Large 
\bf\boldmath 問.$x+y={10},\;xy=1$のとき,

\Huge
\hspace{0.2zw}$x^4+y^4,\;x^5+y^5$\vspace{0.3zw}

\Large
\hfill の値を求めよ.
\at(6.8cm,0.2cm){\small\color{bradorange}$\overset{\text{数I数と式}}{\text{計算}}$}

\newpage

\at(0cm,0cm){\includegraphics[width=8cm,bb=0 0 1920 1080]{./youtube/thumbnails/templates/smart_background/数I数と式.jpeg}}
{\color{orange}\bf\boldmath\huge\underline{対称式の計算}}\vspace{0.3zw}

\Large 
\bf\boldmath 問.$x+y={10},\;xy=1$のとき,

\fontsize{40}{0} \selectfont
\bf\boldmath 
\vspace{0.1zw}
\hspace{1zw}$x^3+y^3$\vspace{0.3zw}

\Large
\hfill の値を求めよ.
\at(6.8cm,0.2cm){\small\color{bradorange}$\overset{\text{数I数と式}}{\text{計算}}$}

\newpage

\at(0cm,0cm){\includegraphics[width=8cm,bb=0 0 1920 1080]{./youtube/thumbnails/templates/smart_background/数I数と式.jpeg}}
{\color{orange}\bf\boldmath\huge\underline{対称式の計算}}\vspace{0.3zw}

\Large 
\bf\boldmath 問.$x+y={10},\;xy=1$のとき,

\fontsize{40}{0} \selectfont
\bf\boldmath 
\vspace{0.1zw}
\hspace{1zw}$x^2+y^2$\vspace{0.3zw}

\Large
\hfill の値を求めよ.
\at(6.8cm,0.2cm){\small\color{bradorange}$\overset{\text{数I数と式}}{\text{計算}}$}



\end{document}

