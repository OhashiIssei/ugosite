\documentclass[10pt,
% a4paper,
%twocolumn,
fleqn,
%landscape, 
% papersize,
dvipdfmx,
uplatex
]{jsarticle}



\def\maru#1{\textcircled{\scriptsize#1}}%丸囲み番号

% \RequirePackage[2020/09/30]{platexrelease}

%太字設定
\usepackage[deluxe]{otf}

\usepackage{emathEy}

\usepackage[g]{esvect}

%大きな文字
\usepackage{fix-cm}

%定理環境
\usepackage{emathThm}
%\theoremstyle{boxed}
\theorembodyfont{\normalfont}
\newtheorem{Question}{問題}[subsection]
\newtheorem{Q}{}[subsection]
\newtheorem{question}[Question]{}
\newtheorem{quuestion}{}[subsection]

%セクション,大問番号のデザイン
\renewcommand{\labelenumi}{(\arabic{enumi})}
\renewcommand{\theenumii}{\alph{enumii})}
\renewcommand{\thesection}{第\arabic{section}章}

%用紙サイズの詳細設定
\usepackage{bxpapersize}
\papersizesetup{size={80mm,45mm}}
\usepackage[top=0.7zw,bottom=0truemm,left=3truemm,right=133truemm]{geometry}
\usepackage[dvipdfmx]{graphicx}

%余白など
\usepackage{setspace} % 行間
\setlength{\mathindent}{1zw}
\setlength\parindent{0pt}


%色カラーに関する設定
\usepackage{color}
\definecolor{shiro}{rgb}{0.95703125,0.87109375,0.7421875}
\definecolor{kin}{rgb}{0.95703125,0.87109375,0.7421875}
\definecolor{orange}{rgb}{1,0.7,0.2}
\definecolor{bradorange}{rgb}{1,0.5,0}
\definecolor{pink}{rgb}{0.9176,0.5686,0.5960}
\definecolor{mizu}{rgb}{0.6156,0.8,0.9955}
\color{kin}
% \pagecolor{hukamido}

\usepackage{at}%図の配置
% \usepackage{wallpaper}

\begin{document}

\at(0cm,0cm){\includegraphics[width=8cm,bb=0 0 1920 1080]{./youtube/thumbnails/templates/smart_background/指数対数.jpeg}}
{\color{orange}\bf\boldmath\LARGE\underline{指数計算に対数を利用}}\vspace{0.3zw}

\Large 
\bf\boldmath 問.\LARGE$2^x=5^y={10}^z$\Large\;のとき,

\HUGE
\vspace{-0.3zw}
\hspace{0.5zw}$xy-yz-zx$\vspace{0.3zw}

\Large 
\hfill の値を求めよ.
\at(7.0cm,0.2cm){\small\color{bradorange}$\overset{\text{指数対数}}{\text{計算}}$}

\newpage

\at(0cm,0cm){\includegraphics[width=8cm,bb=0 0 1920 1080]{./youtube/thumbnails/templates/smart_background/指数対数.jpeg}}
{\color{orange}\bf\boldmath\huge\underline{底の変換公式}}\vspace{0.3zw}

\Large 
\bf\boldmath 問.次の式の値を求めよ.

\huge
\vspace{0.1zw}
\hspace{0.2zw}$\left(\log _29+\log _83\right)$\\
\hfill$\times\left(\log _32+\log _94\right)\vspace{0.3zw}$

\at(7.0cm,0.2cm){\small\color{bradorange}$\overset{\text{指数対数}}{\text{計算}}$}

\newpage

\at(0cm,0cm){\includegraphics[width=8cm,bb=0 0 1920 1080]{./youtube/thumbnails/templates/smart_background/指数対数.jpeg}}
{\color{orange}\bf\boldmath\LARGE\underline{無理数乗の大小比較}}\vspace{0.1zw}

\tiny
\bf\boldmath 問.次の$\fbox{\phantom{J}}$に$=,\;<,\;>$のいずれかを入れよ.\\
(1)  $\left(\sqrt 2\right)^2\fbox{\phantom{J}}\log _{\sqrt 2}2$

\scriptsize
(2)  $\left(\sqrt 2\right)^4\fbox{\phantom{J}}\log _{\sqrt 2}4$

\small
(3)  $\left(\sqrt 2\right)^8\fbox{\phantom{J}}\log _{\sqrt 2}8$

\vspace{-0.2zw}
\LARGE
(4)  $\left(\sqrt 2\right)^{\sqrt 8}\fbox{\phantom{J}}\log _{\sqrt 2}\sqrt 8$

\at(7.0cm,0.2cm){\small\color{bradorange}$\overset{\text{指数対数}}{\text{計算}}$}

\newpage

\at(0cm,0cm){\includegraphics[width=8cm,bb=0 0 1920 1080]{./youtube/thumbnails/templates/smart_background/指数対数.jpeg}}
{\color{orange}\bf\boldmath\huge\underline{対数計算}}\vspace{0.3zw}

\large 
\bf\boldmath 問.次の式の値を求めよ.

\LARGE
\vspace{0.2zw}
\hspace{0.5zw}$\log _5\sqrt 2-\bunsuu{1}{2}\log _5\bunsuu{1}{3}$\\
\hfill $-\bunsuu{3}{2}\log _5\sqrt[3]{{30}}\vspace{0.3zw}$

\at(7.0cm,0.2cm){\small\color{bradorange}$\overset{\text{指数対数}}{\text{計算}}$}

\newpage

\at(0cm,0cm){\includegraphics[width=8cm,bb=0 0 1920 1080]{./youtube/thumbnails/templates/smart_background/指数対数.jpeg}}
{\color{orange}\bf\boldmath\huge\underline{肩の上の対数}}\vspace{0.3zw}

\large 
\bf\boldmath 問.次の式の値を求めよ.

\fontsize{60}{0} \selectfont 
\vspace{0.2zw}
\hspace{0.5zw}$3^{\log _98}\vspace{0.3zw}$

\at(7.0cm,0.2cm){\small\color{bradorange}$\overset{\text{指数対数}}{\text{計算}}$}

\newpage

\at(0cm,0cm){\includegraphics[width=8cm,bb=0 0 1920 1080]{./youtube/thumbnails/templates/smart_background/指数対数.jpeg}}
{\color{orange}\bf\boldmath\huge\underline{$3$乗根の有理化}}\vspace{0.2zw}

\large
\bf\boldmath 問.

\fontsize{32}{0} \selectfont
\vspace{-0.6zw}
\hspace{0.8zw} $\bunsuu{5}{{}\sqrt[3]4+1}\vspace{0.3zw}$

\large
\vspace{-0.5zw}
\hfill
の分母を有理化せよ.

\at(7.0cm,0.2cm){\small\color{bradorange}$\overset{\text{指数対数}}{\text{計算}}$}

\newpage

\at(0cm,0cm){\includegraphics[width=8cm,bb=0 0 1920 1080]{./youtube/thumbnails/templates/smart_background/指数対数.jpeg}}
{\color{orange}\bf\boldmath\huge\underline{無理数乗の計算}}\vspace{0.3zw}

\large 
\bf\boldmath 問.次の式の値を求めよ.

\Huge
\vspace{0.5zw}
\hspace{0.5zw}$6^{\sqrt 6}×2^{\sqrt 6}÷3^{\sqrt 6}\vspace{0.3zw}$

\at(7.0cm,0.2cm){\small\color{bradorange}$\overset{\text{指数対数}}{\text{計算}}$}

\newpage

\at(0cm,0cm){\includegraphics[width=8cm,bb=0 0 1920 1080]{./youtube/thumbnails/templates/smart_background/指数対数.jpeg}}
{\color{orange}\bf\boldmath\huge\underline{累乗根の計算}}\vspace{0.3zw}

\large 
\bf\boldmath 問.次の式の値を求めよ.

\huge
\vspace{0.5zw}
\hspace{0.2zw}$\bunsuu{{\sqrt[3]4}}{{\sqrt {16}}}÷\bunsuu{{\sqrt {64}}}{{\sqrt[3]{64}}}×\bunsuu{{\sqrt {32}}}{{\sqrt[3]{32}}}\vspace{0.3zw}$

\at(7.0cm,0.2cm){\small\color{bradorange}$\overset{\text{指数対数}}{\text{計算}}$}

\newpage

\at(0cm,0cm){\includegraphics[width=8cm,bb=0 0 1920 1080]{./youtube/thumbnails/templates/smart_background/指数対数.jpeg}}
{\color{orange}\bf\boldmath\huge\underline{分数の分数乗}}\vspace{0.3zw}

\large 
\bf\boldmath 問.\vspace{-1zw}

\fontsize{35}{0} \selectfont
\vspace{-0.5zw}
\hspace{1.2zw}$\left(\bunsuu{{{27}}}{8}\right)^{\hspace{-0.4zw}-\frac{2}{3}}\vspace{0.3zw}$

\large
\vspace{-1.7zw}
\hfill の値を求めよ.

\at(7.0cm,0.2cm){\small\color{bradorange}$\overset{\text{指数対数}}{\text{計算}}$}

\end{document}

