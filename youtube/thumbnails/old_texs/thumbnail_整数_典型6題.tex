\documentclass[10pt,
% a4paper,
%twocolumn,
fleqn,
%landscape, 
% papersize,
dvipdfmx,
uplatex
]{jsarticle}



\def\maru#1{\textcircled{\scriptsize#1}}%丸囲み番号

% \RequirePackage[2020/09/30]{platexrelease}

%太字設定
\usepackage[deluxe]{otf}

\usepackage{emathEy}

\usepackage[g]{esvect}

%定理環境
\usepackage{emathThm}
%\theoremstyle{boxed}
\theorembodyfont{\normalfont}
\newtheorem{Question}{問題}[subsection]
\newtheorem{Q}{}[subsection]
\newtheorem{question}[Question]{}
\newtheorem{quuestion}{}[subsection]

%セクション,大問番号のデザイン
\renewcommand{\labelenumi}{(\arabic{enumi})}
\renewcommand{\theenumii}{\alph{enumii})}
\renewcommand{\thesection}{第\arabic{section}章}

%用紙サイズの詳細設定
\usepackage{bxpapersize}
\papersizesetup{size={80mm,45mm}}
\usepackage[top=0.7zw,bottom=0truemm,left=3truemm,right=133truemm]{geometry}
\usepackage[dvipdfmx]{graphicx}

%余白など
\usepackage{setspace} % 行間
\setlength{\mathindent}{1zw}
\setlength\parindent{0pt}


%色カラーに関する設定
\usepackage{color}
\definecolor{shiro}{rgb}{0.95703125,0.87109375,0.7421875}
\definecolor{kin}{rgb}{0.95703125,0.87109375,0.7421875}
\definecolor{orange}{rgb}{1,0.7,0.2}
\definecolor{bradorange}{rgb}{1,0.5,0}
\color{kin}
% \pagecolor{hukamido}

\usepackage{at}%図の配置
% \usepackage{wallpaper}

\begin{document}




\at(0cm,0cm){\includegraphics[width=8cm,bb=0 0 1920 1080]{./media_local/smart_background/整数.jpeg}}
{\color{orange}\bf\boldmath\huge\underline{倍数証明}}\vspace{0.3zw}\\
\huge 
\bf\boldmath 問.$n$が奇数のとき,\\$n^5-n$は${240}$の倍数であることを証明せよ.
\at(7.2cm,0.2cm){\small\color{bradorange}$\overset{\text{整数}}{\text{典型}}$}


\newpage



\at(0cm,0cm){\includegraphics[width=8cm,bb=0 0 1920 1080]{./media_local/smart_background/整数.jpeg}}
{\color{orange}\bf\boldmath\LARGE\underline{余りによる分類}}\vspace{0.3zw}\\
\huge 
\bf\boldmath 問.$n^2$が$3$の倍数ならば,$n$も$3$の倍数であることを示せ.
\at(7.2cm,0.2cm){\small\color{bradorange}$\overset{\text{整数}}{\text{典型}}$}


\newpage



\at(0cm,0cm){\includegraphics[width=8cm,bb=0 0 1920 1080]{./media_local/smart_background/整数.jpeg}}
{\color{orange}\bf\boldmath\LARGE\underline{互いに素の証明}}\vspace{0.3zw}\\
\LARGE 
\bf\boldmath 問.自然数$a$と$b$が互いに素であるとき,$a$と$a+b$も互いに素であることを示せ.
\at(7.2cm,0.2cm){\small\color{bradorange}$\overset{\text{整数}}{\text{典型}}$}


\newpage



\at(0cm,0cm){\includegraphics[width=8cm,bb=0 0 1920 1080]{./media_local/smart_background/整数.jpeg}}
{\color{orange}\bf\boldmath\Large\underline{$3$元不定方程式の有名問題}}\vspace{0.3zw}\\
\Large 
\bf\boldmath 問.$\bunsuu{1}{l}+\bunsuu{1}{m}+\bunsuu{1}{n}=1$を満たす\vspace{0.3zw}\\
自然数解$\left(l,\;m,\;n\right)$をすべて求めよ.
ただし,$l\leqq m\leqq n$とする.
\at(7.2cm,0.2cm){\small\color{bradorange}$\overset{\text{整数}}{\text{典型}}$}


\newpage



\at(0cm,0cm){\includegraphics[width=8cm,bb=0 0 1920 1080]{./media_local/smart_background/整数.jpeg}}
{\color{orange}\bf\boldmath\large\underline{$1$次不定方程式の整数解$〜$すべて$〜$}}\vspace{0.3zw}\\
\huge 
\bf\boldmath 問.${29}x+{42}y=4$の整数解をすべて求めよ.
\at(7.2cm,0.2cm){\small\color{bradorange}$\overset{\text{整数}}{\text{典型}}$}


\newpage



\at(0cm,0cm){\includegraphics[width=8cm,bb=0 0 1920 1080]{./media_local/smart_background/整数.jpeg}}
{\color{orange}\bf\boldmath\large\underline{$1$次不定方程式の整数解$〜1$組$〜$}}\vspace{0.3zw}\\
\huge 
\bf\boldmath 問.${29}x+{42}y=4$の整数解を1組求めよ.
\at(7.2cm,0.2cm){\small\color{bradorange}$\overset{\text{整数}}{\text{典型}}$}


\end{document}

