\documentclass[10pt,
% a4paper,
%twocolumn,
fleqn,
%landscape, 
% papersize,
dvipdfmx,
uplatex
]{jsarticle}



\def\maru#1{\textcircled{\scriptsize#1}}%丸囲み番号

% \RequirePackage[2020/09/30]{platexrelease}

%太字設定
\usepackage[deluxe]{otf}

\usepackage{emathEy}

\usepackage[g]{esvect}

%定理環境
\usepackage{emathThm}
%\theoremstyle{boxed}
\theorembodyfont{\normalfont}
\newtheorem{Question}{問題}[subsection]
\newtheorem{Q}{}[subsection]
\newtheorem{question}[Question]{}
\newtheorem{quuestion}{}[subsection]

%セクション,大問番号のデザイン
\renewcommand{\labelenumi}{(\arabic{enumi})}
\renewcommand{\theenumii}{\alph{enumii})}
\renewcommand{\thesection}{第\arabic{section}章}

%用紙サイズの詳細設定
\usepackage{bxpapersize}
\papersizesetup{size={80mm,45mm}}
\usepackage[top=0.7zw,bottom=0truemm,left=3truemm,right=133truemm]{geometry}
\usepackage[dvipdfmx]{graphicx}

%余白など
\usepackage{setspace} % 行間
\setlength{\mathindent}{1zw}
\setlength\parindent{0pt}


%色カラーに関する設定
\usepackage{color}
\definecolor{shiro}{rgb}{0.95703125,0.87109375,0.7421875}
\definecolor{kin}{rgb}{0.95703125,0.87109375,0.7421875}
\definecolor{orange}{rgb}{1,0.7,0.2}
\definecolor{bradorange}{rgb}{1,0.5,0}
\color{kin}
% \pagecolor{hukamido}

\usepackage{at}%図の配置
% \usepackage{wallpaper}

\begin{document}




\at(0cm,0cm){\includegraphics[width=8cm,bb=0 0 1920 1080]{./media_local/smart_background/平面ベクトル.jpeg}}
{\color{orange}\bf\boldmath\LARGE\underline{内心の位置ベクトル}}\vspace{0.3zw}\\
\normalsize 
\bf\boldmath 問.$\angle \text{A}={60}^\circ ,\;\text{AB}=8,\;\text{AC}=5$である三角形$\text{ABC}$の内心を$\text{I}$とし,直線$\text{AI}$と辺$\text{BC}$の交点を$D$とする.$\vv{\text{AB}} =$$\vv{b},\;$$\vv{\text{AC}} =$$\vv{c}$とする.

\Large
(1) $\vv{\text{AD}}$を$\vv{b},\;$$\vv{c}$を用いて表せ.\\
(2) $\vv{\text{AI}}$を$\vv{b},\;$$\vv{c}$を用いて表せ.\\

\at(6.6cm,0.2cm){\small\color{bradorange}$\overset{\text{平面ベクトル}}{\text{典型}}$}


\newpage



\at(0cm,0cm){\includegraphics[width=8cm,bb=0 0 1920 1080]{./media_local/smart_background/平面ベクトル.jpeg}}
{\color{orange}\bf\boldmath\large\underline{ベクトルによる三角形の面積公式}}\vspace{0.3zw}\\
\large 
\bf\boldmath 問.平面上の$4$点$\text{O},\;\text{A},\;\text{B},\;\text{C}$に対し,
\vspace{0.3zw}

\ \ $\vv {\text{OA}}+\vv {\text{OB}}+\vv {\text{OC}}=\vv 0,$

\vspace{0.3zw}
$\text{OA}=2,\;\text{OB}=1,\;\text{OC}=\sqrt 2$のとき,\\
(1)  内積$\vv {\text{OA}}\cdot \vv {\text{OB}}$を求めよ.\\
(2)  三角形$\text{OAB}$の面積を求めよ.\\

\at(6.6cm,0.2cm){\small\color{bradorange}$\overset{\text{平面ベクトル}}{\text{典型}}$}


\newpage



\at(0cm,0cm){\includegraphics[width=8cm,bb=0 0 1920 1080]{./media_local/smart_background/平面ベクトル.jpeg}}
{\color{orange}\bf\boldmath\LARGE\underline{円のベクトル方程式}}\vspace{0.3zw}\\
\Large 
\bf\boldmath 問.平面上の異なる$2$定点$A,\;B$に対して,等式$\zettaiti{2\vv {\text{AP}}+\vv {\text{BP}}}=6$をみたす動点$\text{P}$はどのような図形を描くか.
\at(6.6cm,0.2cm){\small\color{bradorange}$\overset{\text{平面ベクトル}}{\text{典型}}$}


\newpage



\at(0cm,0cm){\includegraphics[width=8cm,bb=0 0 1920 1080]{./media_local/smart_background/平面ベクトル.jpeg}}
{\color{orange}\bf\boldmath\large\underline{ベクトル方程式$〜$円の方程式$〜$}}\vspace{0.5zw}\\
\large 
\bf\boldmath 問.次のような円の方程式を求めよ.

\vspace{0.8zw}
\normalsize 
\ (1)  中心が原点$\text{O}\left(0,\;0\right)$で,半径が$2$の円\\
\ (2)  中心が$\text{A}\left(1,\;5\right)$で,点$\text{B}\left(2,\;2\right)$を通る円\\
\ (3)  $\text{A}\left(5,\;0\right),\;\text{B}\left(9,\;4\right)$を直径の両端とする円\\

\at(6.6cm,0.2cm){\small\color{bradorange}$\overset{\text{平面ベクトル}}{\text{典型}}$}


\newpage



\at(0cm,0cm){\includegraphics[width=8cm,bb=0 0 1920 1080]{./media_local/smart_background/平面ベクトル.jpeg}}
{\color{orange}\bf\boldmath\large\underline{法線ベクトル$〜2$直線のなす鋭角$〜$}}\vspace{0.3zw}\\
\Large 
\bf\boldmath 問.$2$直線\vspace{-0.5zw}
\[\left\{\begin{array}{l}x+\sqrt 3y-5=0,\;\\\sqrt 3x+y+1=0\end{array}\right.\vspace{-0.5zw}\]
がなす鋭角$\theta$を求めよ.
\at(6.6cm,0.2cm){\small\color{bradorange}$\overset{\text{平面ベクトル}}{\text{典型}}$}


\newpage



\at(0cm,0cm){\includegraphics[width=8cm,bb=0 0 1920 1080]{./media_local/smart_background/平面ベクトル.jpeg}}
{\color{orange}\bf\boldmath\huge\underline{法線ベクトル}}\vspace{0.3zw}\\
\huge 
\bf\boldmath 問.点$\text{A}\left(3,\;4\right)$を通り,$\vv{n}=\left(2,\;1\right)$に垂直な直線の方程式を求めよ.
\at(6.6cm,0.2cm){\small\color{bradorange}$\overset{\text{平面ベクトル}}{\text{典型}}$}


\newpage



\at(0cm,0cm){\includegraphics[width=8cm,bb=0 0 1920 1080]{./media_local/smart_background/平面ベクトル.jpeg}}
{\color{orange}\bf\boldmath\large\underline{ベクトルの終点の存在範囲$〜$中級$〜$}}\vspace{0.3zw}\\
\large 
\bf\boldmath 問.三角形$\text{OAB}$において,次の条件を満たす点$\text{P}$の存在範囲を求めよ.
\vspace{0.3zw}

\Large 
$\vv{\text{OP}} =s$$\vv{\text{OA}} +t$$\vv{\text{OB}} ,\;\vspace{-0.0zw}\\\hfill s+t=2,\;s\geqq 0,\;t\geqq 0$

\at(6.6cm,0.2cm){\small\color{bradorange}$\overset{\text{平面ベクトル}}{\text{典型}}$}


\newpage



\at(0cm,0cm){\includegraphics[width=8cm,bb=0 0 1920 1080]{./media_local/smart_background/平面ベクトル.jpeg}}
{\color{orange}\bf\boldmath\large\underline{ベクトルの終点の存在範囲$〜$初級$〜$}}\vspace{0.3zw}\\
\large 
\bf\boldmath 問.三角形$\text{OAB}$において,次の条件を満たす点$\text{P}$の存在範囲を求めよ.
\vspace{0.3zw}

\Large 
$\vv{\text{OP}} =s$$\vv{\text{OA}} +t$$\vv{\text{OB}} ,\;\vspace{-0.0zw}\\\hfill s+t=\bunsuu{1}{3},\;s\geqq 0,\;t\geqq 0$

\at(6.6cm,0.2cm){\small\color{bradorange}$\overset{\text{平面ベクトル}}{\text{典型}}$}


\newpage



\at(0cm,0cm){\includegraphics[width=8cm,bb=0 0 1920 1080]{./media_local/smart_background/平面ベクトル.jpeg}}
{\color{orange}\bf\boldmath\LARGE\underline{垂心の位置ベクトル}}\vspace{0.3zw}\\
\Large 
\bf\boldmath 問.$\text{OA}=2\sqrt 2,\;\text{OB}=\sqrt 3,\;$\\
\ \ \ \ \ $\vv{\text{OA}}\cdot\vv{\text{OB}}=2$\ \ である\\
三角形$\text{OAB}$の垂心$\text{H}$に対して,\\
$\vv{\text{OH}}$を$\vv{\text{OA}}$と$\vv{\text{OB}}$を用いて表せ.
\at(6.6cm,0.2cm){\small\color{bradorange}$\overset{\text{平面ベクトル}}{\text{典型}}$}


\newpage



\at(0cm,0cm){\includegraphics[width=8cm,bb=0 0 1920 1080]{./media_local/smart_background/平面ベクトル.jpeg}}
{\color{orange}\bf\boldmath\Large\underline{内積を利用した図形証明}}\vspace{0.3zw}\\
\Large 
\bf\boldmath 問.三角形$\text{ABC}$において,辺$\text{BC}$の中点を$\text{M}$とするとき,等式
\vspace{0.3zw}

\ \ $\text{AB}^2+\text{AC}^2=2\left(\text{AM}^2+\text{BM}^2\right)\vspace{0.3zw}$
% 

が成立することを示せ.
\at(6.6cm,0.2cm){\small\color{bradorange}$\overset{\text{平面ベクトル}}{\text{典型}}$}


\newpage



\at(0cm,0cm){\includegraphics[width=8cm,bb=0 0 1920 1080]{./media_local/smart_background/平面ベクトル.jpeg}}
{\color{orange}\bf\boldmath\Large\underline{交点の位置ベクトル$〜$中級$〜$}}\vspace{0.3zw}\\
\normalsize 
\bf\boldmath 問.三角形$\text{OAB}$において,辺$\text{OA}$を$2:1$に内分する点を$\text{C},\;$辺$\text{OB}$の中点を$\text{D}$とする.

\Large
線分$\text{AD}$と線分$\text{BC}$の交点を$\text{P}$とするとき,$\vv{\text{OP}}$を$\vv{\text{OA}}$と$\vv{\text{OB}}$を用いて表せ.\\

\at(6.6cm,0.2cm){\small\color{bradorange}$\overset{\text{平面ベクトル}}{\text{典型}}$}


\newpage



\at(0cm,0cm){\includegraphics[width=8cm,bb=0 0 1920 1080]{./media_local/smart_background/平面ベクトル.jpeg}}
{\color{orange}\bf\boldmath\Large\underline{交点の位置ベクトル$〜$初級$〜$}}\vspace{0.3zw}\\
\normalsize
\bf\boldmath 問.三角形$\text{OAB}$において,辺$\text{OA}$を$2:1$に内分する点を$\text{C},\;$辺$\text{OB}$の中点を$\text{D}$とする.

\Large
直線$\text{OP}$と辺$\text{AB}$の交点を$\text{Q}$とするとき,$\vv{\text{OQ}}$を$\vv{\text{OA}}$と$\vv{\text{OB}}$を用いて表せ.\\

\at(6.6cm,0.2cm){\small\color{bradorange}$\overset{\text{平面ベクトル}}{\text{典型}}$}


\newpage



\at(0cm,0cm){\includegraphics[width=8cm,bb=0 0 1920 1080]{./media_local/smart_background/平面ベクトル.jpeg}}
{\color{orange}\bf\boldmath\large\underline{$3$点が一直線上にあることの証明}}\vspace{0.3zw}\\
\large 
\bf\boldmath 問.平行四辺形$\text{ABCD}$において,辺$\text{CD}$を$2:1$に内分する点を$E,\;$対角線$\text{BD}$を$3:1$に内分する点を$\text{F}$とする.

\Large $3$点$\text{A},\;\text{F},\;\text{E}$は一直線上にあることを証明せよ.
\at(6.6cm,0.2cm){\small\color{bradorange}$\overset{\text{平面ベクトル}}{\text{典型}}$}


\newpage



\at(0cm,0cm){\includegraphics[width=8cm,bb=0 0 1920 1080]{./media_local/smart_background/平面ベクトル.jpeg}}
{\color{orange}\bf\boldmath\Large\underline{点が一致することの証明}}\vspace{0.3zw}\\
\large 
\bf\boldmath 問.三角形$\text{ABC}$において,辺\text{BC},\;\text{CA},\;\\
\text{AB}を$3:1$に内分する点を,それぞれ\text{P},\;\text{Q},\;\text{R},\;三角形$\text{PQR}$の重心を$\text{G}',\;$三角形$\text{ABC}$の重心を$\text{G}$とする.このとき,

\Large 
\ \ $\text{G}$と$\text{G}'$は一致することを示せ.
\at(6.6cm,0.2cm){\small\color{bradorange}$\overset{\text{平面ベクトル}}{\text{典型}}$}


\newpage


\at(0cm,0cm){\includegraphics[width=8cm,bb=0 0 1920 1080]{./media_local/smart_background/平面ベクトル.jpeg}}
{\color{orange}\bf\boldmath\Large\underline{ベクトル和の等式$〜$三角形$〜$}}\vspace{0.3zw}\\
\Large 
\bf\boldmath 問.三角形$\text{ABC}$に対して,次の式を満たす点$\text{P}$の位置を求めよ.
\vspace{0.8zw}

\LARGE
\ \ $2\vv {\text{PA}}+3\vv {\text{PB}}+\vv {\text{PC}}=\vv {0}$


\at(6.6cm,0.2cm){\small\color{bradorange}$\overset{\text{平面ベクトル}}{\text{典型}}$}


\newpage



\at(0cm,0cm){\includegraphics[width=8cm,bb=0 0 1920 1080]{./media_local/smart_background/平面ベクトル.jpeg}}
{\color{orange}\bf\boldmath\LARGE\underline{ベクトルのなす角}}\vspace{0.3zw}\\
\LARGE 
\bf\boldmath 問.$\vv {a}=\left(7,\;-1\right)$と${45}^\circ$の角をなし,大きさが${10}$である$\vv {b}$を求めよ.
\at(6.6cm,0.2cm){\small\color{bradorange}$\overset{\text{平面ベクトル}}{\text{典型}}$}


\newpage



\at(0cm,0cm){\includegraphics[width=8cm,bb=0 0 1920 1080]{./media_local/smart_background/平面ベクトル.jpeg}}
{\color{orange}\bf\boldmath\Large\underline{ベクトルの内積計算$〜$中級$〜$}}\vspace{0.3zw}\\
\Large 
\bf\boldmath 問.$\zettaiti{\vv {a}}=2,\;\zettaiti{\vv {b}}=1$で,$\vv {a}+\vv {b}$と$2\vv {a}-5\vv {b}$が垂直であるとする.\\
(1)  内積$\vv {a}\cdot \vv {b}$を求めよ.\\
(2)  大きさ$\zettaiti{\vv {a}-2\vv {b}}$を求めよ.\\

\at(6.6cm,0.2cm){\small\color{bradorange}$\overset{\text{平面ベクトル}}{\text{典型}}$}


\newpage



\at(0cm,0cm){\includegraphics[width=8cm,bb=0 0 1920 1080]{./media_local/smart_background/平面ベクトル.jpeg}}
{\color{orange}\bf\boldmath\Large\underline{ベクトルの大きさの最小}}\vspace{0.3zw}\\
\Large 
\bf\boldmath 問.$\vv {a}=\left(3,\;1\right),\;\vv {b}=\left(1,\;2\right)$と実数$t$に対して,
\vspace{0.3zw}

\LARGE\ \ $\vv {c}=\vv {a}+t\vv {b}$
\vspace{0.3zw}

\Large の大きさ$\zettaiti{\vv {c}}$の最小値を求めよ.
\at(6.6cm,0.2cm){\small\color{bradorange}$\overset{\text{平面ベクトル}}{\text{典型}}$}


\end{document}

