\documentclass[10pt,
% a4paper,
%twocolumn,
fleqn,
%landscape, 
% papersize,
dvipdfmx,
uplatex
]{jsarticle}



\def\maru#1{\textcircled{\scriptsize#1}}%丸囲み番号

% \RequirePackage[2020/09/30]{platexrelease}

%太字設定
\usepackage[deluxe]{otf}

\usepackage{emathEy}

\usepackage[g]{esvect}

%定理環境
\usepackage{emathThm}
%\theoremstyle{boxed}
\theorembodyfont{\normalfont}
\newtheorem{Question}{問題}[subsection]
\newtheorem{Q}{}[subsection]
\newtheorem{question}[Question]{}
\newtheorem{quuestion}{}[subsection]

%セクション,大問番号のデザイン
\renewcommand{\labelenumi}{(\arabic{enumi})}
\renewcommand{\theenumii}{\alph{enumii})}
\renewcommand{\thesection}{第\arabic{section}章}

%用紙サイズの詳細設定
\usepackage{bxpapersize}
\papersizesetup{size={80mm,45mm}}
\usepackage[top=0.7zw,bottom=0truemm,left=3truemm,right=133truemm]{geometry}
\usepackage[dvipdfmx]{graphicx}

%余白など
\usepackage{setspace} % 行間
\setlength{\mathindent}{1zw}
\setlength\parindent{0pt}


%色カラーに関する設定
\usepackage{color}
\definecolor{shiro}{rgb}{0.95703125,0.87109375,0.7421875}
\definecolor{kin}{rgb}{0.95703125,0.87109375,0.7421875}
\definecolor{orange}{rgb}{1,0.7,0.2}
\definecolor{bradorange}{rgb}{1,0.5,0}
\color{kin}
% \pagecolor{hukamido}

\usepackage{at}%図の配置
% \usepackage{wallpaper}

\begin{document}




\at(0cm,0cm){\includegraphics[width=8cm,bb=0 0 1920 1080]{./media_local/smart_background/空間ベクトル.jpeg}}
{\color{orange}\bf\boldmath\huge\underline{平面の方程式}}\vspace{0.3zw}

\Large 
\bf\boldmath 問.次の平面の方程式を求めよ.

\normalsize
(1)  点$\text{A}\left(1,\;2,\;2\right)$を通り,\\
\hfill$\vv n=\left(2,\;-2,\;4\right)$に垂直な平面\\
(2)  $3$点$\text{A}\left(0,\;1,\;1\right),\text{B}\left(1,\;0,\;2\right),\text{C}\left(-3,2,3\right)$\\
\hfill を通る平面\\

\at(6.6cm,0.2cm){\small\color{bradorange}$\overset{\text{空間ベクトル}}{\text{典型}}$}


\newpage



\at(0cm,0cm){\includegraphics[width=8cm,bb=0 0 1920 1080]{./media_local/smart_background/空間ベクトル.jpeg}}
{\color{orange}\bf\boldmath\Large\underline{空間内の直線の方程式}}\vspace{0.3zw}

\Large 
\bf\boldmath 問.次の直線の媒介変数表示と,\\
\hfill 直線の方程式を求めよ.

\normalsize
(1)  点$\text{A}\left(4,\;5,\;3\right)$を通り,\\
\hfill$\vv d=\left(3,\;2,\;-4\right)$に平行な直線\\
(2)  $2$点$\text{A}\left(1,\;2,\;3\right),\text{B}\left(2,\;-1,\;5\right)$を通る直線\\

\at(6.6cm,0.2cm){\small\color{bradorange}$\overset{\text{空間ベクトル}}{\text{典型}}$}


\newpage



\at(0cm,0cm){\includegraphics[width=8cm,bb=0 0 1920 1080]{./media_local/smart_background/空間ベクトル.jpeg}}
{\color{orange}\bf\boldmath\Large\underline{空間ベクトルの和の等式}}\vspace{0.3zw}

\Large 
\bf\boldmath 問.四面体$\text{ABCD}$において等式

\vspace{0.3zw}
\hspace{0.3zw}$\vv {\text{AP}}+4\vv {\text{BP}}+3\vv {\text{CP}}+5\vv {\text{DP}}=\vv 0\vspace{0.3zw}$


を満たす点$\text{P}$はどのような点か.
\at(6.6cm,0.2cm){\small\color{bradorange}$\overset{\text{空間ベクトル}}{\text{典型}}$}


\newpage



\at(0cm,0cm){\includegraphics[width=8cm,bb=0 0 1920 1080]{./media_local/smart_background/空間ベクトル.jpeg}}
{\color{orange}\bf\boldmath\Large\underline{空間ベクトルの大きさの最小}}\vspace{0.3zw}

\large 
\bf\boldmath 問.原点$\text{O}$と$3$点$\text{P}\left(1,2,1\right)$,$\text{Q}\left(2,1,2\right)$,\\
$\text{R}\left(1,-2,3\right)$について

\Large
\vspace{0.3zw}
\hspace{0.5zw}$\zettaiti{x\vv {\text{OP}}+y\vv {\text{OQ}}+\vv {\text{OR}}}\vspace{0.3zw}$

\large
の最小値と,\\
\hfill そのときの実数$x,\;y$の値を求めよ.
\at(6.6cm,0.2cm){\small\color{bradorange}$\overset{\text{空間ベクトル}}{\text{典型}}$}


\newpage



\at(0cm,0cm){\includegraphics[width=8cm,bb=0 0 1920 1080]{./media_local/smart_background/空間ベクトル.jpeg}}
{\color{orange}\bf\boldmath\Large\underline{正四面体の第四の点の座標}}\vspace{0.3zw}

\Large 
\bf\boldmath 問.$3$点

\large
\vspace{0.3zw}
\hspace{0.3zw}$\text{A}\left(6,\;0,\;0\right)$,\;$\text{B}\left(0,\;6,\;0\right)$,\;$\text{C}\left(0,\;0,\;6\right)\vspace{0.3zw}$

\Large 
に対して,正四面体$\text{ABCD}$の頂点$\text{D}$の座標を求めよ.
\at(6.6cm,0.2cm){\small\color{bradorange}$\overset{\text{空間ベクトル}}{\text{典型}}$}


\newpage



\at(0cm,0cm){\includegraphics[width=8cm,bb=0 0 1920 1080]{./media_local/smart_background/空間ベクトル.jpeg}}
{\color{orange}\bf\boldmath\large\underline{平面に下ろした垂線の足の座標}}\vspace{0.3zw}

\Large 
\bf\boldmath 問.$\text{A}\left(2,\;0,\;0\right),\;\text{B}\left(0,\;3,\;0\right),\;$\\
$\text{C}\left(0,\;0,\;3\right)$の定める平面$\text{ABC}$に原点$\text{O}$から下ろした垂線を$\text{OH}$とするとき,点$\text{H}$の座標を求めよ.
\at(6.6cm,0.2cm){\small\color{bradorange}$\overset{\text{空間ベクトル}}{\text{典型}}$}


\newpage



\at(0cm,0cm){\includegraphics[width=8cm,bb=0 0 1920 1080]{./media_local/smart_background/空間ベクトル.jpeg}}
{\color{orange}\bf\boldmath\Large\underline{空間における点の一致の証明}}\vspace{0.3zw}

\large 
\bf\boldmath 問.四面体$\text{ABCD}$において,\\
辺$\text{AB}$,\;$\text{BC}$,\;$\text{CD}$,\;$\text{DA}$,\;$\text{AC}$,\;$\text{BD}$の中点をそれぞれ$\text{K},\;\text{L},\;\text{M},\;\text{N},\;\text{Q},\;\text{R}$とする.

\Large
線分$\text{KM},\;\text{LN},\;\text{QR}$の中点は一致することを証明せよ.
\at(6.6cm,0.2cm){\small\color{bradorange}$\overset{\text{空間ベクトル}}{\text{典型}}$}


\newpage



\at(0cm,0cm){\includegraphics[width=8cm,bb=0 0 1920 1080]{./media_local/smart_background/空間ベクトル.jpeg}}
{\color{orange}\bf\boldmath\Large\underline{空間における三角形の面積}}\vspace{0.3zw}

\Large 
\bf\boldmath 問.$3$点

\normalsize
\vspace{0.3zw}
\hspace{0.5zw}$\text{A}\left(3,\;2,\;4\right)$,\;$\text{B}\left(3,\;-1,\;-1\right)$,\;$\text{C}\left(5,\;3,\;-3\right)$
\vspace{0.3zw}

\Large 
を頂点とする三角形$\text{ABC}$の面積を求めよ.
\at(6.6cm,0.2cm){\small\color{bradorange}$\overset{\text{空間ベクトル}}{\text{典型}}$}


\newpage



\at(0cm,0cm){\includegraphics[width=8cm,bb=0 0 1920 1080]{./media_local/smart_background/空間ベクトル.jpeg}}
{\color{orange}\bf\boldmath\LARGE\underline{空間図形の垂直証明}}\vspace{0.3zw}

\Large 
\bf\boldmath 問.正四面体$\text{ABCD}$において,\\
三角形$\text{BCD}$の重心を$\text{G}$とする.

\LARGE 
$\text{AG}$と$\text{BC}$が垂直であることを証明せよ.

\at(6.6cm,0.2cm){\small\color{bradorange}$\overset{\text{空間ベクトル}}{\text{典型}}$}


\newpage



\at(0cm,0cm){\includegraphics[width=8cm,bb=0 0 1920 1080]{./media_local/smart_background/空間ベクトル.jpeg}}
{\color{orange}\bf\boldmath\Large\underline{$4$点が同一平面上にある条件}}\vspace{0.3zw}

\Large 
\bf\boldmath 問.$4$点

\normalsize
\hspace{3zw}$\text{A}\left(3,\;1,\;2\right),\;\text{B}\left(4,\;2,\;3\right)$,\;\\
\hspace{3zw}$\text{C}\left(5,\;2,\;5\right),\;\text{D}\left(-2,\;-1,\;z\right)$

\Large 
が同一平面上にあるとき,\\
\hfill$z$の値を求めよ.
\at(6.6cm,0.2cm){\small\color{bradorange}$\overset{\text{空間ベクトル}}{\text{典型}}$}


\newpage



\at(0cm,0cm){\includegraphics[width=8cm,bb=0 0 1920 1080]{./media_local/smart_background/空間ベクトル.jpeg}}
{\color{orange}\bf\boldmath\Large\underline{$3$点が同一直線上にある条件}}\vspace{0.3zw}

\Large 
\bf\boldmath 問.$3$点

\normalsize
\vspace{0.3zw}
\hspace{0.5zw}$\text{A}\left(2,\;-1,\;5\right),\;\text{B}\left(3,\;6,\;9\right),\;\text{C}\left(1,\;y,\;z\right)\vspace{0.3zw}$

\Large
が一直線上にあるとき,\\
\hfill$y,\;z$の値を求めよ.
\at(6.6cm,0.2cm){\small\color{bradorange}$\overset{\text{空間ベクトル}}{\text{典型}}$}


\newpage



\at(0cm,0cm){\includegraphics[width=8cm,bb=0 0 1920 1080]{./media_local/smart_background/空間ベクトル.jpeg}}
{\color{orange}\bf\boldmath\Large\underline{平面$\text{ABC}$と直線の交点}}\vspace{1zw}

\Large 
\bf\boldmath 問.直方体OABD-CEFGにおいて,対角線$\text{OF}$と平面$\text{ABC}$の交点を$\text{P}$とする.
$\text{OP}:\text{OF}$を求めよ.
\at(6.6cm,0.2cm){\small\color{bradorange}$\overset{\text{空間ベクトル}}{\text{典型}}$}


\newpage



\at(0cm,0cm){\includegraphics[width=8cm,bb=0 0 1920 1080]{./media_local/smart_background/空間ベクトル.jpeg}}
{\color{orange}\bf\boldmath\LARGE\underline{空間における垂直条件}}\vspace{0.3zw}

\Large 
\bf\boldmath 問.2つのベクトル

\large
\vspace{0.3zw}
\hspace{0.3zw}$\vv{a}=\left(2,\;-1,\;0\right),\;$$\vv{b}=\left(6,\;-2,\;1\right)$
\vspace{0.3zw}

\Large 
の両方に垂直で,大きさが$3$であるベクトル$\vv{p}$を求めよ.
\at(6.6cm,0.2cm){\small\color{bradorange}$\overset{\text{空間ベクトル}}{\text{典型}}$}


\end{document}

