\documentclass[10pt,
% a4paper,
%twocolumn,
fleqn,
%landscape, 
% papersize,
dvipdfmx,
uplatex
]{jsarticle}



\def\maru#1{\textcircled{\scriptsize#1}}%丸囲み番号

% \RequirePackage[2020/09/30]{platexrelease}

%太字設定
\usepackage[deluxe]{otf}

\usepackage{emathEy}

\usepackage[g]{esvect}

%定理環境
\usepackage{emathThm}
%\theoremstyle{boxed}
\theorembodyfont{\normalfont}
\newtheorem{Question}{問題}[subsection]
\newtheorem{Q}{}[subsection]
\newtheorem{question}[Question]{}
\newtheorem{quuestion}{}[subsection]

%セクション,大問番号のデザイン
\renewcommand{\labelenumi}{(\arabic{enumi})}
\renewcommand{\theenumii}{\alph{enumii})}
\renewcommand{\thesection}{第\arabic{section}章}

%用紙サイズの詳細設定
\usepackage{bxpapersize}
\papersizesetup{size={80mm,45mm}}
\usepackage[top=0.7zw,bottom=0truemm,left=3truemm,right=133truemm]{geometry}
\usepackage[dvipdfmx]{graphicx}

%余白など
\usepackage{setspace} % 行間
\setlength{\mathindent}{1zw}
\setlength\parindent{0pt}


%色カラーに関する設定
\usepackage{color}
\definecolor{shiro}{rgb}{0.95703125,0.87109375,0.7421875}
\definecolor{kin}{rgb}{0.95703125,0.87109375,0.7421875}
\definecolor{orange}{rgb}{1,0.7,0.2}
\definecolor{bradorange}{rgb}{1,0.5,0}
\color{kin}
% \pagecolor{hukamido}

\usepackage{at}%図の配置
% \usepackage{wallpaper}

\begin{document}




\at(0cm,0cm){\includegraphics[width=8cm,bb=0 0 1920 1080]{./media_local/smart_background/数II式と証明.jpeg}}
{\color{orange}\bf\boldmath\huge\underline{不等式の証明}}\vspace{0.3zw}

\LARGE 
\bf\boldmath 問.次の不等式を証明せよ.

\vspace{0.3zw}
\hspace{0.3zw}$a^2+b^2+c^2\geqq ab+bc+ca\vspace{0.3zw}$


\at(6.6cm,0.2cm){\small\color{bradorange}$\overset{\text{数Ⅱ式と証明}}{\text{典型}}$}


\newpage



\at(0cm,0cm){\includegraphics[width=8cm,bb=0 0 1920 1080]{./media_local/smart_background/数II式と証明.jpeg}}
{\color{orange}\bf\boldmath\Large\underline{少なくとも1つは$1$に等しい}}\vspace{0.3zw}

\large 
\bf\boldmath 問.$\alpha +\beta +\gamma=\bunsuu{1}{\alpha }+\bunsuu{1}{\beta }+\bunsuu{1}{\gamma}=1\vspace{0.3zw}$

\large 
ならば,$\alpha ,\;\beta ,\;\gamma$のうち

\LARGE 
\hspace{0.3zw}少なくとも\vspace{0.2zw}$1$つは$1$に等しい

\large \hfill ことを証明せよ.
\at(6.6cm,0.2cm){\small\color{bradorange}$\overset{\text{数Ⅱ式と証明}}{\text{典型}}$}


\newpage



\at(0cm,0cm){\includegraphics[width=8cm,bb=0 0 1920 1080]{./media_local/smart_background/数II式と証明.jpeg}}
{\color{orange}\bf\boldmath\huge\underline{比例式の計算}}\vspace{0.3zw}

\large 
\bf\boldmath 問.$x+y=\bunsuu{{y+z}}{2}=\bunsuu{{z+x}}{5}\neq 0$\\
\hfill のとき,\vspace{-0.3zw}

\LARGE 
\hspace{0.3zw}$\bunsuu{{xy+yz+zx}}{{x^2+y^2+z^2}}$
\large の値を求めよ.
\at(6.6cm,0.2cm){\small\color{bradorange}$\overset{\text{数Ⅱ式と証明}}{\text{典型}}$}


\newpage



\at(0cm,0cm){\includegraphics[width=8cm,bb=0 0 1920 1080]{./media_local/smart_background/数II式と証明.jpeg}}
{\color{orange}\bf\boldmath\LARGE\underline{条件つきの等式の証明}}\vspace{0.3zw}

\LARGE 
\bf\boldmath 問.$a+b+c=0$のとき,

\large
\vspace{0.5zw}
\hspace{0.3zw}$a^2\left(b+c\right)+b^2\left(c+a\right)+c^2\left(a+b\right)$\\
\hfill $+3abc=0$

\LARGE 
であることを示せ.
\at(6.6cm,0.2cm){\small\color{bradorange}$\overset{\text{数Ⅱ式と証明}}{\text{典型}}$}


\newpage



\at(0cm,0cm){\includegraphics[width=8cm,bb=0 0 1920 1080]{./media_local/smart_background/数II式と証明.jpeg}}
{\color{orange}\bf\boldmath\huge\underline{恒等式の基本}}\vspace{0.3zw}

\Large 
\bf\boldmath 問.次の式が恒等式となるように,定数$a,\;b,\;c,\;d$の値を定めよ.


\large
\vspace{0.3zw}
\hspace{0.5zw}$x^3=a\left(x-1\right)^3+b\left(x-1\right)^2$\\
\hfill $+c\left(x-1\right)+d\vspace{0.3zw}$

\at(6.6cm,0.2cm){\small\color{bradorange}$\overset{\text{数Ⅱ式と証明}}{\text{典型}}$}


\end{document}

