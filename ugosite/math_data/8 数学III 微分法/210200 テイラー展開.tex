\newif\iffuru
%\furutrue%フルバージョン
\furufalse%解説のみ

\newif\iffigure
%\figuretrue%図表あり
\figurefalse%図表なし

\documentclass[10pt,
b5paper,
%twocolumn,
fleqn,
%landscape, 
%papersize
dvipdfmx,
uplatex
]{jsarticle}

\def\maru#1{\textcircled{\scriptsize#1}}%丸囲み番号

%\renewcommand{\bf}{}
%\usepackage{アプローチ}
%\usepackage{amsthm}
\usepackage{amsmath}
\usepackage{ascmac}
\usepackage{graphics}
\usepackage{emath}
\usepackage{emathMw}
\usepackage{enumerate}
\usepackage{emathC}
\usepackage{emathEy}
\usepackage{emathP}
\usepackage{emathPp}
\usepackage{emathPl}
\usepackage{emathPk}
\usepackage{emathPh}
\usepackage{emathPs}
\usepackage[g]{esvect}
\usepackage{color}
\usepackage{pxrubrica}%ふりがな
%\usepackage{EMesvect}
%\usepackage[dvipdfmx]{graphicx}%箱ひげ図
%\usepackage{emathSt}%箱ひげ図
%\usepackage{emathG}%箱ひげ図%ヒストグラム
\usepackage{emathPs}
\usepackage{ascmac}%囲み
\usepackage{fancybox}
%\usepackage{fancybx}%囲み
\usepackage[top=12truemm,bottom=12truemm,left=12truemm,right=10truemm]{geometry}
\usepackage{setspace} % 行間
\setstretch{1} % ページ全体の行間を設定
\usepackage{wallpaper}
\usepackage{hako}%センター形式
\usepackage{mathtools}%\abs(絶対値)など
\DeclarePairedDelimiter{\abs}{\lvert}{\rvert}
\usepackage{fancyhdr}%ヘッダの設定
\usepackage{cancel}%消し取り線
\usepackage{ifthen}
%\usepackage{exam}
\usepackage{tcolorbox}
\usepackage{extarrows}%伸縮性のある矢印

%ページレイアウト
%\setlength{\columnseprule}{0.4pt}
%\columnsep=3cm
%%\setlength{\mathindent}{0zw}
\preHEqlabel{$\cdotfill[2em]~$}%houtesikiの点線の長さ
%\linespread{1.3}
%\setstretch{1.2}%行間
\postEqlabel{\hspace{0zw}\null}%式番号の位置
\preEqlabel*{\cdotpfill[2em]~}%式番号の点々の長さ
%\setlength{\columnseprule}{0.4pt}
%\columnsep=1cm
%\setlength{\mathindent}{1zw}%数式の位置


%\pagestyle{headings}
\pagestyle{empty}%ページ番号を消す
% \pagestyle{fancy}
%  \fancyhead{}
%  \fancyhead[RO,RE]{\rightmark}
%%  \fancyhead[LE,LO]{\leftmark}
%  \cfoot{\thepage}
%  \renewcommand{\chaptermark}[1]{\markboth{第\ \thechapter\ 章~#1}{}}
%  \renewcommand{\sectionmark}[1]{\markright{\thesection #1}{}}
 
%\markboth{}{\thesection}
 
%\fancyhead{} % clear all fields
%\fancyhead[CE]{偶数ページ}
%\fancyhead[CO]{奇数ページ}

%定理環境
\usepackage{emathThm}
%\theoremstyle{boxed}
\theorembodyfont{\normalfont}
\newtheorem{Question}{問題}[subsection]
\newtheorem{Q}{}[subsection]
\newtheorem{question}[Question]{}
\newtheorem{quuestion}{}[subsection]

%問題レイアウト
\tcbuselibrary{raster,skins}
\newenvironment{黒tcolorbox}{
\begin{tcolorbox}[enhanced,
frame style={left color=orange!50!white,
right color=black!50!orange},
colback=black!0!white,
drop fuzzy shadow]}{\end{tcolorbox}}
\newcommand{\sub}{\newpage\ \vspace{-4zw}\subsection}
\newcommand{\bqu}{\begin{黒tcolorbox}\begin{question}}
\newcommand{\equ}{\end{question}\end{黒tcolorbox}}
\newcommand{\mondaisettei}{\kaisetukaitoufalse
\renewcommand{\sub}{\subsection}
\renewcommand{\bqu}{\vspace{0.5zw}\begin{question}}
\renewcommand{\equ}{\end{question}\vspace{3zw}\vfill}
}%問題設定
\newcommand{\kaisetutukinosettei}{\kaisetukaitoutrue
\renewcommand{\sub}{\newpage\ \vspace{-4zw}\subsection}
\renewcommand{\bqu}{\begin{黒tcolorbox}\begin{question}}
\renewcommand{\equ}{\end{question}\end{黒tcolorbox}}
}%解答解説の設定

%箇条書きの調整
%\setlength{\itemsep}{5pt}      %2. ブロック間の余白
%\setlength{\parskip}{0pt}      %4. 段落間余白.
%\setlength{\itemindent}{0pt}   %5. 最初のインデント
%\setlength{\labelsep}{5pt}     %6. item と文字の間

%箇条書き省略コマンド
\newcommand{\benu}{\begin{enumerate}}
\newcommand{\eenu}{\end{enumerate}}
\newcommand{\beda}{\begin{edaenumerate}}
\newcommand{\eeda}{\end{edaenumerate}}

\newcommand{\bb}{\bf\boldmath}%全部太字にする
%\newcommand{\bb}{\gtfamily\ebseries\boldmath}%全部極太にする
\newcommand{\doo}{^{\circ}}%角度マーク
\newcommand{\sq}{\textstyle\sqrt}
\newcommand{\ANA}{\hakosenhaba{1pt}\Hako}
\newcommand{\REFANA}{\hakosenhaba{0.3pt}\refHako*}
\newcommand{\C}{\text{C}}
\newcommand{\dsum}{\displaystyle\sum}
\newcommand{\barr}{\left\{\begin{array}{l}}
\newcommand{\earr}{\end{array}\right.}
\newcommand{\cdotss}{\hfill\cdots\cdots}

\usepackage{tabularx}
%\newcolumntype{Y}{&gt;{\centering\arraybackslash}X} %中央揃え
%\includegraphics[width=90mm,bb=9 9 358 434]{./lrep_e1.eps}

%セクション,大問番号のデザイン
\renewcommand{\labelenumi}{(\arabic{enumi})}
%\renewcommand{\labelenumi}{\ \fbox{\protect\makebox[1em][c]{\large{\bfseries\arabic{enumi}}}}\ }
%\renewcommand{\labelenumi}{\textbf{\theenumi}}
%\renewcommand{\theenumiii}{(\alph{enumiii})}
%\renewcommand{\theenumii}{\arabic{enumii}}
%\renewcommand{\thesection}{\Huge  第\arabic{section}章}
\renewcommand{\thesubsection}{\bb 第\arabic{subsection}回
\ }
\renewcommand{\theQuestion}{%\arabic{subsection}-
\large\arabic{Question}.}

%横に縦線
\usepackage{framed}
\makeatletter
\renewenvironment{leftbar}{%
\def\FrameCommand{\vrule width 1pt \hspace{1zw}}
\MakeFramed{\advance\hsize-\width \FrameRestore}}%
{\endMakeFramed}
\makeatother

\newenvironment{leftbbar}{%
\def\FrameCommand{\color{mygray} \vrule width 5pt \hspace{1zw}
\color{black}}%
\MakeFramed {\advance\hsize-\width \FrameRestore}}%
{\endMakeFramed}
\makeatother

%アプローチ
\newenvironment{アプローチ}{
\hspace{-2zw}\underbar{\large \bf Approach}\vspace{-1zw}\begin{leftbar}}{\end{leftbar}}

\newenvironment{解答}{
\hspace{-2zw}\phkasen<linethickness=7pt,iro=mygray,kasenUehosei=-3pt>{\bf \large \ 解答\ }\vspace{-1zw}\begin{leftbbar}\vspace{0zw}}{\end{leftbbar}}

\newenvironment{証明}{
\hspace{-2zw}\phkasen<linethickness=7pt,iro=mygray,kasenUehosei=-3pt>{\bf \large \ 証明\ }\vspace{-1zw}\begin{leftbbar}\vspace{-2zw}}{\end{leftbbar}}

\newenvironment{解答1}{
\hspace{-2zw}\phkasen<linethickness=7pt,iro=mygray,kasenUehosei=-3pt>{\bf \large \ 解答1\ }\vspace{-1zw}\begin{leftbbar}\vspace{-2zw}}{\end{leftbbar}}
\newenvironment{解答2}{
\hspace{-2zw}\phkasen<linethickness=7pt,iro=mygray,kasenUehosei=-3pt>{\bf \large \ 解答2\ }\vspace{-1zw}\begin{leftbbar}}{\end{leftbbar}}
\newenvironment{解答3}{
\hspace{-2zw}\phkasen<linethickness=7pt,iro=mygray,kasenUehosei=-3pt>{\bf \large \ 解答3\ }\vspace{-1zw}\begin{leftbbar}}{\end{leftbbar}}

\newcommand{\kaitoui}{{\bb \color{mygray} $\hookrightarrow$}\phkasen<linethickness=7pt,iro=mygray,kasenUehosei=-3pt>{\bf \ 解答1\ }}
\newcommand{\kaitouii}{{\bb \color{mygray} $\hookrightarrow$}\phkasen<linethickness=7pt,iro=mygray,kasenUehosei=-3pt>{\bf \ 解答2\ }}
\newcommand{\kaitouiii}{{\bb \color{mygray} $\hookrightarrow$}\phkasen<linethickness=7pt,iro=mygray,kasenUehosei=-3pt>{\bf \ 解答3\ }}

%QRコード
\usepackage{qrcode}
\setlength\normallineskiplimit{0pt}

%カラー,色
\usepackage{color}
\definecolor{link}{rgb}{0.63671875,0.99609375,0.99609375}
\definecolor{usumido}{rgb}{0.953125,0.95703125,0.9375}
%\pagecolor{usumido}
\definecolor{mygray}{gray}{0.75}

\newif\ifkaisetukaitou

\newcommand{\kaisetukaitou}{%問題のみ
\mondaisettei
\myfor{1} % ループ実行       
\newpage   
\setcounter{subsection}{0}
\setcounter{Question}{0}
\kaisetutukinosettei
\myfor{1} % ループ実行   
%\TileWallPaper{110mm}{160mm}{方眼紙.pdf} %方眼紙
%\myfor{1} % ループ実行   
}
\begin{document}

{\bb\Large 第3回\ \ $\sin x$と$\cos x$を多項式関数で近似する}

%\hfill 大橋\\

\begin{tcolorbox}[enhanced,
frame style={left color=orange!50!white,
right color=black!50!orange},
colback=black!0!white,
drop fuzzy shadow]
\bb 
{\large 問.} $x>0$のとき,次の不等式が成り立つことを示せ.
\beda
\item $\sin x<x$
\item $\cos x >1-\bunsuu{x^2}{2!}$
\item $\sin x>x-\bunsuu{x^3}{3!}$
\item $\cos x <1-\bunsuu{x^2}{2!}+\bunsuu{x^4}{4!}$
\eeda
\end{tcolorbox}

\begin{解答}\vspace{-2.5zw}
\benu
\item 
$y=\sin x$の$x=0$における微分係数は1なので,
右のグラフより
,$\sin x <x$.\hfill □
\item $f(x)=\cos x-1+\bunsuu{x^2}{2!}$とおくと,
\begin{mawarikomi}{}{
\vspace{2zw}
%\iffigure
\begin{zahyou}[ul=5mm](-1,3)(-1,2)
\YGraph<linethickness=1pt>{sin(X)}
\YGraph{X}
\end{zahyou}
%\fi
}
\[f'(x)=-\sin x+x>0 \ \ \ \because\ \ \ (1)\]
よって,$f(x)$は単調増加するので,$f(x)>f(0)=0$.ゆえに,$\cos x >1-\bunsuu{x^2}{2!}$.\hfill □
\end{mawarikomi}
\item $g(x)=\sin x-x+\bunsuu{x^3}{3!}$とおくと,
\[g'(x)=\cos x-1+\bunsuu{x^2}{2!} >0 \ \ \ \because\ \ \ (2)\]
よって,$g(x)$は単調増加するので,$g(x)>g(0)=0$.ゆえに,$\sin x>x-\bunsuu{x^3}{3!}$.\hfill □
\item $h(x)=\cos x -1+\bunsuu{x^2}{2!}-\bunsuu{x^4}{4!}$とおくと,
\[h'(x)=-\sin x+x-\bunsuu{x^3}{3!} <0 \ \ \ \because\ \ \ (3)\]
よって,$h(x)$は単調減少するので,$h(x)<h(0)=0$.ゆえに,$\cos x <1-\bunsuu{x^2}{2!}+\bunsuu{x^4}{4!}$.\hfill □
\eenu
\end{解答}

この問題から得られたのは,$\sin x$と$\cos x$の多項式関数による評価である;
\beda
\item[] \bb ${\color{blue} x-\bunsuu{x^3}{3!}}<\sin x<{\color{red}x}$\\
%\iffigure
\begin{zahyou}[ul=8mm](-4,4)(-2,2)
\YGraph<linethickness=1pt>{sin(X)}
\YGraph<linethickness=1pt,color=red>{X}
\YGraph<linethickness=1pt,color=blue>{X-X*X*X/(3*2*1)}
\end{zahyou}
\item[] \bb ${\color{orange}1-\bunsuu{x^2}{2!}}<\cos x <{\color{green}1-\bunsuu{x^2}{2!}+\bunsuu{x^4}{4!}}$\\
\begin{zahyou}[ul=8mm](-4,4)(-2,2)
\YGraph<linethickness=1pt>{cos(X)}
\YGraph<linethickness=1pt,color=orange>{1-X*X/(2*1)}
\YGraph<linethickness=1pt,color=green>{1-X*X/(2*1)+X*X*X*X/(4*3*2*1)}
\end{zahyou}
%\fi
\eeda
これを繰り返していくと,$\sin x$,$\cos x$の多項式近似が見えてくる;
\begin{tcolorbox}[enhanced,
frame style={left color=orange!50!white,
right color=black!50!orange},
colback=black!0!white,
drop fuzzy shadow,
title={\bb $\sin x$,$\cos x$のテイラー展開},
coltitle=black]
\bb 
$\sin x=x-\bunsuu{x^3}{3!}+\bunsuu{x^5}{5!}-\cdots+(-1)^{n-1}\bunsuu{x^{2n-1}}{(2n-1)!}+\cdots \ \ \ \cdotss\MARU{1}$\\
$\cos x=1-\bunsuu{x^2}{2!}+\bunsuu{x^4}{4!}-\cdots+(-1)^{n-1}\bunsuu{x^{2n-2}}{(2n-2)!}+\cdots\ \ \ \cdotss\MARU{2}$
\end{tcolorbox}

\newpage

{\bb\Large 第4回\ \ $e^x$を多項式関数で近似する}

\begin{tcolorbox}[enhanced,
frame style={left color=orange!50!white,
right color=black!50!orange},
colback=black!0!white,
drop fuzzy shadow]
\bb 
{\large 問.}\ \ $x>0$のとき,次の各不等式を証明せよ.ただし,$n$は自然数とする.
\beda
%\item $e^x>1$
\item $e^x>1+x$
\item $e^x>1+x+\bunsuu{x^2}{2!}$
%\item $e^x>1+x+\bunsuu{x^2}{2!}+\bunsuu{x^3}{3!}$
\item $e^x>1+x+\bunsuu{x^2}{2!}+\bunsuu{x^3}{3!}+\cdots+\bunsuu{x^n}{n!}$
\eeda
\end{tcolorbox}

\begin{解答} \vspace{-2.5zw}
\benu
\item $f(x)=e^x-1-x$とおくと,$f'(x)=e^x-1>0$.\\
よって,$f(x)$は単調増加ので,$f(x)>f(0)=0$.ゆえに,$e^x>1+x$.\hfill □
\item $g(x)=e^x-1-x-\bunsuu{x^2}{2}$とおくと,$f'(x)=e^x-1-x>0\ \ \ \because\ \ \ \text{(i)}$.\\
よって,$g(x)$は単調増加するので,$g(x)>g(0)=0$.ゆえに,$e^x>1+x+\bunsuu{x^2}{2}$.\hfill □
\item $f_{n}(x)=e^x-1-x-\bunsuu{x^2}{2!}-\cdots-\bunsuu{x^n}{n!}$とおく.
$k$を自然数とし$f_k(x)>0$と仮定すると,
\begin{align*}
f_{k+1}'(x)
&=\left(e^x-1-x-\bunsuu{x^2}{2!}-\cdots-\bunsuu{x^{k+1}}{(k+1)!}\right)'\\
&=e^x-1-x-\bunsuu{x^2}{2!}-\cdots-\bunsuu{x^k}{k!}=f_k(x)>0.
\end{align*}
よって,$f_{k+1}(x)$は単調増加するので,$f_{k+1}(x)>f_{k+1}(0)=0$.\\
これと(i)より$f_1(x)>0$から,全ての自然数$n$に対して,$f_{n}(x)>0$が成立する.\\
ゆえに,$e^x>1+x+\bunsuu{x^2}{2!}+\bunsuu{x^3}{3!}+\cdots+\bunsuu{x^n}{n!}$が成立する.\hfill □
\eenu
\end{解答}
\beda<4>
%\iffigure
\item[] \hspace{-2zw}{\color{blue}$y=1+x$}\\
\begin{zahyou}[ul=5mm](-3,3)(-1,5)
%\YGraph{1}
\YGraph<linethickness=1pt>{exp(X)}
\YGraph<color=blue,linethickness=1pt>{1+X}
%\YGraph<color=red,linethickness=1pt>{1+X+X*X/(2*1)}
%\YGraph<color=orange,linethickness=1pt>{1+X+X*X/(2*1)+X*X*X/(3*2*1)}
%\YGraph<color=green,linethickness=1pt>{1+X+X*X/(2*1)+X*X*X/(3*2*1)+X*X*X*X/(4*3*2*1)}
\end{zahyou}
\item[] \hspace{-2zw}{\color{red}$y=1+x+\bunsuu{x^2}{2!}$}\\
\begin{zahyou}[ul=5mm](-3,3)(-1,5)
%\YGraph{1}
\YGraph<linethickness=1pt>{exp(X)}
%\YGraph<color=pink,linethickness=1pt>{1+X}
\YGraph<color=red,linethickness=1pt>{1+X+X*X/(2*1)}
%\YGraph<color=orange,linethickness=1pt>{1+X+X*X/(2*1)+X*X*X/(3*2*1)}
%\YGraph<color=green,linethickness=1pt>{1+X+X*X/(2*1)+X*X*X/(3*2*1)+X*X*X*X/(4*3*2*1)}
\end{zahyou}
\item[] \hspace{-2zw}{\color{orange} $y=1+x+\bunsuu{x^2}{2!}+\bunsuu{x^3}{3!}$}\\
\begin{zahyou}[ul=5mm](-3,3)(-1,5)
%\YGraph{1}
\YGraph<linethickness=1pt>{exp(X)}
%\YGraph<color=pink,linethickness=1pt>{1+X}
%\YGraph<color=red,linethickness=1pt>{1+X+X*X/(2*1)}
\YGraph<color=orange,linethickness=1pt>{1+X+X*X/(2*1)+X*X*X/(3*2*1)}
%\YGraph<color=green,linethickness=1pt>{1+X+X*X/(2*1)+X*X*X/(3*2*1)+X*X*X*X/(4*3*2*1)}
\end{zahyou}
\item[] \hspace{-2zw}\scalebox{0.8}[1]{\color{green} $y=1+x+\bunsuu{x^2}{2!}+\bunsuu{x^3}{3!}+\bunsuu{x^4}{4!}$}\\
\begin{zahyou}[ul=5mm](-3,3)(-1,5)
%\YGraph{1}
\YGraph<linethickness=1pt>{exp(X)}
%\YGraph<color=pink,linethickness=1pt>{1+X}
%\YGraph<color=red,linethickness=1pt>{1+X+X*X/(2*1)}
%\YGraph<color=orange,linethickness=1pt>{1+X+X*X/(2*1)+X*X*X/(3*2*1)}
\YGraph<color=green,linethickness=1pt>{1+X+X*X/(2*1)+X*X*X/(3*2*1)+X*X*X*X/(4*3*2*1)}
\end{zahyou}%\fi
\eeda
(3)の不等式において,右辺の級数は増加数列であり,かつ,$e^x$を超えないので,収束することがわかる.実は,その極限は左辺に一致することが知られている;
%\[e^x \geqq 1+x+\bunsuu{x^2}{2!}+\bunsuu{x^3}{3!}+\cdots+\bunsuu{x^n}{n!}+\cdots\]
%となることがわかる.実は,これは等式になる;
\begin{tcolorbox}[enhanced,
frame style={left color=orange!50!white,
right color=black!50!orange},
colback=black!0!white,
drop fuzzy shadow,
title={\bb $e^x$のテイラー展開},
coltitle=black]
\bb 
$e^x=1+x+\bunsuu{x^2}{2!}+\bunsuu{x^3}{3!}+\cdots+\bunsuu{x^n}{n!}+\cdots$
\end{tcolorbox}
この等式の証明は大学以降に譲る.ここに$x=1$を代入すると
\begin{tcolorbox}[enhanced,
frame style={left color=orange!50!white,
right color=black!50!orange},
colback=black!0!white,
drop fuzzy shadow,
title={\bb 自然対数の底$e$の値の近似式},
coltitle=black]
\bb 
$e=1+1+\bunsuu{1}{2!}+\bunsuu{1}{3!}+\cdots+\bunsuu{1}{n!}+\cdots$
\end{tcolorbox}
が得られる.これはなかなか収束の速い近似式で,実用性が高い.
%また,$x$を$x i$に形式的に置き換えて,前ページの$\MARU{1}$,$\MARU{2}$と組み合せれば
%\begin{tcolorbox}[enhanced,
%frame style={left color=orange!50!white,
%right color=black!50!orange},
%colback=black!0!white,
%drop fuzzy shadow,
%title={\bb 指数関数と三角関数の関係式},
%coltitle=black]
%\bb 
%$e^{ix}=\cos x +i\sin x$
%\end{tcolorbox}
%を得る.さらに,$x=\pi$とすると,有名な公式が導かれる;
%\begin{tcolorbox}[enhanced,
%frame style={left color=orange!50!white,
%right color=black!50!orange},
%colback=black!0!white,
%drop fuzzy shadow,
%title={\bb オイラーの公式},
%coltitle=black]
%\bb 
%$e^{i\pi}=-1$
%\end{tcolorbox}

%\newpage
%
%\begin{tcolorbox}[enhanced,
%frame style={left color=orange!50!white,
%right color=black!50!orange},
%colback=black!0!white,
%drop fuzzy shadow,
%title={\bb $n$次導関数},
%coltitle=black]
%\bb 
%\begin{align*}
%(\sin x)^{(n)}&=
%%\barr  (-1)^{\frac {n-1}2}\cos x\ (n\ は奇数)\\
%%(-1)^{\frac n2}\sin x\ (n\ は偶数)
%% \earr\\
%\left\{\begin{array}{rl} 
%\sin x\ &(n=4k)\\
%\cos x\ &(n=4k+1)\\
%-\sin x\ &(n=4k+2)\\
%-\cos x\ &(n=4k+3)\
% \end{array}\right.=\sin\left(x+\bunsuu{n}{2}\pi\right)\\
%(\cos x)^{(n)}&=
%\left\{\begin{array}{rl}
%\cos x\ &(n=4k)\\
%-\sin x\ &(n=4k+1)\\
%-\cos x\ &(n=4k+2)\\
%\sin x\ &(n=4k+3)
% \end{array}\right.=\cos\left(x+\bunsuu{n}{2}\pi\right)\\
%%\barr
%% (-1)^{\frac {n+1}2}\sin x\ (n\ は奇数)\\
%% (-1)^{\frac n2}\cos x\ (n\ は偶数)
%% \earr\\
%(e^x)^{(n)}&=e^x
%\end{align*}
%%\\
%%\log x&=1+x+\bunsuu{x^2}{2!}+\bunsuu{x^3}{3!}+\bunsuu{x^4}{4!}+\cdots\\
%\end{tcolorbox}



%\newpage
%
%\begin{tcolorbox}[enhanced,
%frame style={left color=orange!50!white,
%right color=black!50!orange},
%colback=black!0!white,
%drop fuzzy shadow,
%title={\bb テイラーの定理},
%coltitle=black]
%\bb 
%$n$階微分可能な関数$f(x)$に対して
%\[f(x)=f(a)+f'(a)(x-a)+\bunsuu{f''(0)}{2!}(x-a)^2+\cdots\]
%\end{tcolorbox}
%
%\begin{証明}
%$x \neq 0$とすると,平均値の定理より,$0$と$x$の間の$c$で
%\[\bunsuu{f(x)-f(0)}{x}=f'(c)\]
%となるものが存在する.
%\end{証明}
%
%
%\begin{tcolorbox}[enhanced,
%frame style={left color=orange!50!white,
%right color=black!50!orange},
%colback=black!0!white,
%drop fuzzy shadow,
%title={\bb テイラー展開},
%coltitle=black]
%\bb 
%\begin{align*}
%\sin x&=x-\bunsuu{x^3}{3!}+\bunsuu{x^5}{5!}-\cdots\\
%\cos x&=1-\bunsuu{x^2}{2!}+\bunsuu{x^4}{4!}-\cdots\\
%e^x&=1+x+\bunsuu{x^2}{2!}+\bunsuu{x^3}{3!}+\bunsuu{x^4}{4!}+\cdots
%\end{align*}
%%\\
%%\log x&=1+x+\bunsuu{x^2}{2!}+\bunsuu{x^3}{3!}+\bunsuu{x^4}{4!}+\cdots\\
%\end{tcolorbox}






\iffuru
}\kaisetukaitou
\fi                                                                 

\end{document}