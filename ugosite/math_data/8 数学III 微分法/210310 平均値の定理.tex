\newif\iffuru
\furutrue%フルバージョン
%\furufalse%解説のみ

\newif\iffigure
\figuretrue%図表あり
%\figurefalse%図表なし

\documentclass[10pt,
b5paper,
%twocolumn,
fleqn,
%landscape, 
%papersize
dvipdfmx,
uplatex
]{jsarticle}

\def\maru#1{\textcircled{\scriptsize#1}}%丸囲み番号

%\renewcommand{\bf}{}
%\usepackage{アプローチ}
%\usepackage{amsthm}
\usepackage{amsmath}
\usepackage{ascmac}
\usepackage{graphics}
\usepackage{emath}
\usepackage{emathMw}
\usepackage{enumerate}
\usepackage{emathC}
\usepackage{emathEy}
\usepackage{emathP}
\usepackage{emathPp}
\usepackage{emathPl}
\usepackage{emathPk}
\usepackage{emathPh}
\usepackage{emathPs}
\usepackage[g]{esvect}
\usepackage{color}
\usepackage{pxrubrica}%ふりがな
%\usepackage{EMesvect}
%\usepackage[dvipdfmx]{graphicx}%箱ひげ図
%\usepackage{emathSt}%箱ひげ図
%\usepackage{emathG}%箱ひげ図%ヒストグラム
\usepackage{emathPs}
\usepackage{ascmac}%囲み
\usepackage{fancybox}
%\usepackage{fancybx}%囲み
\usepackage[top=12truemm,bottom=12truemm,left=12truemm,right=10truemm]{geometry}
\usepackage{setspace} % 行間
\setstretch{1} % ページ全体の行間を設定
\usepackage{wallpaper}
\usepackage{hako}%センター形式
\usepackage{mathtools}%\abs(絶対値)など
\DeclarePairedDelimiter{\abs}{\lvert}{\rvert}
\usepackage{fancyhdr}%ヘッダの設定
\usepackage{cancel}%消し取り線
\usepackage{ifthen}
%\usepackage{exam}
\usepackage{tcolorbox}
\usepackage{extarrows}%伸縮性のある矢印

%ページレイアウト
%\setlength{\columnseprule}{0.4pt}
%\columnsep=3cm
%%\setlength{\mathindent}{0zw}
\preHEqlabel{$\cdotfill[2em]~$}%houtesikiの点線の長さ
%\linespread{1.3}
%\setstretch{1.2}%行間
\postEqlabel{\hspace{0zw}\null}%式番号の位置
\preEqlabel*{\cdotpfill[2em]~}%式番号の点々の長さ
%\setlength{\columnseprule}{0.4pt}
%\columnsep=1cm
%\setlength{\mathindent}{1zw}%数式の位置


%\pagestyle{headings}
\pagestyle{empty}%ページ番号を消す
% \pagestyle{fancy}
%  \fancyhead{}
%  \fancyhead[RO,RE]{\rightmark}
%%  \fancyhead[LE,LO]{\leftmark}
%  \cfoot{\thepage}
%  \renewcommand{\chaptermark}[1]{\markboth{第\ \thechapter\ 章~#1}{}}
%  \renewcommand{\sectionmark}[1]{\markright{\thesection #1}{}}
 
%\markboth{}{\thesection}
 
%\fancyhead{} % clear all fields
%\fancyhead[CE]{偶数ページ}
%\fancyhead[CO]{奇数ページ}

%定理環境
\usepackage{emathThm}
%\theoremstyle{boxed}
\theorembodyfont{\normalfont}
\newtheorem{Question}{問題}[subsection]
\newtheorem{Q}{}[subsection]
\newtheorem{question}[Question]{}
\newtheorem{quuestion}{}[subsection]

%問題レイアウト
\tcbuselibrary{raster,skins}
\tcbuselibrary{xparse}
\newtcolorbox{mybox}{
enhanced,
frame style={left color=orange!50!white,
right color=black!50!orange},
colback=black!0!white,
drop fuzzy shadow
}
\newcommand{\sub}{\newpage\ \vspace{-4zw}\subsection}
\newcommand{\bqu}{\begin{question}}
\newcommand{\equ}{\end{question}}
\newcommand{\bQ}{\begin{Q}}
\newcommand{\eQ}{\end{Q}}

\newcommand{\mondaisettei}{\kaisetufalse
\renewcommand{\sub}{\subsection}
\renewcommand{\bqu}{\vspace{0.5zw}\begin{question}}
\renewcommand{\equ}{\end{question}\vspace{3zw}}
\renewcommand{\bQ}{\vspace{0.5zw}\begin{Q}}
\renewcommand{\eQ}{\end{Q}\vspace{3zw}}
}%問題設定

\newcommand{\kaisetutukinosettei}{\kaisetutrue
\renewcommand{\sub}{\newpage\ \vspace{-4zw}\subsection}
\renewcommand{\bqu}{\begin{mybox}\begin{question}}
\renewcommand{\equ}{\end{question}\end{mybox}}
\renewcommand{\bQ}{\begin{mybox}\begin{Q}}
\renewcommand{\eQ}{\end{Q}\end{mybox}}
}%解答解説の設定

%箇条書きの調整
%\setlength{\itemsep}{5pt}      %2. ブロック間の余白
%\setlength{\parskip}{0pt}      %4. 段落間余白.
%\setlength{\itemindent}{0pt}   %5. 最初のインデント
%\setlength{\labelsep}{5pt}     %6. item と文字の間

%箇条書き省略コマンド
\newcommand{\benu}{\begin{enumerate}}
\newcommand{\eenu}{\end{enumerate}}
\newcommand{\beda}{\begin{edaenumerate}}
\newcommand{\eeda}{\end{edaenumerate}}

\newcommand{\bb}{\bf\boldmath}%全部太字にする
%\newcommand{\bb}{\gtfamily\ebseries\boldmath}%全部極太にする
\newcommand{\doo}{^{\circ}}%角度マーク
\newcommand{\sq}{\textstyle\sqrt}
\newcommand{\ANA}{\hakosenhaba{1pt}\Hako}
\newcommand{\REFANA}{\hakosenhaba{0.3pt}\refHako*}
\newcommand{\C}{\text{C}}
\newcommand{\dsum}{\displaystyle\sum}
\newcommand{\barr}{\left\{\begin{array}{l}}
\newcommand{\earr}{\end{array}\right.}
\newcommand{\cdotss}{\hfill\cdots\cdots}
\newcommand{\adots}{\reflectbox{$\ddots$}}

\usepackage{tabularx}
%\newcolumntype{Y}{&gt;{\centering\arraybackslash}X} %中央揃え
%\includegraphics[width=90mm,bb=9 9 358 434]{./lrep_e1.eps}

%セクション,大問番号のデザイン
\renewcommand{\labelenumi}{(\arabic{enumi})}
%\renewcommand{\labelenumi}{\ \fbox{\protect\makebox[1em][c]{\large{\bfseries\arabic{enumi}}}}\ }
%\renewcommand{\labelenumi}{\textbf{\theenumi}}
%\renewcommand{\theenumiii}{(\alph{enumiii})}
%\renewcommand{\theenumii}{\arabic{enumii}}
%\renewcommand{\thesection}{\Huge  第\arabic{section}章}
\renewcommand{\thesubsection}{\bb 第\arabic{subsection}回
\ }
\renewcommand{\theQuestion}{%\arabic{subsection}-
問題\arabic{Question}.}
\renewcommand{\theQ}{%\arabic{subsection}-
例題\arabic{Q}.}

%横に縦線
\usepackage{framed}
\makeatletter
\renewenvironment{leftbar}{%
\def\FrameCommand{\vrule width 1pt \hspace{1zw}}
\MakeFramed{\advance\hsize-\width \FrameRestore}}%
{\endMakeFramed}
\makeatother

\newenvironment{leftbbar}{%
\def\FrameCommand{\color{mygray} \vrule width 5pt \hspace{1zw}
\color{black}}%
\MakeFramed {\advance\hsize-\width \FrameRestore}}%
{\endMakeFramed}
\makeatother

%アプローチ
\newenvironment{証明}{
\hspace{-2zw}\phkasen<linethickness=7pt,iro=mygray,kasenUehosei=-3pt>{\bf \large \ 証明\ }\vspace{-1zw}\begin{leftbbar}}{\end{leftbbar}}


\newenvironment{アプローチ}{
\hspace{-2zw}\underbar{\large \bf Approach}\vspace{-1zw}\begin{leftbar}}{\end{leftbar}}

\newenvironment{解答}{
\hspace{-2zw}\phkasen<linethickness=7pt,iro=mygray,kasenUehosei=-3pt>{\bf \large \ 解答\ }\vspace{-1zw}\begin{leftbbar}}{\end{leftbbar}}

\newenvironment{解答1}{
\hspace{-2zw}\phkasen<linethickness=7pt,iro=mygray,kasenUehosei=-3pt>{\bf \large \ 解答1\ }\vspace{-1zw}\begin{leftbbar}}{\end{leftbbar}}
\newenvironment{解答2}{
\hspace{-2zw}\phkasen<linethickness=7pt,iro=mygray,kasenUehosei=-3pt>{\bf \large \ 解答2\ }\vspace{-1zw}\begin{leftbbar}}{\end{leftbbar}}
\newenvironment{解答3}{
\hspace{-2zw}\phkasen<linethickness=7pt,iro=mygray,kasenUehosei=-3pt>{\bf \large \ 解答3\ }\vspace{-1zw}\begin{leftbbar}}{\end{leftbbar}}
\newenvironment{別解}{
\hspace{-2zw}\phkasen<linethickness=7pt,iro=mygray,kasenUehosei=-3pt>{\bf \large \ 別解\ }\vspace{-1zw}\begin{leftbbar}}{\end{leftbbar}}

\newcommand{\kaitoui}{{\bb \color{mygray} $\hookrightarrow$}\phkasen<linethickness=7pt,iro=mygray,kasenUehosei=-3pt>{\bf \ 解答1\ }}
\newcommand{\kaitouii}{{\bb \color{mygray} $\hookrightarrow$}\phkasen<linethickness=7pt,iro=mygray,kasenUehosei=-3pt>{\bf \ 解答2\ }}
\newcommand{\kaitouiii}{{\bb \color{mygray} $\hookrightarrow$}\phkasen<linethickness=7pt,iro=mygray,kasenUehosei=-3pt>{\bf \ 解答3\ }}
\newcommand{\bekkai}{{\bb \color{mygray} $\hookrightarrow$}\phkasen<linethickness=7pt,iro=mygray,kasenUehosei=-3pt>{\bf \ 別解\ }}

%QRコード
\usepackage{qrcode}
\setlength\normallineskiplimit{0pt}

%カラー,色
\usepackage{color}
\definecolor{link}{rgb}{0.63671875,0.99609375,0.99609375}
\definecolor{usumido}{rgb}{0.953125,0.95703125,0.9375}
%\pagecolor{usumido}
\definecolor{mygray}{gray}{0.75}

\newif\ifkaisetu

\newcommand{\mondaitokaitou}{
\mondaisettei
\myfor{1} % ループ実行       
\newpage   
\setcounter{subsection}{0}
\setcounter{Question}{0}
\setcounter{Q}{0}
\kaisetutukinosettei
\myfor{1} % ループ実行   
%\TileWallPaper{110mm}{160mm}{方眼紙.pdf} %方眼紙
%\myfor{1} % ループ実行   
}
\begin{document}

\kaisetutukinosettei

%{\large 17期\ 2年\ 春季学習講座}\\
\ 

\vspace{10zw}

\hfill {\bb \HUGE 平均値の定理}\hfill\ \\

\hfill {\LARGE〜基礎から「解けない漸化式」への応用まで〜}\hfill \ 

\vfill
\vfill


\hfill {\HUGE $1+\bunsuu{1}
{2+\scalebox{0.8}{$\bunsuu{1}{2+\scalebox{0.8}{$\bunsuu{1}
{2+\scalebox{0.8}{$\bunsuu{1}{2+\scalebox{0.8}{$\bunsuu{1}
{2+\scalebox{0.8}{$\bunsuu{1}{2+\scalebox{0.8}{$\bunsuu{1}
{2+\scalebox{0.8}{$\bunsuu{1}{2+\scalebox{0.8}{$\bunsuu{1}
{2+\scalebox{0.8}{$\bunsuu{1}{2+\scalebox{0.8}{$\bunsuu{1}
{2+\scalebox{0.8}{$\bunsuu{1}{2+\scalebox{0.8}{$\bunsuu{1}
{2+\scalebox{0.8}{$\bunsuu{1}{2+\scalebox{0.8}{$\bunsuu{1}
{2+\scalebox{0.8}{$\bunsuu{1}{2+\scalebox{0.8}{$\bunsuu{1}
{2+\scalebox{0.8}{$\bunsuu{1}{2+\scalebox{0.8}{$\bunsuu{1}
{2+\scalebox{0.8}{$\bunsuu{1}{2+\bunsuu{1}{}}$}}$}}$}}$}}$}}$}}$}}$}}$}}$}}$}}$}}$}}$}}$}}$}}$}}$}}$}}=\sq{2}$}\hfill \ 

\vfill
\vfill
\vfill



\newpage

{\bb\Large 平均値の定理の基本}\\

関数$f(x)$が,閉区間$[a,\ b]$で連続,かつ,開区間$(a,\ b)$で微分可能であるとする.\\
このとき,開区間$(a,\ b)$内の値$c$で,
\[\bunsuu{f(b)-f(a)}{b-a}=f'(c)\]
を満たすものが存在する.
\vspace{15zw}


\iffuru
\newcommand{\myfor}[1]{
\fi

\ifkaisetu
\ \newpage
\fi

\bQ 
関数$f(x)=\log x$の区間$1\leqq x\leqq e$について,平均値の定理における$c$の値を求めよ.
\eQ

\ifkaisetu
\begin{解答}
 $f'(x)=\bunsuu 1x$より,{\bb 平均値の定理}より,$1 < c< e$なる$c$で,
\[\bunsuu{\log e -\log 1}{e-1}=\bunsuu 1{c}\]
を満たすものが存在する.このとき,$c=\text{\bb $e-1$}$.
\end{解答}
\fi

\bQ
極限値$\dlim_{x \to 0} \bunsuu{\sin x-\sin(\sin x)}{x-\sin x}$を求めよ.
\eQ

\ifkaisetu
\begin{解答} $f(x)=\sin x$とおくと,$f'(x)=\cos x$である.\\
{\bb 平均値の定理}より,$x$と$\sin x$の間の値$c$で,
\[\bunsuu{f(x)-f(\sin x)}{x-\sin x}=f'(c)\ \ \ すなわち\ \ \ \bunsuu{\sin x-\sin (\sin x)}{x-\sin x}=\cos c\]
となるものが存在する.
$x \to 0$のとき,$c \to 0$より,
\[\dlim_{x \to 0} \bunsuu{\sin x-\sin (\sin x)}{x-\sin x}=\dlim_{c \to 0}\cos c=\text{\bb 1}\]
\end{解答}
\fi

\newpage

\bqu 
 双曲線$y=\frac 1x$上の2点$(1,\ 1)$,$(2,\ \frac 12)$を結ぶ線分と平行な直線で,この双曲線に接するものの方程式を求めよ.
 \equ
 
 \ifkaisetu
\begin{解答}\vspace{-2zw}
 $y'=-\bunsuu{1}{x^2}$より,接点の座標を$(c,\ \frac 1c)$とおくと,
\[\bunsuu{\frac 12-1}{2-1}=-\bunsuu{1}{c^2}\ \ \ \therefore\ \ \ c=\sq 2\]
ゆえに,求める直線の方程式は
\[y=-\bunsuu{1}{\sq 2 ^2}(x-\sq 2)+\bunsuu{1}{\sq 2}\ \ \ \therefore\ \ \ \text{\bb $y=-\bunsuu 12x +\sq 2$}\]
\end{解答}
\fi
 
\bqu
 $A_n=\bunsuu 12(\log n)^2$であるとき,
$\dlim_{n \to \infty}(A_{n+1}-A_n)$を求めよ.
ただし,$\dlim_{x \to \infty}\bunsuu{\log x}{x}=0$は用いてよい.
\ifkaisetu \hfill(名大)\fi
\equ

\ifkaisetu
\begin{解答}\vspace{-2zw}
 $f(x)=\bunsuu 12(\log x)^2$とおくと,$f'(x)=\bunsuu{\log x}{x}$.\\
{\bb 平均値の定理}より,$n<x<n+1$かつ
\[\bunsuu{f(n+1)-f(n)}{(n+1)-n}=\bunsuu{\log c}{c}\]
となる$c$が存在する.$n \to \infty$のとき,$c \to \infty$より,
\[\dlim_{x\to \infty}(A_{n+1}-A_n)=\dlim_{c\to \infty}\bunsuu{\log c}{c}={\bb 0}\]
\end{解答}

%\newpage

\fi

\vfill

\bQ $e$を自然対数の底とする.$e \leqq p<q$のとき,不等式
\[\log(\log q)-\log(\log p) < \bunsuu{q-p}{e}\]
が成り立つことを証明せよ.
\ifkaisetu \hfill(名大)\fi
\eQ

\ifkaisetu
\begin{解答}\vspace{-2zw}
$f(x)=\log(\log x)$とおくと,$f'(x)=\bunsuu{1}{x\log x}$.\\
{\bb 平均値の定理}より,$p<c<q$となる$c$で
\[\bunsuu{f(q)-f(p)}{q-p}=f'(c)\ \ \ すなわち\ \ \ \bunsuu{\log(\log q)-\log(\log p)}{q-p} = \bunsuu{1}{c\log c}\]
を満たすものが存在する.$p \geqq e$より,$c>e$なので,
\[\bunsuu{1}{c\log c}<\bunsuu{1}{e\log e}=\bunsuu 1e\ \ \ \therefore\ \ \ \bunsuu{\log(\log q)-\log(\log p)}{q-p}<\bunsuu 1e\]
したがって,$\log(\log q)-\log(\log p) < \bunsuu{q-p}{e}$.\hfill □
%$x>e$において,$f''(x)=-\bunsuu{\log x- 1}{(x\log x)^2}<0$より,$f'(x)$は単調減少する.\\
%\[\bunsuu{\log(\log q)-\log(\log p)}{q-p} = f'(c) < f'(e) = \bunsuu 1e\]
%\underset{\MARU{1}}{<} \bunsuu{1}{e\log e} =\bunsuu 1e\]
%ここで,$x > e$において,$y=x$も,$y=\log x$も正の数で単調増加するので,$f'(x)=\bunsuu{1}{x\log x}$は単調減少するので,\MARU{1}が成り立つ.
%よって\ \ \ $\log(\log q)-\log(\log p) =(q-p) f'(c) < (q-p) f'(e)=\bunsuu{q-p}{e}$.
\end{解答}

\newpage

\fi

%\bqu $n$を2以上の整数とするとき,
%\[\bunsuu{\log n}{n-1}>\bunsuu{\log(n+1)}{n}\]
%を証明せよ.\hfill (千葉大)
%\equ

%\bqu
%$A_n=\bunsuu 12(\log n)^2$であるとき,
%$\dlim_{n \to \infty}(A_{n+1}-A_n)$を求めよ.
%ただし,$\dlim_{x \to \infty}\bunsuu{\log x}{x}=0$は用いてよい.
%\hfill(名大)
%\equ

{\bb\Large 「解けない漸化式」への応用}

\bQ $f(x)=\bunsuu 12\cos x$とする.
\benu
\item 任意の$x$,$y$に対し,$\left|f(x)-f(y)\right|\leqq \bunsuu 12|x-y|$が成り立つことを証明せよ.
\item $x=f(x)$はただ1つの解をもつことを証明せよ.
\item 任意の$a$に対して,$a_0=a$,$a_n=f(a_{n-1})$\ $(n=1,2,3,\cdots)$で定められる数列$\{a_n\}$は,$f(x)=x$の解に収束することを示せ.\ifkaisetu \hfill(三重大)\fi
\eenu
\eQ

\ifkaisetu
\begin{解答}
\benu
\item $x=y$のとき,両辺=0より,成立するので,以下$x \neq y$とする.\\
$f'(x)=-\bunsuu 12 \sin x$.{\bb 平均値の定理}より,$x$と$y$の間の値$c$で
\begin{align*}
\bunsuu{f(x)-f(y)}{x-y}&=-\bunsuu 12 \sin c
\end{align*}
となるものが存在する.したがって,
\[\bunsuu{|f(x)-f(y)|}{|x-y|} \leqq \bunsuu 12\]
$\therefore\ \ \ |f(x)-f(y)| \leqq \bunsuu 12|x-y|.$\hfill □
%\item $0\leqq  x \leqq \bunsuu {\pi}2$において,$g(x)$は単調増加,$f(x)$は単調減少であり,
%\[g(0)=0<\bunsuu 12=f(0)\ \ \ かつ\ \ \ g\left(\bunsuu{\pi}{2} \right)=\bunsuu{\pi}{2}>0=f\left(\bunsuu{\pi}{2}\right)\]
%よって,$g(x)=f(x)$となる$x$がただ1つ存在する.
%\item $-\bunsuu {\pi}2 < x<0$において,$g(x)<0<f(x)$\ \ \ より\ \ \ $g(x)\neq f(x)$.
%\item $|x| \geqq \bunsuu{\pi}{2}$において,$|g(x)| \geqq \bunsuu{\pi}{2} >1 \geqq |f(x)|$\ \ \ より\ \ \ $g(x)\neq f(x)$.
%\eenu
%以上より,$g(x)=f(x)$となる$x$はただ1つ存在する.\hfill □
\item $g(x)=x-f(x)$とおくと,$g'(x)=1+\bunsuu 12 \sin x >0$より,$g'(x)$は単調増加.\\
これと,$g(0)=-\bunsuu 12<0<\bunsuu{\pi}{2}=g\left(\bunsuu{\pi}{2}\right)$より,\\
$g(x)=0$となる$x$がただ1つ存在する.\\
これが$x=f(x)$となる唯一の$x$である.\hfill □
\item $f(x)=x$を満たす$x$を$\alpha$とおく.$x=a_{n-1}$,$y=\alpha$を(2)に適用すると
\[|a_{n}- \alpha| \leqq \bunsuu 12|a_{n-1}-\alpha|\]
これを繰り返し用いると
\[|a_{n}- \alpha| \leqq \bunsuu 12|a_{n-1}-\alpha|\leqq \cdots \leqq \left(\bunsuu 12\right)^{n} \left|a_0-\alpha \right|\]
\[0\leqq \left|a_n-\alpha \right| \leqq \left(\bunsuu 12\right)^{n} \left|a-\alpha \right|\xlongrightarrow[]{n\to \infty} 0\]
したがって$\dlim_{n\to \infty}a_n=\alpha$\hfill □
\eenu
\end{解答}

\newpage

蜘蛛の巣図;\\

\begin{zahyou}[ul=50mm,gentenhaiti={[wn]}](-0.1,2.1)(-0.1,2.1)
\def\Fx{1/2*cos(X)}
\def\Gx{X}
\YGraph<linethickness=1pt,color=red>\Fx
\YGraph<linethickness=1pt,color=blue>\Gx
%\YPointPut\Fx{\kval}[syaei=x,xlabel=0.07]{}
\teisuuretu{aval=1.3}
\teisuuretu{bval=1/2*cos(\aval)}
\teisuuretu{cval=1/2*cos(\bval)}
\teisuuretu{dval=1/2*cos(\cval)}
\teisuuretu{eval=1/2*cos(\dval)}
\teisuuretu{fval=1/2*cos(\eval)}
\teisuuretu{gval=1/2*cos(\fval)}
\teisuuretu{hval=1/2*cos(\gval)}
\teisuuretu{ival=1/2*cos(\hval)}
\teisuuretu{jval=1/2*cos(\ival)}
\teisuuretu{lval=1/2*cos(\jval)}
\teisuuretu{mval=1/2*cos(\lval)}
\teisuuretu{nval=1/2*cos(\mval)}
\teisuuretu{oval=1/2*cos(\nval)}
\teisuuretu{pval=1/2*cos(\oval)}
\YPoint\Fx{\aval}\A
\YPoint\Gx{\bval}\B
\YPoint\Fx{\bval}\C
\YPoint\Gx{\cval}\D
\YPoint\Fx{\cval}\E
\YPoint\Gx{\dval}\F
\YPoint\Fx{\dval}\G
\YPoint\Gx{\eval}\I
\YPoint\Fx{\eval}\J
\YPoint\Gx{\fval}\K
\YPoint\Fx{\fval}\L
\YPoint\Gx{\gval}\M
\YPoint\Fx{\gval}\N
\YPoint\Gx{\hval}\O
\YPoint\Fx{\hval}\P
\YPoint\Gx{\ival}\Q
\YPoint\Fx{\ival}\R
\YPoint\Gx{\jval}\S
\YPoint\Fx{\jval}\T
\YPoint\Gx{\lval}\U
\YPoint\Fx{\lval}\V
\YPoint\Gx{\mval}\W
\YPoint\Fx{\mval}\X
\YPoint\Gx{\nval}\Y
\YPoint\Fx{\nval}\Z
\ArrowLine{(\aval,0)}{\A}
\Drawline{\A\B\C\D\E\F\G\I\J\K\L\M\N\O\P\Q\R\S\T\U\V\W\X\Y\Z}
\YPointPut\Fx{\xmax-0.3}[ne]{\color{red} $y=\bunsuu 12 \cos x$}
\YPointPut\Gx{\xmax}[nw]{\color{blue} $y=x$}
%\KuromaruHankei{2pt}
%\Kuromaru\Z
\emathPut{\A}[syaei=x,xlabel=a]{}
\emathPut{\Z}[syaei=x,xlabel=\alpha]{}
%\Drawline{(1.5,1.5)(1.5,2.2)(2.2,2.2)(2.2,1.5)(1.5,1.5)}
\En{\Z}{0.06}
\end{zahyou}
\begin{zahyou*}[ul=500mm,gentenhaiti={[wn]},hidariyohaku=-8zw,sitayohaku=5zw](0.39,0.51)(0.39,0.51)
\def\Fx{1/2*cos(X)}
\def\Gx{X}
\YGraph<linethickness=1pt,color=red>\Fx
\YGraph<linethickness=1pt,color=blue>\Gx
%\YPointPut\Fx{\kval}[syaei=x,xlabel=0.07]{}
\teisuuretu{aval=1.3}
\YPoint\Fx{\aval}\A
%\ArrowLine{(\aval,0)}{\A}
\teisuuretu{bval=1/2*cos(\aval)}
\teisuuretu{cval=1/2*cos(\bval)}
\teisuuretu{dval=1/2*cos(\cval)}
\teisuuretu{eval=1/2*cos(\dval)}
\teisuuretu{fval=1/2*cos(\eval)}
\teisuuretu{gval=1/2*cos(\fval)}
\teisuuretu{hval=1/2*cos(\gval)}
\teisuuretu{ival=1/2*cos(\hval)}
\teisuuretu{jval=1/2*cos(\ival)}
\teisuuretu{lval=1/2*cos(\jval)}
\teisuuretu{mval=1/2*cos(\lval)}
\teisuuretu{nval=1/2*cos(\mval)}
\teisuuretu{oval=1/2*cos(\nval)}
\teisuuretu{pval=1/2*cos(\oval)}
\YPoint\Fx{\aval}\A
\YPoint\Gx{\bval}\B
\YPoint<yval=yi>\Fx{\bval}\C
\YPoint\Gx{\cval}\D
\YPoint\Fx{\cval}\E
\YPoint\Gx{\dval}\F
\YPoint\Fx{\dval}\G
\YPoint\Gx{\eval}\I
\YPoint\Fx{\eval}\J
\YPoint\Gx{\fval}\K
\YPoint\Fx{\fval}\L
\YPoint\Gx{\gval}\M
\YPoint\Fx{\gval}\N
\YPoint\Gx{\hval}\O
\YPoint\Fx{\hval}\P
\YPoint\Gx{\ival}\Q
\YPoint\Fx{\ival}\R
\YPoint\Gx{\jval}\S
\YPoint\Fx{\jval}\T
\YPoint\Gx{\lval}\U
\YPoint\Fx{\lval}\V
\YPoint\Gx{\mval}\W
\YPoint\Fx{\mval}\X
\YPoint\Gx{\nval}\Y
\YPoint\Fx{\nval}\Z
\Drawline{(\xmin,\yi)\D\E\F\G\I\J\K\L\M\N\O\P\Q\R\S\T\U\V\W\X\Y\Z}
\YPointPut\Fx{\xmax}[s]{\color{red} $y=\bunsuu 12 \cos x$}
\YPointPut\Gx{\xmax}[nw]{\color{blue} $y=x$}
%\KuromaruHankei{2pt}
%\Kuromaru\Z
%\emathPut{\Z}[es]{収束}
\En{\Z}{0.06}
\end{zahyou*}

\vfill

$\lim$を用いずにそのまま表してみる遊び;\\

{\huge 


\hfill $\bunsuu 12\cos\left(
\scalebox{0.8}{$\bunsuu 12\cos\left(
\scalebox{0.8}{$\bunsuu 12\cos\left(
\scalebox{0.8}{$\bunsuu 12\cos\left(
\scalebox{0.8}{$\bunsuu 12\cos\left(
\scalebox{0.8}{$\bunsuu 12\cos\left(
\scalebox{0.8}{$\bunsuu 12\cos\left(
\scalebox{0.8}{$\bunsuu 12\cos\left(
\scalebox{0.8}{$\bunsuu 12\cos\left(
\scalebox{0.8}{$\bunsuu 12\cos\left(
\scalebox{0.8}{$\bunsuu 12\cos\left(
\scalebox{0.8}{$\bunsuu 12\cos\left(
\scalebox{0.8}{$\bunsuu 12\cos\left(
\scalebox{0.8}{$\bunsuu 12\cos\left(
\scalebox{0.8}{$\bunsuu 12\cos\left(
\scalebox{0.8}{$\bunsuu 12\cos\left(
\scalebox{0.8}{$\bunsuu 12\cos\left(\ 
\right)$}\right)$}\right)$}\right)$}\right)$}\right)$}\right)$}\right)$}\right)$}\right)$}\right)$}\right)$}\right)$}\right)$}\right)$}\right)$}\right)=\alpha$\hfill 
}\\
\hfill ただし,$\alpha$は$\bunsuu 12\cos \alpha=\alpha$の解.


\hfill {\tiny ※ 消失点における値が初項$a$であり,$a$が任意の実数でこの等式は成り立つ.}

\vfill

\newpage

\fi

\vfill

\newpage

%{\Large\bb 「解けない漸化式」}

\bqu $a_1=1$,\ \ $a_{n+1}=1+\bunsuu{1}{1+a_n}$\ $(n=1,2,3,\cdots)$で定められる数列$\{a_n\}$を考える.
\benu
%\item 任意の自然数$n$に対して,$a_n \geqq 1$を示せ.
%\item $x=f(x)$の$x\geqq 1$の範囲にただ1つの解をもつことを証明せよ.
%\item $x \geqq 1$のとき,$f(x) \geqq 1$を示せ.
%\item 方程式$f(x)=x$を解け.
\item $f(x)=1+\bunsuu{1}{1+x}$とする.
\benu
\item 1以上の実数$x$で,$f(x)=x$を満たすものを求めよ.
\item 1以上の実数$a$,$b$に対し,常に$\left|f(a)-f(b)\right|\leqq \bunsuu 14|a-b|$が成り立つことを証明せよ.
\eenu
\item 極限値$\dlim_{n \to \infty}a_n$を求めよ.
\eenu
\equ

\ifkaisetu
\begin{解答}
\vspace{-2zw}
\benu
\item 
\benu
\item $x=1+\bunsuu{1}{1+x}$とすると,$x^2=2$.\ \ \ 
%\begin{align*}
%%x-1&=\bunsuu{1}{1+x}\\
%%x^2-1&=1\\
%
%\end{align*}
$x \geqq 1$\ \ より,$x={\bb \sq 2}$
\item $a=b$のとき,両辺=0より,成立するので,以下$a \neq b$とする.\\
$f'(x)=-\bunsuu{1}{(1+x)^2}$.{\bb 平均値の定理}より,$a$と$b$の間の値$c$で,
\begin{align*}
\bunsuu{f(a)-f(b)}{a-b}&=-\bunsuu{1}{(1+c)^2}
\end{align*}
を満たすものが存在する.このとき,
\[ \bunsuu{|f(a)-f(b)|}{|a-b|}=\bunsuu{1}{(1+c)^2}\leqq \bunsuu{1}{(1+1)^2}=\bunsuu 14\]
%c \geqq 1\ より\ \ \  \bunsuu{|f(a)-f(b)|}{|a-b|} & \\
ゆえに,\ \ $\left|f(a)-f(b)\right|\leqq \bunsuu 14|a-b|$.\hfill □
\eenu
\item $k$を自然数とし,$a_k \geqq 1$と仮定すると,$a_{k+1} =1+\bunsuu{1}{1+a_k} > 1$.\\
これと,$a_1 \geqq 1$より,任意の自然数$n$に対して,$a_n \geqq 1$である.\\
よって,$a=a_n$,$b=\sq 2$を(1)(b)に適用できて,
%\begin{align*}
\[\left|a_{n+1}-\sq 2\right| \leqq \bunsuu 14\left|a_n-\sq 2\right|\]%2以上の自然数$n$に対して,
\[\therefore\ \ \ 0\leqq \left|a_n-\sq 2\right| \leqq \left(\bunsuu 14\right)^{n-1} \left|a_1-\sq 2\right|\xlongrightarrow[]{n\to \infty} 0\]
%よって,
%\[\dlim_{n \to \infty}\left|a_n-\sq 2\right| \leqq \dlim_{n \to \infty}\left(\bunsuu 14\right)^{n-1} \left|a_1-\sq 2\right|=0\]
%より,$\dlim_{n \to \infty}\left|a_n-\sq 2\right|=0$.\\
したがって,$\dlim_{n \to \infty} a_n={\bb \sq 2}$.
\eenu
\end{解答}

\newpage

蜘蛛の巣図;\\

\begin{zahyou}[ul=25mm,gentenhaiti={[wn]}](-0.1,4.1)(-0.1,4.1)
\def\Fx{1+1/(1+X)}
\def\Gx{X}
\YGraph<linethickness=1pt,color=red,maxx=3.2>\Fx
\YGraph<linethickness=1pt,color=blue>\Gx
%\YPointPut\Fx{\kval}[syaei=x,xlabel=0.07]{}
\teisuuretu{aval=1}
\teisuuretu{bval=1+1/(1+\aval)}
\teisuuretu{cval=1+1/(1+\bval)}
\teisuuretu{dval=1+1/(1+\cval)}
\teisuuretu{eval=1+1/(1+\dval)}
\teisuuretu{fval=1+1/(1+\eval)}
\teisuuretu{gval=1+1/(1+\fval)}
\teisuuretu{hval=1+1/(1+\gval)}
\teisuuretu{ival=1+1/(1+\hval)}
\teisuuretu{jval=1+1/(1+\ival)}
\teisuuretu{lval=1+1/(1+\jval)}
\teisuuretu{mval=1+1/(1+\lval)}
\teisuuretu{nval=1+1/(1+\mval)}
\teisuuretu{oval=1+1/(1+\nval)}
\teisuuretu{pval=1+1/(1+\oval)}
\YPoint\Fx{\aval}\A
\YPoint\Gx{\bval}\B
\YPoint\Fx{\bval}\C
\YPoint\Gx{\cval}\D
\YPoint\Fx{\cval}\E
\YPoint\Gx{\dval}\F
\YPoint\Fx{\dval}\G
\YPoint\Gx{\eval}\I
\YPoint\Fx{\eval}\J
\YPoint\Gx{\fval}\K
\YPoint\Fx{\fval}\L
\YPoint\Gx{\gval}\M
\YPoint\Fx{\gval}\N
\YPoint\Gx{\hval}\O
\YPoint\Fx{\hval}\P
\YPoint\Gx{\ival}\Q
\YPoint\Fx{\ival}\R
\YPoint\Gx{\jval}\S
\YPoint\Fx{\jval}\T
\YPoint\Gx{\lval}\U
\YPoint\Fx{\lval}\V
\YPoint\Gx{\mval}\W
\YPoint\Fx{\mval}\X
\YPoint\Gx{\nval}\Y
\YPoint\Fx{\nval}\Z
\ArrowLine{(\aval,0)}{\A}
\Drawline{\A\B\C\D\E\F\G\I\J\K\L\M\N\O\P\Q\R\S\T\U\V\W\X\Y\Z}
\YPointPut\Fx{2}[s]{\color{red} $y=1+\bunsuu{1}{1+x}$}
\YPointPut\Gx{\xmax}[nw]{\color{blue} $y=x$}
%\KuromaruHankei{2pt}
%\Kuromaru\Z
\emathPut{\A}[syaei=x,xlabel=1]{}
\emathPut{\Z}[syaei=x,xlabel=\sq 2]{}
%\Drawline{(1.5,1.5)(1.5,2.2)(2.2,2.2)(2.2,1.5)(1.5,1.5)}
\En{\Z}{0.12}
\end{zahyou}
\begin{zahyou*}[ul=250mm,gentenhaiti={[wn]},hidariyohaku=-8zw,sitayohaku=5zw](1.29,1.53)(1.29,1.53)
\def\Fx{1+1/(1+X)}
\def\Gx{X}
\YGraph<linethickness=1pt,color=red>\Fx
\YGraph<linethickness=1pt,color=blue>\Gx
%\YPointPut\Fx{\kval}[syaei=x,xlabel=0.07]{}
\teisuuretu{aval=1}
\teisuuretu{bval=1+1/(1+\aval)}
\teisuuretu{cval=1+1/(1+\bval)}
\teisuuretu{dval=1+1/(1+\cval)}
\teisuuretu{eval=1+1/(1+\dval)}
\teisuuretu{fval=1+1/(1+\eval)}
\teisuuretu{gval=1+1/(1+\fval)}
\teisuuretu{hval=1+1/(1+\gval)}
\teisuuretu{ival=1+1/(1+\hval)}
\teisuuretu{jval=1+1/(1+\ival)}
\teisuuretu{lval=1+1/(1+\jval)}
\teisuuretu{mval=1+1/(1+\lval)}
\teisuuretu{nval=1+1/(1+\mval)}
\teisuuretu{oval=1+1/(1+\nval)}
\teisuuretu{pval=1+1/(1+\oval)}
\YPoint<yval=yi>\Fx{\aval}\A
\YPoint\Gx{\bval}\B
\YPoint\Fx{\bval}\C
\YPoint\Gx{\cval}\D
\YPoint\Fx{\cval}\E
\YPoint\Gx{\dval}\F
\YPoint\Fx{\dval}\G
\YPoint\Gx{\eval}\I
\YPoint\Fx{\eval}\J
\YPoint\Gx{\fval}\K
\YPoint\Fx{\fval}\L
\YPoint\Gx{\gval}\M
\YPoint\Fx{\gval}\N
\YPoint\Gx{\hval}\O
\YPoint\Fx{\hval}\P
\YPoint\Gx{\ival}\Q
\YPoint\Fx{\ival}\R
\YPoint\Gx{\jval}\S
\YPoint\Fx{\jval}\T
\YPoint\Gx{\lval}\U
\YPoint\Fx{\lval}\V
\YPoint\Gx{\mval}\W
\YPoint\Fx{\mval}\X
\YPoint\Gx{\nval}\Y
\YPoint\Fx{\nval}\Z
%\ArrowLine{(\aval,0)}{\A}
\Drawline{(\xmin,\yi)\B\C\D\E\F\G\I\J\K\L\M\N\O\P\Q\R\S\T\U\V\W\X\Y\Z}
\YPointPut\Fx{\xmax}[s]{\color{red} $y=1+\bunsuu{1}{1+x}$}
\YPointPut\Gx{\xmax}[nw]{\color{blue} $y=x$}
%\KuromaruHankei{2pt}
%\Kuromaru\Z
%\emathPut{\Z}[es]{収束}
%\Drawline{(1.5,1.5)(1.5,2.2)(2.2,2.2)(2.2,1.5)(1.5,1.5)}
\En{\Z}{0.12}
\end{zahyou*}

\vfill

$\lim$を用いずにそのまま表してみる遊び;\\

{\huge 

\hfill 
$1+\bunsuu{1}
{2+\scalebox{0.8}{$\bunsuu{1}{2+\scalebox{0.8}{$\bunsuu{1}
{2+\scalebox{0.8}{$\bunsuu{1}{2+\scalebox{0.8}{$\bunsuu{1}
{2+\scalebox{0.8}{$\bunsuu{1}{2+\scalebox{0.8}{$\bunsuu{1}
{2+\scalebox{0.8}{$\bunsuu{1}{2+\scalebox{0.8}{$\bunsuu{1}
{2+\scalebox{0.8}{$\bunsuu{1}{2+\scalebox{0.8}{$\bunsuu{1}
{2+\scalebox{0.8}{$\bunsuu{1}{2+\scalebox{0.8}{$\bunsuu{1}
{2+\scalebox{0.8}{$\bunsuu{1}{2+\scalebox{0.8}{$\bunsuu{1}
{2+\scalebox{0.8}{$\bunsuu{1}{2+\scalebox{0.8}{$\bunsuu{1}
{2+\scalebox{0.8}{$\bunsuu{1}{2+\scalebox{0.8}{$\bunsuu{1}
{2+\scalebox{0.8}{$\bunsuu{1}{2+\bunsuu{1}{}}$}}$}}$}}$}}$}}$}}$}}$}}$}}$}}$}}$}}$}}$}}$}}$}}$}}$}}$}}=\sq{2}$\hfill 
}

\vfill

\newpage

\fi

%\bqu
%極限値$\dlim_{n \to \infty}n\{\log(n+2)-\log n\}$を求めよ.
%\equ
%
%\ifkaisetu
%\begin{解答}
% $f(x)=\log x$とおくと,$f'(x)=\bunsuu 1x$.平均値の定理より,
%\[n<c<n+2,\ \ \ かつ\ \ \ \bunsuu{\log (n+2)-\log n}{(n+2)-n} = \bunsuu{1}{c}\]
%となる$c$が存在する.このとき,
%\begin{align*}
%a_n&=n\{\log (n+2)-\log n\} = \bunsuu{2n}{c}\\
%これと\ \ \ \bunsuu{2n}{n+2} &< \bunsuu{2n}{c} < \bunsuu{2n}{n}
%より,\ \ \ \bunsuu{2n}{n+2} < a_n < \bunsuu{2n}{n}
%\end{align*}
%であり,$n \to \infty$のとき,$(左辺) \to 2$かつ$(右辺) \to 2$.\\
%はさみうちの原理より,$\dlim_{n\to \infty} a_n=2$
%\end{解答}
%\fi


%%%\bqu $a_1>0$,\ \ $a_{n+1}=\bunsuu 12 e^{-a_n}$\ $(n=1,2,3,\cdots)$を満たす数列$\{a_n\}$がある.
%%%%\benu
%%%%\item $f(x)=$とする.
%%%\benu
%%%\item 等式$\bunsuu 12e^{-\alpha}=\alpha$を満たす正の数$\alpha$がただ1つ存在することを示せ.
%%%%\item 正の数$a$,$b$に対し,常に$\left|f(a)-f(b)\right|<\bunsuu 12|a-b|$が成り立つことを証明せよ.
%%%%\eenu
%%%\item $\dlim_{n \to \infty}a_n=\alpha$であることを証明せよ.
%%%\eenu
%%%\equ
%%%
%%%\ifkaisetu
%%%\begin{解答}
%%%\vspace{-2zw}
%%%\benu
%%%\item $g(x)=\bunsuu 12e^{-x}-x$とおく.\\
%%%$g'(x)=-\bunsuu 12 e^{-x}-1<0$より,$g(x)$は常に単調減少する.\\
%%%$g(0)=\bunsuu 12>0$かつ$\dlim_{x \to \infty} g(x)=-\infty$より,\\
%%%%$f(x)=\bunsuu 12e^{-x}$は単調減少かつ,$f(0)=\bunsuu 12$,$\dlim_{x\to \infty}f(x)=0$\\
%%%%一方,$g(x)=x$は単調増加かつ,$g(0)=0$,$\dlim_{x\to \infty}g(x)=\infty$\\
%%%%よって,
%%%$g(\alpha)=0$となる正の数$\alpha$がただ1つ存在する.\\
%%%この$\alpha$が,$\bunsuu 12e^{-\alpha}=\alpha$を満たす唯一の正の数に他ならない.
%%%\item $f(x)=\bunsuu 12 e^{-x}$とおく.\ \ $f'(x)=-\bunsuu 12 e^{-x}$.\\
%%%{\bb 平均値の定理}より,$a_n$と$\alpha$の間の値$c$で
%%%\[\bunsuu{f(a_{n})-f(\alpha)}{a_n-\alpha} =f'(c)\ \ \ すなわち\ \ \ \bunsuu{a_{n+1}-\alpha}{a_n-\alpha} =-\bunsuu 12 e^{-c}\]
%%%を満たすものが存在する.\\
%%%$a_n>0$かつ$\alpha>0$より,$c>0$であることに注意すると,
%%%\[\bunsuu{|a_{n+1}-\alpha|}{|a_n-\alpha|} =\bunsuu 12 e^{-c} <\bunsuu 12\ \ \ \]
%%%\[\therefore\ \ \ |a_{n+1}-\alpha| <\bunsuu 12 |a_n-\alpha|\]
%%%これを繰り返し用いると,
%%%\[\therefore\ \ \ |a_n-\alpha| < \bunsuu 12 |a_{n-1}-\alpha| < \cdots < \left(\bunsuu 12\right)^{n-1} |a_1-\alpha|\]
%%%\[\therefore\ \ \ 0\leqq |a_n-\alpha| <  \left(\bunsuu 12\right)^{n-1} |a_1-\alpha| \xlongrightarrow[]{n\to \infty} 0\]
%%%したがって,\ \ \ $\dlim_{n\to \infty}a_n=\alpha$.
%%%\hfill □
%%%\eenu
%%%\end{解答}
%%%
%%%\newpage
%%%
%%%蜘蛛の巣図;\\
%%%
%%%\begin{zahyou}[ul=50mm,gentenhaiti={[wn]}](-0.1,2.1)(-0.1,2.1)
%%%\def\Fx{1/2*exp(-X)}
%%%\def\Gx{X}
%%%\YGraph<linethickness=1pt,color=red>\Fx
%%%\YGraph<linethickness=1pt,color=blue>\Gx
%%%%\YPointPut\Fx{\kval}[syaei=x,xlabel=0.07]{}
%%%\teisuuretu{aval=1.5}
%%%\teisuuretu{bval=1/2*exp(-\aval)}
%%%\teisuuretu{cval=1/2*exp(-\bval)}
%%%\teisuuretu{dval=1/2*exp(-\cval)}
%%%\teisuuretu{eval=1/2*exp(-\dval)}
%%%\teisuuretu{fval=1/2*exp(-\eval)}
%%%\teisuuretu{gval=1/2*exp(-\fval)}
%%%\teisuuretu{hval=1/2*exp(-\gval)}
%%%\teisuuretu{ival=1/2*exp(-\hval)}
%%%\teisuuretu{jval=1/2*exp(-\ival)}
%%%\teisuuretu{lval=1/2*exp(-\jval)}
%%%\teisuuretu{mval=1/2*exp(-\lval)}
%%%\teisuuretu{nval=1/2*exp(-\mval)}
%%%\teisuuretu{oval=1/2*exp(-\nval)}
%%%\teisuuretu{pval=1/2*exp(-\oval)}
%%%\YPoint\Fx{\aval}\A
%%%\YPoint\Gx{\bval}\B
%%%\YPoint\Fx{\bval}\C
%%%\YPoint\Gx{\cval}\D
%%%\YPoint\Fx{\cval}\E
%%%\YPoint\Gx{\dval}\F
%%%\YPoint\Fx{\dval}\G
%%%\YPoint\Gx{\eval}\I
%%%\YPoint\Fx{\eval}\J
%%%\YPoint\Gx{\fval}\K
%%%\YPoint\Fx{\fval}\L
%%%\YPoint\Gx{\gval}\M
%%%\YPoint\Fx{\gval}\N
%%%\YPoint\Gx{\hval}\O
%%%\YPoint\Fx{\hval}\P
%%%\YPoint\Gx{\ival}\Q
%%%\YPoint\Fx{\ival}\R
%%%\YPoint\Gx{\jval}\S
%%%\YPoint\Fx{\jval}\T
%%%\YPoint\Gx{\lval}\U
%%%\YPoint\Fx{\lval}\V
%%%\YPoint\Gx{\mval}\W
%%%\YPoint\Fx{\mval}\X
%%%\YPoint\Gx{\nval}\Y
%%%\YPoint\Fx{\nval}\Z
%%%\ArrowLine{(\aval,0)}{\A}
%%%\Drawline{\A\B\C\D\E\F\G\I\J\K\L\M\N\O\P\Q\R\S\T\U\V\W\X\Y\Z}
%%%\YPointPut\Fx{\xmax}[n]{\color{red} $y=\bunsuu 12 e^{-x}$}
%%%\YPointPut\Gx{\xmax}[nw]{\color{blue} $y=x$}
%%%%\KuromaruHankei{2pt}
%%%%\Kuromaru\Z
%%%\emathPut{\A}[syaei=x,xlabel=a_1]{}
%%%\emathPut{\Z}[syaei=x,xlabel=\alpha]{}
%%%%\Drawline{(1.5,1.5)(1.5,2.2)(2.2,2.2)(2.2,1.5)(1.5,1.5)}
%%%\En{\Z}{0.06}
%%%\end{zahyou}
%%%\begin{zahyou*}[ul=500mm,gentenhaiti={[wn]},hidariyohaku=-8zw,sitayohaku=5zw](0.29,0.41)(0.29,0.41)
%%%\def\Fx{1/2*exp(-X)}
%%%\def\Gx{X}
%%%\YGraph<linethickness=1pt,color=red>\Fx
%%%\YGraph<linethickness=1pt,color=blue>\Gx
%%%%\YPointPut\Fx{\kval}[syaei=x,xlabel=0.07]{}
%%%\teisuuretu{aval=1.5}
%%%\teisuuretu{bval=1/2*exp(-\aval)}
%%%\teisuuretu{cval=1/2*exp(-\bval)}
%%%\teisuuretu{dval=1/2*exp(-\cval)}
%%%\teisuuretu{eval=1/2*exp(-\dval)}
%%%\teisuuretu{fval=1/2*exp(-\eval)}
%%%\teisuuretu{gval=1/2*exp(-\fval)}
%%%\teisuuretu{hval=1/2*exp(-\gval)}
%%%\teisuuretu{ival=1/2*exp(-\hval)}
%%%\teisuuretu{jval=1/2*exp(-\ival)}
%%%\teisuuretu{lval=1/2*exp(-\jval)}
%%%\teisuuretu{mval=1/2*exp(-\lval)}
%%%\teisuuretu{nval=1/2*exp(-\mval)}
%%%\teisuuretu{oval=1/2*exp(-\nval)}
%%%\teisuuretu{pval=1/2*exp(-\oval)}
%%%\YPoint\Fx{\aval}\A
%%%\YPoint\Gx{\bval}\B
%%%\YPoint\Fx{\bval}\C
%%%\YPoint\Gx{\cval}\D
%%%\YPoint<yval=yi>\Fx{\cval}\E
%%%\YPoint\Gx{\dval}\F
%%%\YPoint\Fx{\dval}\G
%%%\YPoint\Gx{\eval}\I
%%%\YPoint\Fx{\eval}\J
%%%\YPoint\Gx{\fval}\K
%%%\YPoint\Fx{\fval}\L
%%%\YPoint\Gx{\gval}\M
%%%\YPoint\Fx{\gval}\N
%%%\YPoint\Gx{\hval}\O
%%%\YPoint\Fx{\hval}\P
%%%\YPoint\Gx{\ival}\Q
%%%\YPoint\Fx{\ival}\R
%%%\YPoint\Gx{\jval}\S
%%%\YPoint\Fx{\jval}\T
%%%\YPoint\Gx{\lval}\U
%%%\YPoint\Fx{\lval}\V
%%%\YPoint\Gx{\mval}\W
%%%\YPoint\Fx{\mval}\X
%%%\YPoint\Gx{\nval}\Y
%%%\YPoint\Fx{\nval}\Z
%%%%\ArrowLine{(\aval,0)}{\A}
%%%\Drawline{(\xmax,\yi)
%%%\F\G\I\J\K\L\M\N\O\P\Q\R\S\T\U\V\W\X\Y\Z}
%%%\YPointPut\Fx{\xmax-0.03}[ne]{\color{red} $y=\bunsuu 12 e^{-x}$}
%%%\YPointPut\Gx{\xmax-0.03}[nw]{\color{blue} $y=x$}
%%%%\KuromaruHankei{2pt}
%%%%\Kuromaru\Z
%%%%\emathPut{\Z}[es]{収束}
%%%%\Drawline{(1.5,1.5)(1.5,2.2)(2.2,2.2)(2.2,1.5)(1.5,1.5)}
%%%\En{\Z}{0.06}
%%%\end{zahyou*}
%%%
%%%\vfill
%%%
%%%
%%%$\lim$を用いずにそのまま表してみる遊び;\\
%%%
%%%{\huge 
%%%
%%%
%%%\hfill 
%%%$\bunsuu 12e^{
%%%\scalebox{0.7}{$-\bunsuu 12e^{
%%%\scalebox{0.7}{$-\bunsuu 12e^{
%%%\scalebox{0.7}{$-\bunsuu 12e^{
%%%\scalebox{0.7}{$-\bunsuu 12e^{
%%%\scalebox{0.7}{$-\bunsuu 12e^{
%%%\scalebox{0.7}{$-\bunsuu 12e^{
%%%\scalebox{0.7}{$-\bunsuu 12e^{
%%%\scalebox{0.7}{$-\bunsuu 12e^{
%%%\scalebox{0.7}{$-\bunsuu 12e^{
%%%\scalebox{0.7}{$-\bunsuu 12e^{
%%%\scalebox{0.7}{$-\bunsuu 12e^{
%%%\scalebox{0.7}{$-\bunsuu 12e^{
%%%\scalebox{0.7}{$-\bunsuu 12e^{
%%%\scalebox{0.7}{$-\bunsuu 12e^{\ 
%%%}$}}$}}$}}$}}$}}$}}$}}$}}$}}$}}$}}$}}$}}$}}=\alpha$\hfill 
%%%}
%%%ただし,$\alpha$は$\bunsuu 12 e^{-\alpha}=\alpha$の解.
%%%\vfill
%%%
%%%\newpage
%%%
%%%\fi

%\bqu 
%\benu
%\item %$x>4$のとき,不等式$\sq{2}^x>x$が成立することを示せ.また,
%方程式$x=\sq{2}^x$を満たす実数$x$は$x=2$と$x=4$以外にないことを示せ.
%\item $0<x<2$のとき,不等式$\bunsuu{2-\sq{2}^x}{2-x}< \log 2 $が成立することを示せ.
%\item 
\bqu
$a_1=\sq 2,\ \ a_{n+1} =\sq{2}^{a_n}\ \ \ (n=1,2,3,\cdots)$
によって定められる数列$\{a_n\}$の極限値$\dlim_{n \to \infty}a_n$を求めよ.
\equ

\ifkaisetu
\begin{解答}
$f(x)=\sq{2}^{x}$とおくと,$f(2)=2$,$f(4)=4$である.\\
$y=f(x)$のグラフは常に下に凸であり,$y=x$のグラフは直線であるから,\\
この2つの交点は多くて2点しかない.\\
以上より,$f(\alpha)=\alpha$となる実数$\alpha$は$\alpha=2,4$しかない.

さて,$k$を自然数とし,$a_k<2$と仮定すると,$a_{k+1}=\sq 2^{a_k} < \sq 2^{2}=2$.\\
これと$a_1=\sq 2 < 2$より,任意の自然数$n$に対して$a_n<2$.\\
$f'(x)={\sq 2}^{x} \log \sq{2} $.\\
{\bb 平均値の定理}より,$a_n<c<2$なる実数$c$で
\[\bunsuu{f(2)-f(a_{n})}{2-a_n} =f'(c)\ \ \ すなわち\ \ \ \bunsuu{2-a_{n+1}}{2-a_n} ={\sq 2}^{c} \log \sq{2}\]
を満たすものが存在する.ここで,
\[{\sq 2}^{c} \log \sq{2}<{\sq 2}^{2} \log \sq{2}=2\log \sq{2}=\log \sq{2}^2=\log 2\]
に注意すると,
\[\bunsuu{2-a_{n+1}}{2-a_n} < \log 2\ \ \ すなわち\ \ \ 2-a_{n+1} < \log 2(2-a_n)\]
これをくり返し用いると,
\[\therefore\ \ \ 2-a_n < \log 2  (2-a_{n-1})<\cdots <\left(\log 2 \right)^{n-1} (2-a_1)\]
\[\therefore\ \ \ 0\leqq 2-a_n < \left(\log 2 \right)^{n-1} (2-a_1)\xlongrightarrow[]{n\to \infty} 0.\]
したがって\ \ $\dlim_{n\to \infty}a_n={\bb 2} $
\end{解答}

\newpage

 蜘蛛の巣図;\\
 
\begin{zahyou}[ul=12.5mm,gentenhaiti={[wn]}](-0.1,8.1)(-0.1,8.1)
\def\Fx{exp(log(sqrt(2))*X)}
\def\Gx{X}
\YGraph<linethickness=1pt,color=red>\Fx
\YGraph<linethickness=1pt,color=blue>\Gx
%\YPointPut\Fx{\kval}[syaei=x,xlabel=0.07]{}
\teisuuretu{aval=sqrt(2)}
\teisuuretu{bval=exp(log(sqrt(2))*\aval)}
\teisuuretu{cval=exp(log(sqrt(2))*\bval)}
\teisuuretu{dval=exp(log(sqrt(2))*\cval)}
\teisuuretu{eval=exp(log(sqrt(2))*\dval)}
\teisuuretu{fval=exp(log(sqrt(2))*\eval)}
\teisuuretu{gval=exp(log(sqrt(2))*\fval)}
\teisuuretu{hval=exp(log(sqrt(2))*\gval)}
\teisuuretu{ival=exp(log(sqrt(2))*\hval)}
\teisuuretu{jval=exp(log(sqrt(2))*\ival)}
\teisuuretu{lval=exp(log(sqrt(2))*\jval)}
\teisuuretu{mval=exp(log(sqrt(2))*\lval)}
\teisuuretu{nval=exp(log(sqrt(2))*\mval)}
\teisuuretu{oval=exp(log(sqrt(2))*\nval)}
\teisuuretu{pval=exp(log(sqrt(2))*\oval)}
\YPoint\Fx{\aval}\A
\YPoint\Gx{\bval}\B
\YPoint\Fx{\bval}\C
\YPoint\Gx{\cval}\D
\YPoint\Fx{\cval}\E
\YPoint\Gx{\dval}\F
\YPoint\Fx{\dval}\G
\YPoint\Gx{\eval}\I
\YPoint\Fx{\eval}\J
\YPoint\Gx{\fval}\K
\YPoint\Fx{\fval}\L
\YPoint\Gx{\gval}\M
\YPoint\Fx{\gval}\N
\YPoint\Gx{\hval}\O
\YPoint\Fx{\hval}\P
\YPoint\Gx{\ival}\Q
\YPoint\Fx{\ival}\R
\YPoint\Gx{\jval}\S
\YPoint\Fx{\jval}\T
\YPoint\Gx{\lval}\U
\YPoint\Fx{\lval}\V
\YPoint\Gx{\mval}\W
\YPoint\Fx{\mval}\X
\YPoint\Gx{\nval}\Y
\YPoint\Fx{\nval}\Z
\ArrowLine{(\aval,0)}{\A}
\Drawline{\A\B\C\D\E\F\G\I\J\K\L\M\N\O\P\Q\R\S\T\U\V\W\X\Y\Z}
\YPointPut\Fx{5.5}[w]{\color{red} $y=\sqrt{2}^{x}$}
\YPointPut\Gx{\xmax-1}[se]{\color{blue} $y=x$}
%\KuromaruHankei{2pt}
%\Kuromaru\Z
\emathPut{\A}[syaei=x,xlabel=\sq 2]{}
\emathPut{\Z}[syaei=x,xlabel=2]{}
%\Drawline{(1.5,1.5)(1.5,2.2)(2.2,2.2)(2.2,1.5)(1.5,1.5)}
\En{\Z}{0.25}
\end{zahyou}
 \begin{zahyou*}[ul=125mm,gentenhaiti={[wn]},hidariyohaku=-8zw,sitayohaku=5zw](1.81,2.21)(1.81,2.21)
\def\Fx{exp(log(sqrt(2))*X)}
\def\Gx{X}
\YGraph<linethickness=1pt,color=red>\Fx
\YGraph<linethickness=1pt,color=blue>\Gx
%\YPointPut\Fx{\kval}[syaei=x,xlabel=0.07]{}
\teisuuretu{aval=sqrt(2)}
\teisuuretu{bval=exp(log(sqrt(2))*\aval)}
\teisuuretu{cval=exp(log(sqrt(2))*\bval)}
\teisuuretu{dval=exp(log(sqrt(2))*\cval)}
\teisuuretu{eval=exp(log(sqrt(2))*\dval)}
\teisuuretu{fval=exp(log(sqrt(2))*\eval)}
\teisuuretu{gval=exp(log(sqrt(2))*\fval)}
\teisuuretu{hval=exp(log(sqrt(2))*\gval)}
\teisuuretu{ival=exp(log(sqrt(2))*\hval)}
\teisuuretu{jval=exp(log(sqrt(2))*\ival)}
\teisuuretu{lval=exp(log(sqrt(2))*\jval)}
\teisuuretu{mval=exp(log(sqrt(2))*\lval)}
\teisuuretu{nval=exp(log(sqrt(2))*\mval)}
\teisuuretu{oval=exp(log(sqrt(2))*\nval)}
\teisuuretu{pval=exp(log(sqrt(2))*\oval)}
\YPoint\Fx{\aval}\A
\YPoint\Gx{\bval}\B
\YPoint\Fx{\bval}\C
\YPoint\Gx{\cval}\D
\YPoint\Fx{\cval}\E
\YPoint<xval=xi>\Gx{\dval}\F
\YPoint\Fx{\dval}\G
\YPoint\Gx{\eval}\I
\YPoint\Fx{\eval}\J
\YPoint\Gx{\fval}\K
\YPoint\Fx{\fval}\L
\YPoint\Gx{\gval}\M
\YPoint\Fx{\gval}\N
\YPoint\Gx{\hval}\O
\YPoint\Fx{\hval}\P
\YPoint\Gx{\ival}\Q
\YPoint\Fx{\ival}\R
\YPoint\Gx{\jval}\S
\YPoint\Fx{\jval}\T
\YPoint\Gx{\lval}\U
\YPoint\Fx{\lval}\V
\YPoint\Gx{\mval}\W
\YPoint\Fx{\mval}\X
\YPoint\Gx{\nval}\Y
\YPoint\Fx{\nval}\Z
%\ArrowLine{(\aval,0)}{\A}
\Drawline{(\xi,\ymin)\G\I\J\K\L\M\N\O\P\Q\R\S\T\U\V\W\X\Y\Z}
\YPointPut\Fx{\xmax-0.1}[se]{\color{red} $y=\sqrt{2}^{x}$}
\YPointPut\Gx{\xmax-0.1}[nw]{\color{blue} $y=x$}
%\KuromaruHankei{2pt}
%\Kuromaru\Z
%\emathPut{\Z}[es]{収束}
%\Drawline{(1.5,1.5)(1.5,2.2)(2.2,2.2)(2.2,1.5)(1.5,1.5)}
\En{\Z}{0.25}
\end{zahyou*}

\vfill

$\lim$を用いずにそのまま表してみる遊び;\\

{\huge 


\hfill 
$\sqrt{2}^{
\scalebox{0.7}{$\sqrt{2}^{
\scalebox{0.7}{$\sqrt{2}^{
\scalebox{0.7}{$\sqrt{2}^{
\scalebox{0.7}{$\sqrt{2}^{
\scalebox{0.7}{$\sqrt{2}^{
\scalebox{0.7}{$\sqrt{2}^{
\scalebox{0.7}{$\sqrt{2}^{
\scalebox{0.7}{$\sqrt{2}^{
\scalebox{0.7}{$\sqrt{2}^{
\scalebox{0.7}{$\sqrt{2}^{
\scalebox{0.7}{$\sqrt{2}^{
\scalebox{0.7}{$\sqrt{2}^{
\scalebox{0.7}{$\sqrt{2}^{
\scalebox{0.7}{$\sqrt{2}^{
\scalebox{0.7}{$\sqrt{2}^{
\scalebox{0.7}{$\sqrt{2}^{\ }
$}}$}}$}}$}}$}}$}}$}}$}}$}}$}}$}}$}}$}}$}}$}}$}}
=2$\hfill 
}

\vfill

\newpage

\fi

\bqu
関数$f(x)$を$f(x)=\bunsuu 12x\{1+e^{-2(x-1)}\}$とする.
\benu
\item $x>\bunsuu 12$ならば$0 \leqq f'(x) <\bunsuu 12$であることを示せ.
\item $x_0$を正の数とするとき,数列$\{x_n\}$\ $(n=0,1,\cdots)$を$x_{n+1}=f(x_n)$によって定める.\\$x_0>\bunsuu 12$であれば,$\dlim_{n \to\infty}x_n=1$であることを示せ.\ifkaisetu \hfill(東大)\fi
\eenu
\equ

\ifkaisetu
\begin{解答}
\vspace{-2zw}
\benu
\item $f'(x)=\bunsuu12 \{1\cdot(1+e^{-2(x-1)})+x\cdot (-2e^{-2(x-1)})\}=\bunsuu 12\{1+e^{-2(x-1)}-2xe^{-2(x-1)}\}$\\
$f''(x)=\bunsuu 12 \{-2e^{-2(x-1)}-2e^{-2(x-1)}+4xe^{-2(x-1)}\}=2(x-1)e^{-2(x-1)}$.\\
よって,$f(x)$の増減は次のようになる.\vspace{-0.5zw}
\[\begin{array}{c|ccccccccccc}
\phantom{\bunsuu 11}x&\frac 12	&\cdots	&1	&\cdots	& (\infty)	\\\hline
\phantom{\bunsuu 11}f''(x)& 		&-		&0	&+		&		\\\hline
\phantom{\bunsuu 11}f'(x)& \frac 12&\SE	&0	&\NE	&(\frac 12)\\
\end{array}
\]
\vspace{-0.5zw}
したがって,$x>\bunsuu 12$ならば$0 \leqq f'(x) <\bunsuu 12$.\hfill □
\item 
(1)の結果より,$x>\bunsuu 12$において,$f(x)$は単調増加する.
これと
\[f\left(\bunsuu 12\right)=\bunsuu 12 \cdot\bunsuu 12\left\{1+e^{-2\cdot (\frac12-1)}\right\}=\bunsuu{1+e}{4}>\bunsuu{1+1}{4}=\bunsuu 12\]
より,$x>\bunsuu 12$ならば$f(x)>\bunsuu 12$である.\\
ゆえに,$k$を0以上の整数として,$x_{k}>\bunsuu 12$を仮定すると,$x_{k+1}=f(x_{k})>\bunsuu 12$.\\
これと$a_0>\bunsuu 12$より,任意の0以上の整数$n$に対して,$x_n>\bunsuu 12$が成立する.\\
{\bb 平均値の定理}より,1と$x_n$の間の値$c$で,
\[\bunsuu{f(x_n)-f(1)}{x_n-1}=f'(c)\ \ \ すなわち\ \ \ \bunsuu{x_{n+1}-1}{x_n-1}=f'(c)\]
を満たすものが存在する.\\
%このとき,
%\[\bunsuu{|x_{n+1}-1|}{|x_n-1|}=f'(c)\]
ここで,$1>\bunsuu 12$かつ$x_n>\bunsuu 12$より,$c > \bunsuu 12$なので,(1)の結果から%$0\leqq f'(c) < \bunsuu 12$.
\[\bunsuu{|x_{n+1}-1|}{|x_n-1|}=f'(c)<\bunsuu 12\ \ \ \]
\[\therefore\ \ \ |x_{n+1}-1|<\bunsuu 12|x_n-1|.\]
これを繰り返し用いると,
\[|x_{n}-1|<\bunsuu12 |x_{n-1}-1|<\cdots<\left(\bunsuu12\right)^n |x_{0}-1|\]
\[\therefore\ \ \ 0\leqq |x_{n}-1|<\left(\bunsuu12\right)^n |x_{0}-1|\xlongrightarrow[]{n\to \infty} 0\]
したがって\ \ $\dlim_{n\to \infty}x_n=1 $\hfill □
\eenu
\end{解答}
%\vfill
\newpage

 蜘蛛の巣図;\\
 
 \begin{zahyou}[ul=2.5mm,gentenhaiti={[wn]},yokozikuhaiti={[en]}](-1,41)(-1,41)
\def\Fx{1/2*X*(1+exp(-2*(X-1)))}
\def\Gx{X}
\YGraph<linethickness=1pt,color=red,maxx=30>\Fx
\YGraph<linethickness=1pt,color=blue>\Gx
%\YPointPut\Fx{\kval}[syaei=x,xlabel=0.07]{}
\teisuuretu{aval=20}
\teisuuretu{bval=1/2*\aval*(1+exp(-2*(\aval-1)))}
\teisuuretu{cval=1/2*\bval*(1+exp(-2*(\bval-1)))}
\teisuuretu{dval=1/2*\cval*(1+exp(-2*(\cval-1)))}
\teisuuretu{eval=1/2*\dval*(1+exp(-2*(\dval-1)))}
\teisuuretu{fval=1/2*\eval*(1+exp(-2*(\eval-1)))}
\teisuuretu{gval=1/2*\fval*(1+exp(-2*(\fval-1)))}
\teisuuretu{hval=1/2*\gval*(1+exp(-2*(\gval-1)))}
\teisuuretu{ival=1/2*\hval*(1+exp(-2*(\hval-1)))}
\teisuuretu{jval=1/2*\ival*(1+exp(-2*(\ival-1)))}
\teisuuretu{lval=1/2*\jval*(1+exp(-2*(\jval-1)))}
\teisuuretu{mval=1/2*\lval*(1+exp(-2*(\lval-1)))}
\teisuuretu{nval=1/2*\mval*(1+exp(-2*(\mval-1)))}
\teisuuretu{oval=1/2*\nval*(1+exp(-2*(\nval-1)))}
\teisuuretu{pval=1/2*\oval*(1+exp(-2*(\oval-1)))}
\YPoint\Fx{\aval}\A
\YPoint\Gx{\bval}\B
\YPoint\Fx{\bval}\C
\YPoint\Gx{\cval}\D
\YPoint\Fx{\cval}\E
\YPoint\Gx{\dval}\F
\YPoint\Fx{\dval}\G
\YPoint\Gx{\eval}\I
\YPoint\Fx{\eval}\J
\YPoint\Gx{\fval}\K
\YPoint\Fx{\fval}\L
\YPoint\Gx{\gval}\M
\YPoint\Fx{\gval}\N
\YPoint\Gx{\hval}\O
\YPoint\Fx{\hval}\P
\YPoint\Gx{\ival}\Q
\YPoint\Fx{\ival}\R
\YPoint\Gx{\jval}\S
\YPoint\Fx{\jval}\T
\YPoint\Gx{\lval}\U
\YPoint\Fx{\lval}\V
\YPoint\Gx{\mval}\W
\YPoint\Fx{\mval}\X
\YPoint\Gx{\nval}\Y
\YPoint\Fx{\nval}\Z
\ArrowLine{(\aval,0)}{\A}
\Drawline{\A\B\C\D\E\F\G\I\J\K\L\M\N\O\P\Q\R\S\T\U\V\W\X\Y\Z}
\YPointPut\Fx{25}[nw]{\color{red} $y=f(x)$}
\YPointPut\Gx{30}[nw]{\color{blue} $y=x$}
%\KuromaruHankei{2pt}
%\Kuromaru\Z
\emathPut{\A}[syaei=x,xlabel=x_0]{}
\emathPut{\Z}[syaei=x,xlabel=1]{}
%\Drawline{(1.5,1.5)(1.5,2.2)(2.2,2.2)(2.2,1.5)(1.5,1.5)}
\En{\Z}{1.2}
\end{zahyou}
\begin{zahyou}[ul=25mm,gentenhaiti={[wn]},hidariyohaku=-8zw,sitayohaku=5zw](-0.1,2.1)(-0.1,2.1)
\def\Fx{1/2*X*(1+exp(-2*(X-1)))}
\def\Gx{X}
\YGraph<linethickness=1pt,color=red>\Fx
\YGraph<linethickness=1pt,color=blue>\Gx
%\YPointPut\Fx{\kval}[syaei=x,xlabel=0.07]{}
\teisuuretu{aval=20}
\teisuuretu{bval=1/2*\aval*(1+exp(-2*(\aval-1)))}
\teisuuretu{cval=1/2*\bval*(1+exp(-2*(\bval-1)))}
\teisuuretu{dval=1/2*\cval*(1+exp(-2*(\cval-1)))}
\teisuuretu{eval=1/2*\dval*(1+exp(-2*(\dval-1)))}
\teisuuretu{fval=1/2*\eval*(1+exp(-2*(\eval-1)))}
\teisuuretu{gval=1/2*\fval*(1+exp(-2*(\fval-1)))}
\teisuuretu{hval=1/2*\gval*(1+exp(-2*(\gval-1)))}
\teisuuretu{ival=1/2*\hval*(1+exp(-2*(\hval-1)))}
\teisuuretu{jval=1/2*\ival*(1+exp(-2*(\ival-1)))}
\teisuuretu{lval=1/2*\jval*(1+exp(-2*(\jval-1)))}
\teisuuretu{mval=1/2*\lval*(1+exp(-2*(\lval-1)))}
\teisuuretu{nval=1/2*\mval*(1+exp(-2*(\mval-1)))}
\teisuuretu{oval=1/2*\nval*(1+exp(-2*(\nval-1)))}
\teisuuretu{pval=1/2*\oval*(1+exp(-2*(\oval-1)))}
\YPoint\Fx{\aval}\A
\YPoint\Gx{\bval}\B
\YPoint\Fx{\bval}\C
\YPoint\Gx{\cval}\D
\YPoint\Fx{\cval}\E
\YPoint\Gx{\dval}\F
\YPoint<yval=yi>\Fx{\dval}\G
\YPoint\Gx{\eval}\I
\YPoint\Fx{\eval}\J
\YPoint\Gx{\fval}\K
\YPoint\Fx{\fval}\L
\YPoint\Gx{\gval}\M
\YPoint\Fx{\gval}\N
\YPoint\Gx{\hval}\O
\YPoint\Fx{\hval}\P
\YPoint\Gx{\ival}\Q
\YPoint\Fx{\ival}\R
\YPoint\Gx{\jval}\S
\YPoint\Fx{\jval}\T
\YPoint\Gx{\lval}\U
\YPoint\Fx{\lval}\V
\YPoint\Gx{\mval}\W
\YPoint\Fx{\mval}\X
\YPoint\Gx{\nval}\Y
\YPoint\Fx{\nval}\Z
%\ArrowLine{(\aval,0)}{\A}
\Drawline{(\xmax,\yi)\I\J\K\L\M\N\O\P\Q\R\S\T\U\V\W\X\Y\Z}
\YPointPut\Fx{\xmax-0.5}[s]{\color{red} $y=f(x)$}
\YPointPut\Gx{\xmax}[nw]{\color{blue} $y=x$}
%\KuromaruHankei{2pt}
%\Kuromaru\Z
\emathPut{\Z}[syaei=x,xlabel=1]{}
%\Drawline{(1.5,1.5)(1.5,2.2)(2.2,2.2)(2.2,1.5)(1.5,1.5)}
\En{\Z}{1.2}
\end{zahyou}

 \vfill

$\lim$を用いずにそのまま表してみる遊び;\\

$f(x)=\bunsuu 12x\{1+e^{-2(x-1)}\}$とすると\\

{\huge 


\hfill 
$f\left(
\scalebox{0.8}{$f\left(
\scalebox{0.8}{$f\left(
\scalebox{0.8}{$f\left(
\scalebox{0.8}{$f\left(
\scalebox{0.8}{$f\left(
\scalebox{0.8}{$f\left(
\scalebox{0.8}{$f\left(
\scalebox{0.8}{$f\left(
\scalebox{0.8}{$f\left(
\scalebox{0.8}{$f\left(
\scalebox{0.8}{$f\left(
\scalebox{0.8}{$f\left(
\scalebox{0.8}{$f\left(
\scalebox{0.8}{$f\left(
\scalebox{0.8}{$f\left(
\scalebox{0.8}{$f\left(
\scalebox{0.8}{$f\left(\ 
\right)$}\right)$}\right)$}\right)$}\right)$}\right)$}\right)$}\right)$}\right)$}\right)$}\right)$}\right)$}\right)$}\right)$}\right)$}\right)$}\right)$}\right)=1$\hfill 
}
\\

\hfill {\tiny ※ 消失点における値が初項$x_0$であり,この値が$x_0>\bunsuu 12$を満たすときにこの等式は成立する.}

\vfill

\fi
%\newpage

%{\bb チャレンジ問題}\\

%\bqu
%{\bb 追加問題} \\
%$a$を1より大きい定数とする.微分可能な関数$f(x)$が$f(a)=af(1)$を満たすとき,曲線$y=f(x)$の接線で原点$(0,0)$を通るものが存在することを示せ.
%\hfill(2021京大 理系第6問\ 問2)
%\equ

%\bqu
%次の問いに答えよ.
%\benu
%\item $a$を実数とする.$x$についての方程式$x-\tan x=a$の実数解のうち,$|x|<\bunsuu{\pi}{2}$をみたすものがちょうど1個あることを示せ.
%\item 自然数$n$に対し,$x-\tan x =n\pi$かつ$|x|<\bunsuu{\pi}{2}$をみたす実数$x$を$x_n$とおく.$t$を$|t|<\bunsuu{\pi}{2}$をみたす実数とする.このとき,曲線$C:y=\sin x$上の点P$(t,\ \sin t)$における接線が,不等式$x \leqq \bunsuu{\pi}{2}$の表す領域に含まれる点においても曲線$C$と接するための必要十分条件は,$t$が$x_1$,$x_2$,$x_3$,$\cdots$のいずれかと等しいことであることを示せ.
%\hfill(2021阪大 理系第5問)
%\eenu
%\equ

%\bqu 以下の問いに答えよ.
%\benu
%\item $A$,$\alpha$を実数とする.$\theta$の方程式
%\[A\sin2\theta-\sin(\theta+\alpha)=0\]
%を考える.$A>1$のとき,この方程式は$0\leqq \theta <2\pi$の範囲に少なくとも4個の解をもつことを示せ.
%\item 座標平面上の楕円
%\[C:\bunsuu{x^2}{2}+y^2=1\]
%を考える.また,$0<r<1$を満たす実数$r$に対して,不等式
%\[2x^2+y^2<r^2\]
%が表す領域を$D$とする.$D$内のすべての点Pが以下の条件を満たすような実数$r$\ $(0<r<1)$が存在することを示せ.また,そのような$r$の最大値を求めよ.\\条件:$C$上の点Qで,Qにおける$C$の接線と直線PQが直交するようなものが少なくとも4個ある.\hfill(2020東大 第6問)
%\eenu
%\equ

%\newpage

%{\bb\Large 高校数学を逸脱するより発展的な話題}

%平均値の定理の一般化として,次の定理がある.

%\begin{itembox}[l]{\bb コーシーの平均値定理}
% $f(x)$,$g(x)$は閉区間$a \leqq x \leqq b$で連続,開区間$a < x < b$で微分可能であるとする.さらに,$a < x < b$のどの点においても,$f'(x)$,$g'(x)$が同時に0になることはないものとする.このとき,$g(a) \neq g(b)$ならば
%\[a<c<b\ \ \ \ かつ\ \ \ \ \bunsuu{f(b)-f(a)}{g(b)-g(a)}=\bunsuu{f'(c)}{g'(c)}\]
%となる$c$が存在する.
%\end{itembox}
%
%通常の平均値の定理は,$g(x)=x$という特別な場合だとみなせる.$f(x)$,$g(x)$のそれぞれに,通常の平均値の定理を適用すれば,
%\[\bunsuu{f(b)-f(a)}{b-a}=f'(c_1),\ \ \ \ \bunsuu{g(b)-g(a)}{b-a}=g'(c_2)\]
%となる,$c_1$,$c_2$が存在し,辺々を割れば,
%\[\bunsuu{f(b)-f(a)}{g(b)-g(a)}=\bunsuu{f'(c_1)}{g'(c_2)}\]
%を得るが,ここにおいて$c_1=c_2$となる保証はない.この$c$の値を共通のものとしてとることができる,というのが,コーシーの平均値の定理の主張である.
%
%証明は,目的の式を$F'(c)=0$という形に整理して作られる$F(x)$に,ロルの定理を適用する.
%
%\begin{証明}
%$F(x)=(g(b)-g(a))(f(x)-f(a))-(f(b)-f(a))(g(x)-g(a))$とおくと,\\
%$F(a)=0$かつ$F(b)=0$より,ロルの定理を適用できて,
%\[a<c<b\ \ \ \ かつ\ \ \ \ F'(c)=0\ \ \ となる\ c\ が存在する.\]
%$F'(c)=0$から,
%\[(g(b)-g(a))f'(c)-(f(b)-f(a))g'(c)=0\ \ \ すなわち\ \ \ \bunsuu{f(b)-f(a)}{g(b)-g(a)}=\bunsuu{f'(c)}{g'(c)}\hfill (証明終了)\]
%\end{証明}
%
%
%\vfill
%
%
%\begin{itembox}[l]{\bb ロピタルの定理}
% $f(x)$,$g(x)$は$x=a$の近くで微分可能であり,かつ,$f(a)=0$,$g(a)=0$とし,\\
% さらに,$x=a$の近くで$g'(a)\neq 0$であるとする.このとき,
%\[極限値\ \dlim_{x \to a} \bunsuu{f'(x)}{g'(x)}\ \ \ が存在するならば\ \ \ \ 
%\dlim_{x \to a} \bunsuu{f(x)}{g(x)}=\dlim_{x \to a} \bunsuu{f'(x)}{g'(x)}\]
%\end{itembox}
%
%
%
%\begin{証明}
%コーシーの平均値の定理により$a$の近くの$x$の値に対し,
%\[\bunsuu{f(x)-f(a)}{g(x)-g(a)}=\bunsuu{f'(c)}{g'(c)}\]
%となる$c$が$x$と$a$の間に存在する.$x \to a$のとき,$c \to a$より,
%\[\dlim_{x \to a} \bunsuu{f(x)}{g(x)}
%=\dlim_{x \to a} \bunsuu{f(x)-f(a)}{g(x)-g(a)}=\dlim_{x \to a}\bunsuu{f'(c)}{g'(c)}=\dlim_{c \to a}\bunsuu{f'(c)}{g'(c)}=\dlim_{x \to a}\bunsuu{f'(x)}{g'(x)}\hfill (証明終了)\]
%\end{証明}
%これは,非常に強力な定理で,広く親しまれています.
%\[{\bb 利用例}\ \ :\ \ \dlim_{x \to 0} \bunsuu{1-\cos x}{x^2} =\dlim_{x \to 0} \bunsuu{(1-\cos x)'}{(x^2)'}=\dlim_{x \to 0} \bunsuu{\sin x}{2x}=\bunsuu 12.\]
%\vfill
%
%
\ifkaisetu

\newpage

\ \\

\vfill

...以上の問題を眺めていると自然に浮かんでくるのは...

\ \\

問1.\ 漸化式$a_{n+1}=f(a_n)$をもつ数列$\{a_n\}$
が$\alpha$に収束するとする.このとき,必ず$f(\alpha)=\alpha$が成り立つといえるか?

\ \\

問2.\ 漸化式$a_{n+1}=f(a_n)$をもつ数列$\{a_n\}$が収束するために,初項$a_1$と$f(x)$についてどのような条件が必要か?


\ \\

\fi

\iffuru
}\mondaitokaitou
\fi                                                                 

\end{document}