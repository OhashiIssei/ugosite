\documentclass[10pt,
a4paper,
%twocolumn,
fleqn,
%landscape, 
%papersize
dvipdfmx,
uplatex
]{jsarticle}

\def\maru#1{\textcircled{\scriptsize#1}}%丸囲み番号

%\renewcommand{\bf}{}
\usepackage{comment}
%\usepackage{amsthm}
\usepackage{amsmath}
%\usepackage{graphics}
\usepackage{emath}
\usepackage{emathMw}
\usepackage{enumerate}
\usepackage{emathEy}
\usepackage{emathP}
\usepackage{emathPk}
\usepackage{emathPh}
\usepackage[g]{esvect}
\usepackage{color}
\usepackage{pxrubrica}%ふりがな
%\usepackage{EMesvect}
%\usepackage[dvipdfmx]{graphicx}%箱ひげ図
%\usepackage{emathSt}%箱ひげ図
%\usepackage{emathG}%箱ひげ図%ヒストグラム
\usepackage{emathPs}
\usepackage{ascmac}%囲み
\usepackage{fancybox}
%\usepackage{fancybx}%囲み
\usepackage[top=10truemm,bottom=10truemm,left=10truemm,right=10truemm]{geometry}
\usepackage{setspace} % 行間
\setstretch{1} % ページ全体の行間を設定
\usepackage{wallpaper}
\usepackage{hako}%センター形式
\usepackage{mathtools}%\abs(絶対値)など
\DeclarePairedDelimiter{\abs}{\lvert}{\rvert}
\usepackage{ifthen}

%ページレイアウト
\setlength{\columnseprule}{0.4pt}
\columnsep=3cm
%%\setlength{\mathindent}{0zw}
\preHEqlabel{$\cdotfill[2em]~$}%houtesikiの点線の長さ
%\linespread{1.3}
%\setstretch{1.2}%行間
\postEqlabel{\hspace{0zw}\null}%式番号の位置
\preEqlabel*{\cdotpfill[2em]~}%式番号の点々の長さ
%\setlength{\columnseprule}{0.4pt}
\columnsep=1cm
%\setlength{\mathindent}{1zw}%数式の位置

\pagestyle{empty}%ページ番号を消す

%定理環境
\usepackage{emathThm}
%\theoremstyle{boxed}
\theorembodyfont{\normalfont}
\newtheorem{Question}{問題}[subsection]
\newtheorem{Q}{}[subsection]
\newtheorem{question}[Question]{}
\newtheorem{quuestion}{}[subsection]
\newcommand{\sub}{%\newpage%サブセクション毎に改ページ
\subsection}
\newcommand{\bqu}{\begin{question}}
\newcommand{\equ}{\end{question}\vfill}
\newcommand{\bQ}{\setcounter{equation}{0}\begin{Q}}
\newcommand{\eQ}{\end{Q}}
\newcommand{\bquu}{\begin{leftbar}}
\newcommand{\equu}{\end{leftbar}}
\newcommand{\cdotss}{\hfill\cdots\cdots}
\newtheorem{tyuu}{}[subsection]
\newcommand{\btyu}{\begin{tyuu}}
\newcommand{\etyu}{\end{tyuu}}
\renewcommand{\thetyuu}{\protect\makebox[2em][c]
{注意}
\ \ }

%箇条書き省略コマンド
\newcommand{\benu}{\begin{enumerate}}
\newcommand{\eenu}{\end{enumerate}}
\newcommand{\beda}{\vspace{-1zw}\begin{edaenumerate}}
\newcommand{\eeda}{\end{edaenumerate}}
\newcommand{\bb}{\bf\boldmath}%全部太字にする
\newcommand{\doo}{^{\circ}}%角度マーク
\newcommand{\sq}{\textstyle\sqrt}
\newcommand{\ANA}{\hakosenhaba{1pt}\Hako}
\newcommand{\REFANA}{\hakosenhaba{0.3pt}\refHako*}
\newcommand{\C}{\text{C}}
\newcommand{\dsum}{\displaystyle\sum}

\usepackage{tabularx}
%\newcolumntype{Y}{&gt;{\centering\arraybackslash}X} %中央揃え
%\includegraphics[width=90mm,bb=9 9 358 434]{./lrep_e1.eps}

%セクション,大問番号のデザイン
\renewcommand{\labelenumi}{(\arabic{enumi})}
%\renewcommand{\labelenumi}{\ \fbox{\protect\makebox[1em][c]{\large{\bfseries\arabic{enumi}}}}\ }
%\renewcommand{\labelenumi}{\textbf{\theenumi}}
%\renewcommand{\theenumiii}{(\alph{enumiii})}
%\renewcommand{\theenumii}{\arabic{enumii}}
%\renewcommand{\thesection}{\Huge  第\arabic{section}章}
\renewcommand{\thesubsection}{
%\Large\bb\Alph{subsection}
\ }
\renewcommand{\theQ}{{\underline{$\overline{\bb \ 追加問題\ }$}}
\ \ }
\renewcommand{\theQuestion}{
\arabic{Question}.}
\renewcommand{\thequuestion}{{\underline{$\overline{\bb \ 性質\ }$}}
\ \ }

%横に縦線
\usepackage{framed}
\newcommand{\bche}{\begin{leftbar}
%\underbar{ホワイトボード}\\
}
\newcommand{\eche}{\end{leftbar}}
\newcommand{\bcha}{\begin{leftbar}}
\newcommand{\echa}{\end{leftbar}}
\newcommand{\bapp}{\hspace{-2zw}\underbar{\bf Approach}\vspace{-1zw}\begin{leftbar}}
\newcommand{\eapp}{\end{leftbar}}
\newcommand{\bmem}{\hspace{-2zw}\underbar{\bf Memo}\vspace{-1zw}\begin{leftbar}}
\newcommand{\emem}{\end{leftbar}}
\newcommand{\barr}{\left\{\begin{array}{l}}
\newcommand{\earr}{\end{array}\right.}
\renewcommand{\bar}{\overline}
\renewcommand{\Re}{\text{Re}}
\renewcommand{\Im}{\text{Im}}
\renewcommand{\dlim}{\displaystyle\lim}

\newenvironment{証明}{\hspace{-2zw}\underbar{\bf 証明}\vspace{-1zw}\begin{leftbar}}{\end{leftbar}}
\newenvironment{解答}{\hspace{-2zw}\underbar{\bf 解答}\vspace{-1zw}\begin{leftbar}}{\end{leftbar}}


\makeatletter
\renewenvironment{leftbar}{%
%  \def\FrameCommand{\vrule width 3pt \hspace{10pt}}%  デフォルトの線の太さは3pt
\def\FrameCommand{\vrule width 1pt \hspace{1zw}}%
\MakeFramed {\advance\hsize-\width \FrameRestore}}%
{\endMakeFramed}
\makeatother


%カラー,色
\usepackage{color}
\definecolor{link}{rgb}{0.63671875,0.99609375,0.99609375}
\definecolor{usumido}{rgb}{0.953125,0.95703125,0.9375}
%\pagecolor{usumido}

%はやくち解説ボタン
\usepackage[dvipdfmx]{hyperref}%ハイパーリンク
\usepackage{pxjahyper}
\newcommand{\botanA}{\LARGE\color{usumido}l\includegraphics[width=1.7cm,height=0.4cm,bb=0 0 1489 436]{はやくち解説ボタン3.jpeg}l}
%\hypersetup{% hyperrefオプションリスト
%setpagesize=false,
% bookmarksnumbered=true,%
% bookmarksopen=true,%
% colorlinks=true,%
% linkcolor=blue,
% citecolor=red,
%}

\newcommand{\barabara}{%一問ずつのページを別で作成
\myfor{1} % ループ実行       
\newpage   
\setcounter{subsection}{0}
\setcounter{Question}{0}
\renewcommand{\bqu}{\begin{question}}
\renewcommand{\equ}{\end{question}\newpage}
\renewcommand{\eQ}{\end{Q}\newpage}
\renewcommand{\equu}{\end{quuestion}\newpage}
\renewcommand{\equu}{\end{quuestion}\newpage}
%\myfor{1} % ループ実行   
\newpage
\setcounter{subsection}{0}
\setcounter{Question}{0}
\TileWallPaper{110mm}{160mm}{方眼紙.pdf} %方眼紙
\myfor{1} % ループ実行   
}
\usepackage{qrcode}%QRコード
\setlength\normallineskiplimit{0pt}

\begin{document}

%\newcount\K % int K
%\def\myfor#1{%
%\K=0 \loop\ifnum\K<1 % for(K=0;K<2;K++)

%\sub

{\bb\Large 連続性と中間値の定理}\\

\benu
\item %{\bb 各点における連続性}\\
 関数$f(x)$が{\bb $x=a$において連続}であるとは,
 \[\phantom{極限値\ \dlim_{x \to a} f(x)\ が存在し,かつ\ \ \ \dlim_{x \to a} f(x)=f(a)}\]
  \[\phantom{極限値\ \dlim_{x \to a} f(x)\ が存在し,かつ\ \ \ \dlim_{x \to a} f(x)=f(a)}\]
であることをいう.
関数$f(x)$が区間$I$内の全ての$a$において連続であるとき,\\
$f(x)$は{\bb 区間$I$で連続}であるという.\\
  \vfill
 
\item %{\bb 区間における連続,連続関数}\\
 定義域全体で連続である関数を{\bb 連続関数}という.\\
連続関数$f(x)$,$g(x)$に対して,
\[和f(x)+g(x),\ \ 差f(x)-g(x),\ \ 積f(x)g(x),\ \ 商\bunsuu{f(x)}{g(x)},\ \ 絶対値|f(x)|,\ \ 逆関数f^{-1}(x),\ \ 合成関数(g\circ f)(x)\]
はすべて連続関数である.
  \vfill
  
\eenu
 
%\begin{itembox}[l]{\bb 連続関数の多さ}

%\end{itembox}

%\newpage

\begin{itembox}[l]{\bb 最大値・最小値の存在定理}
関数$f(x)$が,\underline{\hspace{5zw}\phantom{閉区間$a \leqq x \leqq b$で連続}}であるとき,
\[f(x)\ は,この区間内で最大値と最小値をもつ.\]
\vfill
\vspace{7zw}
\end{itembox}

\begin{itembox}[l]{\bb 解の存在定理}
関数$f(x)$が,\underline{\hspace{5zw}\phantom{閉区間$a \leqq x \leqq b$で連続,かつ,$f(a) \times f(b) <0$}}であるとき,
\[a< c< b\ \ \ かつ\ \ \ f(c)=0\ \ \ \ となる c が存在する\]
\vfill
\vspace{7zw}
\end{itembox}


\begin{itembox}[l]{\bb 中間値の定理}
関数$f(x)$が,\underline{\hspace{5zw}\phantom{閉区間$a \leqq x \leqq b$で連続,かつ\ $f(a)\neq f(b)$}}であるとき,
\[f(a)\ と\ f(b)\ の間の任意の値\ m\ に対し,\]
\[a< c< b\ \ \ かつ\ \ \ f(c)=m\ \ となる c が存在する\]
\vfill
\vspace{6zw}
\end{itembox}

\newpage

{\bb\Large 微分可能性と平均値の定理}\\

\benu
\item[(3)] %{\bb 平均変化率}\\
関数$f(x)$と$x$の異なる値$x_1$,$x_2$に対し,
\[\phantom{\bunsuu{f(x_2)-f(x_1)}{x_2-x_1}}\]
\[\phantom{\bunsuu{f(x_2)-f(x_1)}{x_2-x_1}}\]
という値を,関数$f(x)$の{\bb $x_1$と$x_2$の間の平均変化率}という.
\vfill

\item[(4)] %{\bb 微分係数,導関数}\\
関数$f(x)$と定数$a$に対し,
\[\phantom{\dlim_{x \to h} \bunsuu{f(a+h)-f(a)}{h}}\]
\[\phantom{\dlim_{x \to h} \bunsuu{f(a+h)-f(a)}{h}}\]
を,関数$f(x)$の{\bb $x=a$における微分係数}といい,記号{\bb $f'(a)$}で表す.\\
$a$に対して微分係数$f'(a)$を対応させる関数を,$f(x)$の{\bb 導関数}といい,{\bb $f'(x)$}で表す.\\
与えられた関数$f(x)$に対して$f'(x)$を求めることを{\bb 微分する}という.
\vfill

\item[(5)] %{\bb 微分可能性}\\
関数$f(x)$が,{\bb $x=a$において微分可能}であるとは,
\[\phantom{極限値\ f'(a)=\dlim_{x \to h} \bunsuu{f(a+h)-f(a)}{h}\ が存在する}\]
\[\phantom{極限値\ f'(a)=\dlim_{x \to h} \bunsuu{f(a+h)-f(a)}{h}\ が存在する}\]
ことをいう.
関数$f(x)$が区間$I$内の全ての$a$において微分可能であるとき,\\
$f(x)$は{\bb 区間$I$で微分可能}であるという.
\vfill

\eenu

\begin{itembox}[l]{\bb ロルの定理}
関数$f(x)$が,閉区間$a \leqq x \leqq b$で連続,かつ,開区間$a < x < b$で微分可能,かつ,
\[f(a)=f(b)\ を満たすとき,\]
\[a<c<b\ \ \ \ かつ \hspace{5zw}\phantom{\ \ \ \ f'(c)=0 \ \ \ }となる\ c\ が存在する.\]
\vfill
\vspace{6zw}
\end{itembox}


\begin{itembox}[l]{\bb 平均値の定理}
関数$f(x)$が,閉区間$a \leqq x \leqq b$で連続,かつ,開区間$a < x < b$で微分可能であるとき,
\[a<c<b\ \ \ \ かつ\hspace{5zw}\phantom{\ \ \ \ \bunsuu{f(b)-f(a)}{b-a}=f'(c)\ \ \ }となる\ c\ が存在する.\]
\vfill
\vspace{7zw}
\end{itembox}

\newpage

{\bb\Large 高校数学を逸脱するより発展的な話題}

平均値の定理の一般化として,次の定理がある.

\begin{itembox}[l]{\bb コーシーの平均値定理}
 $f(x)$,$g(x)$は閉区間$a \leqq x \leqq b$で連続,開区間$a < x < b$で微分可能であるとする.さらに,$a < x < b$のどの点においても,$f'(x)$,$g'(x)$が同時に0になることはないものとする.このとき,$g(a) \neq g(b)$ならば
\[a<c<b\ \ \ \ かつ\ \ \ \ \bunsuu{f(b)-f(a)}{g(b)-g(a)}=\bunsuu{f'(c)}{g'(c)}\]
となる$c$が存在する.
\end{itembox}

通常の平均値の定理は,$g(x)=x$という特別な場合だとみなせる.$f(x)$,$g(x)$のそれぞれに,通常の平均値の定理を適用すれば,
\[\bunsuu{f(b)-f(a)}{b-a}=f'(c_1),\ \ \ \ \bunsuu{g(b)-g(a)}{b-a}=g'(c_2)\]
となる,$c_1$,$c_2$が存在し,辺々を割れば,
\[\bunsuu{f(b)-f(a)}{g(b)-g(a)}=\bunsuu{f'(c_1)}{g'(c_2)}\]
を得るが,ここにおいて$c_1=c_2$となる保証はない.この$c$の値を共通のものとしてとることができる,というのが,コーシーの平均値の定理の主張である.

証明は,目的の式を$F'(c)=0$という形に整理して作られる$F(x)$に,ロルの定理を適用する.

\begin{証明}
$F(x)=(g(b)-g(a))(f(x)-f(a))-(f(b)-f(a))(g(x)-g(a))$とおくと,\\
$F(a)=0$かつ$F(b)=0$より,ロルの定理が適用できて,
\[a<c<b\ \ \ \ かつ\ \ \ \ F'(c)=0\ \ \ となる\ c\ が存在する.\]
$F'(c)=0$から,
\[(g(b)-g(a))f'(c)-(f(b)-f(a))g'(c)=0\ \ \ すなわち\ \ \ \bunsuu{f(b)-f(a)}{g(b)-g(a)}=\bunsuu{f'(c)}{g'(c)}\hfill (証明終了)\]
\end{証明}


\vfill


\begin{itembox}[l]{\bb ロピタルの定理}
 $f(x)$,$g(x)$は$x=a$の近くで微分可能であり,かつ,$f(a)=0$,$g(a)=0$とし,\\
 さらに,$x=a$の近くで$g'(a)\neq 0$であるとする.このとき,
\[極限値\ \dlim_{x \to a} \bunsuu{f'(x)}{g'(x)}\ \ \ が存在するならば\ \ \ \ 
\dlim_{x \to a} \bunsuu{f(x)}{g(x)}=\dlim_{x \to a} \bunsuu{f'(x)}{g'(x)}\]
\end{itembox}



\begin{証明}
コーシーの平均値の定理により$a$の近くの$x$の値に対し,
\[\bunsuu{f(x)-f(a)}{g(x)-g(a)}=\bunsuu{f'(c)}{g'(c)}\]
となる$c$が$x$と$a$の間に存在する.$x \to a$のとき,$c \to a$より,
\[\dlim_{x \to a} \bunsuu{f(x)}{g(x)}
=\dlim_{x \to a} \bunsuu{f(x)-f(a)}{g(x)-g(a)}=\dlim_{x \to a}\bunsuu{f'(c)}{g'(c)}=\dlim_{c \to a}\bunsuu{f'(c)}{g'(c)}=\dlim_{x \to a}\bunsuu{f'(x)}{g'(x)}\hfill (証明終了)\]
\end{証明}
これは,非常に強力な定理で,広く親しまれています.
\[{\bb 利用例}\ \ :\ \ \dlim_{x \to 0} \bunsuu{1-\cos x}{x^2} =\dlim_{x \to 0} \bunsuu{(1-\cos x)'}{(x^2)'}=\dlim_{x \to 0} \bunsuu{\sin x}{2x}=\bunsuu 12.\]
\vfill

%\advance\K by 1\repeat }
%
%\barabara
%                     

\end{document}