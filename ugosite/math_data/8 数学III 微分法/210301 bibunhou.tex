\newif\ifkaitou
\kaitoutrue%解答あり
%\kaitoufalse%解答なし

\documentclass[9pt,
b4paper,
%twocolumn,
fleqn,
%landscape, 
%papersize
dvipdfmx,
uplatex
]{jsarticle}

\def\maru#1{\textcircled{\scriptsize#1}}%丸囲み番号

%\renewcommand{\bf}{}
%\usepackage{アプローチ}
%\usepackage{amsthm}
\usepackage{amsmath}
\usepackage{ascmac}
%\usepackage{graphics}
\usepackage{emath}
\usepackage{emathMw}
\usepackage{enumerate}
\usepackage{emathC}
\usepackage{emathEy}
\usepackage{emathP}
\usepackage{emathPp}
\usepackage{emathPl}
\usepackage{emathPk}
\usepackage{emathPh}
\usepackage{emathPs}
\usepackage[g]{esvect}
\usepackage{color}
\usepackage{pxrubrica}%ふりがな
%\usepackage{EMesvect}
%\usepackage[dvipdfmx]{graphicx}%箱ひげ図
%\usepackage{emathSt}%箱ひげ図
%\usepackage{emathG}%箱ひげ図%ヒストグラム
\usepackage{emathPs}
\usepackage{ascmac}%囲み
\usepackage{fancybox}
%\usepackage{fancybx}%囲み
\usepackage[top=12truemm,bottom=12truemm,left=12truemm,right=10truemm]{geometry}
\usepackage{setspace} % 行間
\setstretch{1} % ページ全体の行間を設定
\usepackage{wallpaper}
\usepackage{hako}%センター形式
\usepackage{mathtools}%\abs(絶対値)など
\DeclarePairedDelimiter{\abs}{\lvert}{\rvert}
\usepackage{fancyhdr}%ヘッダの設定
\usepackage{cancel}%消し取り線
\usepackage{ifthen}
%\usepackage{exam}
\usepackage{tcolorbox}
\usepackage{extarrows}%伸縮性のある矢印

%ページレイアウト
\setlength{\columnseprule}{0.4pt}
\columnsep=3cm
%\setlength{\mathindent}{0zw}
\preHEqlabel{$\cdotfill[2em]~$}%houtesikiの点線の長さ
%\linespread{1.3}
%\setstretch{1.2}%行間
\postEqlabel{\hspace{0zw}\null}%式番号の位置
\preEqlabel*{\cdotpfill[2em]~}%式番号の点々の長さ
%\setlength{\columnseprule}{0.4pt}
\columnsep=1cm
%\setlength{\mathindent}{1zw}%数式の位置

\pagestyle{empty}%ページ番号を消す

%定理環境
\usepackage{emathThm}
%\theoremstyle{boxed}
\theorembodyfont{\normalfont}
\newtheorem{Question}{問題}[subsection]
\newtheorem{Q}{}[subsection]
\newtheorem{question}[Question]{}
\newtheorem{quuestion}{}[subsection]
\newcommand{\sub}{%\newpage%サブセクション毎に改ページ
\subsection}
\newcommand{\bqu}{\begin{question}}
\newcommand{\equ}{\end{question}}
\newcommand{\cdotss}{\hfill\cdots\cdots}

%箇条書き省略コマンド
\newcommand{\benu}{\begin{enumerate}}
\newcommand{\eenu}{\end{enumerate}}
\newcommand{\beda}{\vspace{-1zw}\begin{edaenumerate}}
\newcommand{\eeda}{\end{edaenumerate}}

%省略
\newcommand{\bb}{\bf\boldmath}%全部太字にする
\newcommand{\doo}{^{\circ}}%角度マーク
\newcommand{\sq}{\textstyle\sqrt}
\newcommand{\ANA}{\hakosenhaba{1pt}\Hako}
\newcommand{\REFANA}{\hakosenhaba{0.3pt}\refHako*}
\newcommand{\C}{\text{C}}
\newcommand{\dsum}{\displaystyle\sum}
\newcommand{\barr}{\left\{\begin{array}{l}}
\newcommand{\earr}{\end{array}\right.}
\renewcommand{\bar}{\overline}
\renewcommand{\Re}{\text{Re}}
\renewcommand{\Im}{\text{Im}}
\renewcommand{\dlim}{\displaystyle\lim}

\usepackage{tabularx}
%\newcolumntype{Y}{&gt;{\centering\arraybackslash}X} %中央揃え
%\includegraphics[width=90mm,bb=9 9 358 434]{./lrep_e1.eps}

%セクション,大問番号のデザイン
\renewcommand{\labelenumi}{(\arabic{enumi})}
%\renewcommand{\labelenumi}{\ \fbox{\protect\makebox[1em][c]{\large{\bfseries\arabic{enumi}}}}\ }
%\renewcommand{\labelenumi}{\textbf{\theenumi}}
%\renewcommand{\theenumiii}{(\alph{enumiii})}
%\renewcommand{\theenumii}{\arabic{enumii}}
%\renewcommand{\thesection}{\Huge  第\arabic{section}章}
\renewcommand{\thesubsection}{
\Large\bb\Alph{subsection}
\ }
\renewcommand{\theQ}{{\underline{$\overline{\bb \ 追加問題\ }$}}
\ \ }
\renewcommand{\theQuestion}{
\large\arabic{Question}.}
\renewcommand{\thequuestion}{{\underline{$\overline{\bb \ 性質\ }$}}
\ \ }

\newenvironment{解答}{
\hspace{-2zw}\phkasen<linethickness=7pt,iro=mygray,kasenUehosei=-3pt>{\bf \large \ 解答\ }\vspace{-1zw}\begin{leftbbar}}{\end{leftbbar}}

%横に縦線
\usepackage{framed}

\makeatletter
\renewenvironment{leftbar}{%
\def\FrameCommand{\vrule width 1pt \hspace{-3zw}}%
\MakeFramed {\advance\hsize-\width \FrameRestore}}%
{\endMakeFramed}
\makeatother

\newenvironment{leftbbar}{%
\def\FrameCommand{\color{mygray} \vrule width 5pt \hspace{1zw}
\color{black}}%
\MakeFramed {\advance\hsize-\width \FrameRestore}}%
{\endMakeFramed}
\makeatother


%カラー,色
\usepackage{color}
\definecolor{mygray}{gray}{0.75}

\newcommand{\barabara}{%一問ずつのページを別で作成
\myfor{1} % ループ実行       
\newpage   
\setcounter{subsection}{0}
\setcounter{Question}{0}
\renewcommand{\bqu}{\begin{question}}
\renewcommand{\equ}{\end{question}\newpage}
\renewcommand{\eQ}{\end{Q}\newpage}
\renewcommand{\equu}{\end{quuestion}\newpage}
\renewcommand{\equu}{\end{quuestion}\newpage}
%\myfor{1} % ループ実行  
}
\usepackage{qrcode}%QRコード
\setlength\normallineskiplimit{0pt} 

\begin{document}

%\newcount\K % int K
%\def\myfor#1{%
%\K=0 \loop\ifnum\K<1 % for(K=0;K<2;K++)


\twocolumn[
     \textbf{
%\LARGE \fbox{解答用紙}  
         \large  【自然科学コース】2年学年末考査 数学探求Ⅰ①  %\textcircled{\footnotesize{1}}
       2021.3.4(木)}
      \hfill
      \ \ \ 2年  \underline{\hspace{3zw}}組\underline{\hspace{3zw}}番  \hspace{1zw}氏名\underline{\hspace{10zw}}\\
\vspace{1zw}
]

%\bqu%{\bb 共通接線}
%\equ

%1辺の四面体ABCDにおいて,等式$|\vv{\text{AB}}|^2+|\vv{\text{CD}}|^2=|\vv{\text{AC}}|^2+|\vv{\text{BD}}|^2$のとき,$\vv{\text{AD}} \perp \vv{\text{BC}}$であることを証明せよ.

\bqu%{\bb 凹凸グラフ,漸近線}\\
関数$f(x)=\bunsuu{x^2+2x+2}{x+1}$の増減,極値,グラフの凹凸,漸近線を調べ,グラフの概形をかけ.\hfill (14点)
\equ

\ifkaitou
\begin{解答}
$f(x)=\bunsuu{(x+1)^2+1}{x+1}=x+1+\bunsuu{1}{x+1}$より,\\
$\dlim_{x \to \pm -1\pm 0} f(x)=\pm \infty$,$\dlim_{x \to \pm \infty}\{f(x)-(x+1)\}=0$\\
$\therefore$\ \ {\bb 直線$x=-1$,$y=x+1$が漸近線}.
\begin{align*}
f'(x)&=1-\bunsuu{1}{(x+1)^2}=\bunsuu{x(x+2)}{(x+1)^2}.\\
f''(x)&=\bunsuu{2}{(x+1)^3}
\end{align*}
ゆえに,$f(x)$の増減,凹凸,グラフは次のようになる.
\[\begin{array}{c|ccccccccc}
\phantom{\bunsuu 11} x	&\cdots	&-2	 	&\cdots	&-1 		&\cdots	 	&0	&\cdots	 \\\hline
\phantom{\bunsuu 11}f'(x)	&+		&0		&-		&/		&-			&0	&+		\\\hline
\phantom{\bunsuu 11}f''(x)	&-		&		&-		&/		&+			&	&+		\\\hline
\phantom{\bunsuu 11}f(x)	&\NEE	&-2		&\SES	&/		&\SEE		&2	&\NEN	
\end{array}
\begin{zahyou}[ul=5mm,haiti=x](-5,3)(-4,4)
\def\Fx{(X*X+2*X+2)/(X+1)}
\def\Gx{X+1}
\YGraph<linethickness=1pt>\Fx
\YGraph<linethickness=0.2pt>\Gx
\XGraph<linethickness=0.2pt>{-1}
\YPointPut\Fx{0}[syaei=y,ypos={[ne]}]{}
\YPointPut\Fx{-2}[syaei=xy,xpos={[nw]}]{}
\YPointPut\Gx{0}[syaei=y]{}
\YPointPut\Gx{-1}[syaei=x,xpos={[nw]}]{}
\end{zahyou}
\]
\\
よって,{\bb $x=-2$で極大値$-2$,$x=0$で極小値2}をとる.
\end{解答}
\fi

\vfill

\bqu%{\bb 最大値・最小値}\\
$-\pi \leqq x \leqq \pi$における$y=2\sin x+\sin2x$の最大値と最小値を求めよ.\hfill (14点)
\equ

\ifkaitou
\begin{解答}
奇関数なので,$0\leqq x \leqq \pi$を調べる.
\begin{align*}
y'&=2\cos x+2\cos2x\\
&=2(\cos x+\cos2x)\\
&=2(\cos x+2\cos^2x-1)\\
&=2(2\cos x-1)(\cos x+1)
%&=4\cos \bunsuu 32 x\cos\bunsuu{x}2
\end{align*}
よって,$y$の増減は次のようになる.
\[\begin{array}{c|ccccccc}
\phantom{\bunsuu 11} x	&0	 	&\cdots	&\frac {\pi}3 		&\cdots	 	&\pi \\\hline
\phantom{\bunsuu 11}f'(x)	&		&+		&0		&-			&\\\hline
\phantom{\bunsuu 11}f(x)	&0		&\NE	&		&\SE		&0\\
\end{array}
\]
よって,求める最大値は$f(\frac{\pi}3)=\text{\bb $\frac 32 \sq 3$}$,最小値は$f(-\frac{\pi}3)=\text{\bb $\frac {-3}2 \sq 3$}$.
\end{解答}
\fi

\vfill

\bqu%{\bb 不等式への応用}\\
すべての正の数$x$に対して,$e^x > 1+x+\bunsuu{x^2}2$が成立することを示せ.\hfill (14点)
\equ

\ifkaitou
\begin{mawarikomi}{}{
\begin{zahyou}[ul=5mm](-2,2)(-1,5)
\def\Fx{exp(X)}
\def\Gx{1+X+X*X/2}
\YGraph<linethickness=1pt>\Fx
\YGraph<linethickness=1pt>\Gx
%\YPointPut\Fx{2.7}[syaei=xy,xlabel=e,ylabel=e]{}
\end{zahyou}
}
\begin{解答}
$f(x)=e^x -1-x-\bunsuu{x^2}2$とおくと,
\begin{align*}
f'(x)&=e^x - 1-x\\
f''(x)&=e^x - 1
\end{align*}
$x>0$のとき,$f''(x)>0$より,$f'(x)$は単調増加する.\\
よって,$f'(x)>f'(0)=0$,$f(x)$は単調増加する.\\
よって,$f(x)>f(0)=0$.故に,$e^x > 1+x+\bunsuu{x^2}2$\hfill □
\end{解答}
\end{mawarikomi}
\fi

\vfill

\newpage

\bqu%{\bb 解の個数}\\
%方程式$x^3-ax^2+a=0$の解の個数を求めよ.
方程式$ax^5-x^2+3=0$が3個の異なる実数解をもつような$a$の値の範囲を求めよ.\hfill (14点)
\equ

\ifkaitou
\begin{解答}
$ax^5-x^2+3=0$において,$x=0$とすると,$3=0$となり成立しない.よって$x \neq 0$とし,
\[a=\bunsuu{x^2-3}{x^5}:=f(x)\ \ とおくと,\]
\[f'(x)=\bunsuu{2x\cdot x^5-(x^2-3)\cdot 5x^4}{(x^5)^2}%\\
%&=\bunsuu{2x^2-(x^2-1)\cdot 5}{x^6}\\
%&
=\bunsuu{-3(x^2-5)}{x^6}
\]
よって,$f(x)$の増減は次のようになる.
\[\begin{array}{c|ccccccc}
\phantom{\bunsuu 11} x	&\cdots	&-\sqrt5	 &\cdots	&0 		&\cdots	 &\sqrt5	&\cdots \\\hline
\phantom{\bunsuu 11}f'(x)	&-		&0				&+		&/		&+		&0				&- \\\hline
\phantom{\bunsuu 11}f(x)	&\SE	&\frac {-2}{25\sqrt{5}}				&\NE	&/		&\NE	&\frac 2{25\sqrt{5}}				&\SE 
\end{array}
\]
また,$\dlim_{x \to \pm \infty} f(x) = 0$,$\dlim_{x \to \pm 0} f(x) = \mp \infty$より,
\[\begin{zahyou}[ul=15mm,yscale=10,xscale=0.5](-6,6)(-0.1,0.1)
\def\Fx{(X*X-3)/(X*X*X*X*X)}
\YGraph<linethickness=1pt>\Fx
%\YPointPut\Fx{1}[syaei=x,xpos={[nw]}]{}
%\YPointPut\Fx{-1}[syaei=x,xpos={[se]}]{}
\YPointPut\Fx{sqrt(5)}[syaei=xy,xlabel=\sqrt{5},ylabel=\frac 2{25\sqrt{5}}]{}
\YPointPut\Fx{-sqrt(5)}[syaei=xy,xlabel=-\sqrt{5},ylabel=\frac {-2}{25\sqrt{5}}]{}
\end{zahyou}\]
よって,求める$a$の値の範囲は
$\text{\bb $\frac {-2}{25\sqrt{5}}<a<0,0<a<\frac 2{25\sqrt{5}}$}$
\end{解答}
\fi

\vfill

%\bqu%{\bb 極値,変曲点をもつ条件}
%\benu
%\item 関数$f(x)=ax+\cos x$が極値をもつような$a$の値の範囲を求めよ.
%\item 関数$g(x)=bx^2+\sin x$のグラフが変曲点をもつような$b$の値の範囲を求めよ.
%\eenu
%\equ

\bqu%{\bb 極値,変曲点をもつ条件}\\
%関数$f(x)=\bunsuu 1x-e^{-ax}$が$x>0$において2つの極値をもつとき,定数$a$のとり得る値の範囲を求めよ.ただし,$\dlim_{x \to \infty} \bunsuu{x^2}{e^x}=0$である.%\hfill(東京電機大)
$f(x)=(x^2+a)e^{x}$とする.ただし,$a$は定数とする.
\benu
\item 関数$f(x)$が極値をもたないような$a$の値の範囲を求めよ.\hfill (7点)
\item 曲線$y=f(x)$が変曲点をもつような$a$の値の範囲を求めよ.\\
\hfill (7点)
\eenu
\equ

\ifkaitou
\begin{解答}
\benu
\item $f'(x)=2x\cdot e^x+(x^2+a)e^x=(x^2+2x+a)e^x$.\\
より,$f(x)$が極値をもたないための条件は,
\[f'(x)\ の符号が一定であること\cdots(*)\]
である.$x^2+2x+a=0$の判別式を$D$とすると,

\begin{mawarikomi}{}{
\begin{zahyou*}[ul=5mm](-2, 2)(-1, 3)
\ArrowLine{(\xmin,0)}{(\xmax,0)}
\YGurafu<linethickness=1pt>{X*X+1}{\xmin}{\xmax}
\YGurafu<linethickness=1pt>{X*X}{\xmin}{\xmax}
\Put{(\xmax,0)}[s]{$x$}%
\end{zahyou*}
}
$(*)$となるための条件は,
\begin{align*}
D&=2^2-4\cdot a \leqq 0%\\
%即ち\ &4-4a<0
\end{align*}
よって,求める$a$の値の範囲は{\bb $a\geqq 1$}.
\end{mawarikomi}
\item 
$f''(x)=(2x+2)e^x+(x^2+2x+a)e^x=(x^2+4x+2+a)e^x$.\\
より,$f(x)$のグラフが変曲点をもつための条件は
\[f''(x)\ に符号変化が起こること\cdots(**)\]
である.$x^2+4x+2+a=0$の判別式を$D$とすると,

\begin{mawarikomi}{}{
\begin{zahyou*}[ul=5mm](-2, 2)(-1, 1)
\ArrowLine{(\xmin,0)}{(\xmax,0)}
\YGurafu<linethickness=1pt>{X*X-1}{\xmin}{\xmax}
\Put{(\xmax,0)}[s]{$x$}%
\end{zahyou*}}
$(**)$となる条件は,
\begin{align*}
D&=4^2-4(2+a)>0%\\
%即ち\ &8-4a>0
\end{align*}
よって,求める$a$の値の範囲は{\bb $a<2$}.
\end{mawarikomi}
%$g'(x)=2bx+\cos x$より,$g''(x)=2b-\sin x$の値域は$2b-1 \leqq g''(x) \leqq 2b+1$なので,
\eenu
\end{解答}
\fi

\vfill

%\hfill (70点)

%\begin{解答}
%\benu
%\item $f(x)$は実数全体で微分可能なので,$f(x)$が極値をもつための条件は,
%$f'(x)$に符号変化が起こることである.\\
%$f'(x)=a-\sin x$の値域は$a-1 \leqq f'(x) \leqq a+1$なので,求める条件は$a-1<0<a+1$\ \ すなわち\ \ {\bb $-1<a<1$}.
%\item $g(x)$は実数全体で2階微分可能なので,$g(x)$のグラフが変曲点をもつための条件は$g''(x)$に符号変化が起こることである.\\
%$g'(x)=2bx+\cos x$より,$g''(x)=2b-\sin x$の値域は$2b-1 \leqq g''(x) \leqq 2b+1$なので,求める条件は$2b-1<0<2b+1$\ \ すなわち\ \ {\bb $-\bunsuu 12<b<\bunsuu 12$}.
%\eenu
%\end{解答}

%\newpage

%\bqu%{\bb 凹凸グラフ・接線の本数}
%\benu
%\item 関数$y=x\log x$の増減,グラフの凹凸,漸近線などを調べ,グラフの概形をかけ.
%%\item 関数$y=x\log x$のグラフの接線で,傾きが$m$であるものが2本存在するような,$m$の値の範囲を求めよ.
%\item 点A$(0,b)$からグラフ$y=x\log x$に引くことのできる接線の本数がちょうど3本となるような$b$の値の範囲を求めよ.
%\eenu
%\equ
%
%\begin{解答}
%\benu
%\item $y=\bunsuu{x}{\log x}$より,
%\begin{align*}
%y'&=\bunsuu{\log x-x\cdot \frac 1x}{(\log x)^2} =\bunsuu{\log x-1}{(\log x)^2}=\bunsuu 1{\log x}-\bunsuu{1}{(\log x)^2}\\
%y''&=-\bunsuu{1}{x (\log x)^2}+\bunsuu{2}{x(\log x)^3}=\bunsuu{-\log x+2}{x (\log x)^3}
%\end{align*}
%\begin{zahyou}[ul=5mm](-1,15)(-2,5)
%\def\Fx{X/(log(X))}
%\teisuuretu{tval=2.7*2.7}
%\YGraph<linethickness=1pt,minx=1.001>\Fx
%\YGraph<linethickness=1pt,minx=0.001,maxx=0.999>\Fx
%\XGraph<linethickness=0.2pt>{1}
%\YGraph{(log(\tval)-1)/(log(\tval)*log(\tval))*(X-\tval)+(\tval)/(log(\tval))}
%\YPointPut\Fx{2.7}[syaei=xy,xlabel=e,ylabel=e]{}
%\YPointPut\Fx{2.7*2.7}[syaei=xy,xlabel=e^2,ylabel=\frac{e^2}{2}]{}
%\end{zahyou}
%\item 接点を$\left(t,\ \bunsuu{t}{\log t}\right)$とおくと,接線の方程式は
%\[y=\bunsuu{\log t-1}{(\log t)^2}\cdot (x-t)+\bunsuu{t}{\log t}\]
%である.これが点$(0,\ b)$を通る条件は,
%\[b=\bunsuu{\log t-1}{(\log t)^2}\cdot (0-t)+\bunsuu{t}{\log t}=\bunsuu{t}{(\log t)^2}\]
%
%\begin{zahyou}[ul=5mm](-1,15)(-2,5)
%\def\Fx{X/((log(X))*(log(X)))}
%\teisuuretu{tval=2.7*2.7}
%\YGraph<linethickness=1pt,minx=1.001>\Fx
%\YGraph<linethickness=1pt,minx=0.001,maxx=0.999>\Fx
%\XGraph<linethickness=0.2pt>{1}
%%\YPointPut\Fx{2.7}[syaei=xy,xlabel=e,ylabel=e]{}
%\YPointPut\Fx{2.7*2.7}[syaei=xy,xlabel=e^2,ylabel=\frac{e^2}{2}]{}
%\end{zahyou}
%\eenu
%\end{解答}

%\bqu%{\bb 方程式への応用}\\
%
%\equ

%\bqu%{\bb 大小}\\
%$2020^{\frac{1}{2020}}$と$2021^{\frac{1}{2021}}$の大小を調べよ.
%\equ

%\newpage



%\bqu%{\bb 体積}\\
%半径1の球Oに内接する四角柱で,底面ABCDが正方形であるものを考える.このような四角柱の体積$V$に対する表面積$S$の
%%$\theta=$とするとき,
%%球の中心Oと$K$の底面の頂点を結んだ線分が,底面となす角を$\theta$とする.
%比$\bunsuu{S}{V}$の最小値と,そのときの$\tan\angle\text{OAC}$の値を求めよ.
%%$S$,$V$をそれぞれ$\theta$を用いて表せ.
%\equ
%
%
%\begin{解答}
%$\angle\text{OAC}=\theta$とおくと,
%\[\text{AC}=2\cos\theta,\ \ \text{AB}=\sq 2\cos \theta,\ \ (四角柱の高さ)=\sin\theta\ \ より\]
%\[
%\begin{zahyou*}[ul=15mm](-1.2,1.2)(-1.2,1.2)
%\teisuuretu{rval=1}
%\teisuuretu{aval=3.14/6}
%\tenretu{O(0,0)es}
%\tenretu{
%A(-rval*cos(\aval),-rval*sin(\aval))ws;
%C(rval*cos(\aval),-rval*sin(\aval))es}
%\tenretu*{
%Z(rval*cos(\aval),rval*sin(\aval));
%B(-rval*cos(\aval),rval*sin(\aval))wn}
%\Drawline{\Z\B\A\C\Z}
%\Drawline{\Z\A}
%\Kakukigou\C\A\Z{$\theta$}
%\En\O{\rval}
%\En\O{1}
%\HenKo\Z\A{$2$}
%\HenKo\A\C{$2\cos\theta$}
%\HenKo\C\Z{$2\sin\theta$}
%\Kuromaru\O
%\end{zahyou*}
%\begin{zahyou*}[ul=15mm](-1.2,1.2)(-1.2,1.2)
%\teisuuretu{rval=sqrt(3)/2}
%\teisuuretu{aval=3.14/4}
%\tenretu{C(rval*cos(\aval),rval*sin(\aval));
%B(-rval*cos(\aval),rval*sin(\aval))wn;
%A(-rval*cos(\aval),-rval*sin(\aval))ws;
%D(rval*cos(\aval),-rval*sin(\aval))es}
%\Drawline{\C\B\A\D\C}
%\Drawline{\A\C}
%\En\O{\rval}
%\En\O{1}
%\HenKo\C\A{$2\cos\theta$}
%\HenKo\D\C{$\sq 2\cos\theta$}
%\HenKo\A\D{$\sq2 \cos\theta$}
%\end{zahyou*}
%\]
%\begin{align*}
%V&=4\cos^2\theta\sin\theta,\ \ S=4\cos^2\theta+8\sq 2\cos\theta\sin\theta\\
%\bunsuu{S}{V}&=\bunsuu{4\cos^2\theta+8\sq 2\cos\theta\sin\theta}{4\cos^2\theta\sin\theta}
%=\bunsuu{1}{\sin\theta}+\bunsuu{2\sq 2}{\cos\theta}=:f(\theta)\\
%f'(\theta)
%&=-\bunsuu{\cos\theta}{\sin^2 \theta}+\bunsuu{2\sq 2\sin\theta }{\cos^2\theta}
%=\bunsuu{-\cos^3\theta+2\sq 2\sin^3 \theta}{\sin^2 \theta\cos^2\theta}\\
%%&=\bunsuu{2\sq 2\sin^3 \theta-\cos^3\theta}{\sin^2 \theta\cos^2\theta}\\
%&=\bunsuu{(\sq 2\sin \theta-\cos\theta)(2\sin^2 \theta+\sqrt{2}\sin\theta\cos\theta+\cos^2\theta)}{\sin^2 \theta\cos^2\theta}\\
%&=\bunsuu{(\sq 2\tan \theta-1)(2\sin^2 \theta+\sq 2 \sin\theta\cos\theta+\cos^2\theta)}{\sin^2 \theta\cos\theta}
%\end{align*}
%
%$0<\alpha<\bunsuu{\pi}{2}$かつ$\tan\alpha=\bunsuu{1}{\sqrt{2}}$なる$\alpha$をとると,$0<\theta<\bunsuu{\pi}{2}$における$f(\theta)$の増減は次のようになる.
%\begin{mawarikomi}{30mm}{
%\begin{zahyou}[ul=15mm,yscale=0.2](0,3.14/2+0.1)(0,10)
%\def\Fx{1/sin(X)+2*sqrt(2)/cos(X)}
%\YGraph<linethickness=1pt,minx=0.01,maxx=3.14/2>\Fx
%\XGraph{$pi/2}
%\YPointPut\Fx{$pi/2*0.4}[syaei=xy,xlabel=\alpha,ylabel=3\sq 3]{}
%\end{zahyou}
%}
%\[\begin{array}{c|ccccccc}
%\phantom{\bunsuu 11} \theta	&0	 	&\cdots	&\alpha		&\cdots	 	&\frac{\pi}{2}\\\hline
%\phantom{\bunsuu 11}f'(\theta)	&		&-		&0		&+			&\\\hline
%\phantom{\bunsuu 11}f(\theta)	&		&\SE	&		&\NE		&\\
%\end{array}
%\]
%ここで,$\sin\alpha=\bunsuu{1}{\sq 3}$,
%$\cos\alpha=\bunsuu{\sq 2}{\sq 3}$より,
%\[求める最大値は,\ \ \ f(\alpha)=\sq 3+2\sq 2\times \bunsuu{\sq 3}{\sq 2}={\bb 3\sq 3}\]
%\end{mawarikomi}
%%\bqu%{\bb 最大値・最小値}\\
%%$y=2\sin x+\sin 2x$の最大値と最小値を求めよ.
%%\equ
%\end{解答}

\newpage

%\bqu%{\bb 整数問題への応用}
%\benu
%\item $y=x^{\frac 1x}$\ $(x>0)$の増減を調べよ.
%\item $a^b=b^a$\ かつ\ $a < b$を満たす正の整数の組$(a,b)$
%%$(a,b)$
%をすべて求めよ.必要であれば,$2.7<e <2.8$を用いてもよい.
%%は$(a,b)=(2,4)$以外に存在しないことを証明せよ.
%\eenu
%\equ
%
%\begin{解答}
%\benu
%\item $y=x^{\frac 1x}$は両辺正なので,自然対数をとると,
%\[\log y=\bunsuu 1x \log x\]
%両辺を$x$について微分すると
%\begin{align*}
%\bunsuu 1y \times y'&=-\bunsuu 1{x^2} \cdot \log x+\bunsuu 1x\cdot \bunsuu 1x\\
%% y'&=y\times\bunsuu{-\log x+1}{x^2}\\
% y'&=x^{\frac 1x}\times\bunsuu{-\log x+1}{x^2}
% \end{align*}
% よって,$f(x)$の増減は次のようになる.
% \[\begin{array}{c|ccccccc}
%\phantom{\bunsuu 11} x	&0 		&\cdots	&e	&\cdots \\\hline
%\phantom{\bunsuu 11}f'(x)	&/		&+		&0	&- \\\hline
%\phantom{\bunsuu 11}f(x)	&/		&\NE	&	&\SE 
%\end{array}
%\]
%\item $a^b=b^a \iff a^{\frac 1a}=b^{\frac 1b}$で,これは$f(a)=f(b)$を意味する.
%$f(x)$の増減より,$0<a<e$かつ$e<b$である必要がある.
%$2.7<e<2.8$より,$a=1,2$である.
%\benu[(i)]
%\item $a=1$のとき,$f(a)=1$であるが,$x>1$のとき,
%%$\frac 1x>0$より,
%\[f(x)=x^{\frac 1x}>x^{0}=1\]
%なので,$f(b)=1$となる$b$は$b>e$の範囲に存在しない.
%\item $a=2$のとき,$f(a)=\sq 2$であり,$f(x)$の増減より,$f(b)=\sq 2$なる実数$b$は,$e<b$の範囲にただ1つ存在する.
%$f(4)=\sqrt[4]{4}=\sq 2$より,それは$b=4$である.
%\eenu
%以上より,求める組$(a,b)=(2,4)$のみである.
%$\dlim_{x \to \infty} \log(x^{\frac 1x})=\dlim_{x \to \infty} \bunsuu 1x \log x =0$より,\\
%$\dlim_{x \to \infty} x^{\frac 1x}=1$,$y=f(x)=x^{\frac 1x}$のグラフは次の図のようになる.
%\[\begin{zahyou}[ul=5mm,yscale=10](-5,15)(0,1.5)
%\def\Fx{exp(log(X)/X)}
%\YGraph<linethickness=1pt,minx=0.001>\Fx
%\YGraph<linethickness=0.2pt>{1}
%\YPointPut\Fx{1}[syaei=xy,ypos={[nw]}]{}
%\YPointPut\Fx{2}[syaei=x]{}
%\YPointPut\Fx{3}[syaei=x]{}
%\YPointPut\Fx{4}[syaei=x]{}
%\YPointPut\Fx{2.7}[syaei=x,xlabel=e]{}
%%,ylabel=e^{\frac 1e}
%%\YPointPut\Fx{5}[syaei=xy]{}
%%\YPointPut\Fx{6}[syaei=xy]{}
%%\YPointPut\Fx{7}[syaei=xy]{}
%\end{zahyou}\]
%\eenu
%\end{解答}


\bqu%{\bb 整数問題への応用}
次の問いに答えよ.
\benu
\item $f(x)=\bunsuu{\log x}{x^2}$\ $(x>0)$の増減を調べよ.\hfill (5点)
\item $a^{b^2}=b^{a^2}$\ かつ\ $a < b$を満たす自然数の組$(a,b)$は存在するか.
%$(a,b)$
%をすべて求めよ.
%必要であれば,$2.7<e <2.8$を用いてもよい.
ただし,$a^{b^2}$は底が$a$であり指数が$b^2$である累乗を表し,$b^{a^2}$は底が$b$で指数が$a^2$である累乗を表している.\hfill (10点)
%は$(a,b)=(2,4)$以外に存在しないことを証明せよ.
\eenu
\equ

\ifkaitou
\begin{解答}
\benu
\item 
\begin{align*}
f'(x)&=-\bunsuu 2{x^3} \cdot \log x+\bunsuu 1{x^2}\cdot \bunsuu 1x\\
% y'&=y\times\bunsuu{-\log x+1}{x^2}\\
&=\bunsuu{-2\log x+1}{x^3}
 \end{align*}
 よって,$f(x)$の増減は次のようになる.
 \[\begin{array}{c|ccccccc}
\phantom{\bunsuu 11} x	&0 		&\cdots	&e^{\frac 12}	&\cdots \\\hline
\phantom{\bunsuu 11}f'(x)	&/		&+		&0				&- \\\hline
\phantom{\bunsuu 11}f(x)	&/		&\NE	&				&\SE 
\end{array}
\]
\item 
\begin{align*}
a^{b^2}=b^{a^2} & \iff \log a^{b^2}=\log b^{a^2}\\
%& \iff \log a^{b^2}=\log b^{a^2}\\
& \iff {b^2}\log a={a^2}\log b\\
& \iff \bunsuu{\log a}{a^2}=\bunsuu{\log b}{b^2}\\
& \iff f(a)=f(b)
\end{align*}
$f(x)$の増減より,$0<a<e^{\frac 12}$かつ$e^{\frac 12}<b$である必要がある.$e^{\frac 12}<2.8^{\frac 12}<2$より,$a=1$.\\
このとき,$f(b)=f(a)=f(1)=0$が必要だが,
%$\frac 1x>0$より,
\[x>1\ のとき,\ \ f(x)=\bunsuu{\log x}{x^2} >0\]
なので,$f(b)=0$となる$b$は$b>e^{\frac 12}$の範囲に存在しない.\\
以上より,$a^{b^2}=b^{a^2}$なる自然数の組$(a,b)$は存在しない.\hfill □
\eenu
\end{解答}
\fi

%\begin{解答}
%\benu
%\item $y=x^{\frac 1{x^2}}$は両辺正なので,自然対数をとると,
%\[\log y=\bunsuu 1{x^2} \log x\]
%両辺を$x$について微分すると
%\begin{align*}
%\bunsuu 1y \times y'&=-\bunsuu 2{x^3} \cdot \log x+\bunsuu 1{x^2}\cdot \bunsuu 1x\\
%% y'&=y\times\bunsuu{-\log x+1}{x^2}\\
% y'&=x^{\frac 1{x^2}}\times\bunsuu{-2\log x+1}{x^3}
% \end{align*}
% よって,$f(x)$の増減は次のようになる.
% \[\begin{array}{c|ccccccc}
%\phantom{\bunsuu 11} x	&0 		&\cdots	&e^{\frac 12}	&\cdots \\\hline
%\phantom{\bunsuu 11}f'(x)	&/		&+		&0				&- \\\hline
%\phantom{\bunsuu 11}f(x)	&/		&\NE	&				&\SE 
%\end{array}
%\]
%\item $a^{b^2}=b^{a^2} \iff a^{\frac 1{a^2}}=b^{\frac 1{b^2}}$で,これは$f(a)=f(b)$を意味する.
%$f(x)$の増減より,$0<a<e^{\frac 12}$かつ$e^{\frac 12}<b$である必要がある.$e^{\frac 12}<2.8^{\frac 12}<2$より,$a=1$.\\
%このとき,$f(a)=f(b)\iff f(b)=1$であるが,
%%$\frac 1x>0$より,
%\[x>1\ のとき,\ \ f(x)=x^{\frac 1{x^2}}>x^{0}=1\]
%なので,$f(b)=1$となる$b$は$b>e$の範囲に存在しない.
%以上より,存在しない$a^{b^2}=b^{a^2}$なる自然数の組$(a,b)$は存在しない.\hfill □
%%$\dlim_{x \to \infty} \log(x^{\frac 1x})=\dlim_{x \to \infty} \bunsuu 1x \log x =0$より,\\
%%$\dlim_{x \to \infty} x^{\frac 1x}=1$,$y=f(x)=x^{\frac 1x}$のグラフは次の図のようになる.
%%\[\begin{zahyou}[ul=10mm](-1,5)(0,2)
%%\def\Fx{exp(log(X)/(X*X))}
%%\YGraph<linethickness=1pt,minx=0.001>\Fx
%%\YGraph<linethickness=0.2pt>{1}
%%\YPointPut\Fx{1}[syaei=xy,ypos={[nw]}]{}
%%\YPointPut\Fx{2}[syaei=x]{}
%%\YPointPut\Fx{sqrt(2.7)}[syaei=x,xlabel=e^{\frac 12}]{}
%%\end{zahyou}\]
%\eenu
%\end{解答}

\newpage

\bqu%{\bb メジアン}\\
1辺の長さが2の正四面体OABCにおいて,辺OA上に点Pをとり,内積$\vv
{\text{OA}}\cdot \vv{\text{PB}}=1$とする.さらに,辺PB上に点Qを,PBとCQが垂直にあるようにとる.OP:PAとPQ:QBをそれぞれ求めよ.\\\hfill (15点)
%$\vv{\text{OP}}$を$\vv{\text{OA}}$を用いて表せ.
%\benu
%\item 
%\item $\text{PB}$:$\text{PQ}$を求めよ.
%\eenu
\equ

\ifkaitou
\begin{解答}
$\vv{\text{OA}}\cdot \vv{\text{PB}}=1\cdots\MARU{1}$\ \ \ $\vv{\text{PB}}\cdot \vv{\text{CQ}}=0\cdots\MARU{2}$\\
$\vv{\text{OP}}=t\vv{\text{OA}}$,\ \ \ $\vv{\text{OQ}}=s\vv{\text{OP}}+(1-s)\vv{\text{OB}}$\ \ \ $(t,sは実数)$とおく.
\begin{align*}
&\MARU{1}より,\\
&\vv{\text{OA}}\cdot (\vv{\text{OB}}-\vv{\text{OP}})=1\\
&\vv{\text{OA}}\cdot \vv{\text{OB}}-\vv{\text{OA}}\cdot \vv{\text{OP}}=1\\
&\vv{\text{OA}}\cdot \vv{\text{OB}}-t|\vv{\text{OA}}|^2=1\\
&2\cdot 2\cdot \bunsuu 12 -t\cdot 2^2=1\ \ \therefore\ \ \ t=\bunsuu 14
\end{align*}
よって,$\vv{\text{OP}}=\bunsuu 14\vv{\text{OA}}$\ \ $\therefore$\ \ \ \text{\bb $\text{OP:PA}=1:3$}
\begin{align*}
&\MARU{2}より,\\
&(\vv{\text{OB}}-\vv{\text{OP}})\cdot (\vv{\text{OQ}}-\vv{\text{OC)}}=0\\
&\left(\vv{\text{OB}}-\bunsuu 14\vv{\text{OA}}\right)\cdot \left(\bunsuu 14s\vv{\text{OA}}+(1-s)\vv{\text{OB}}-\vv{\text{OC}}\right)=0\\
%\bunsuu 14s \vv{\text{OB}}\cdot \vv{\text{OA}}+(1-s)|\vv{\text{OB}}|^2-\vv{\text{OB}}\cdot \vv{\text{OC}}\\
%\ \ \ -\bunsuu{1}{16}s|\vv{\text{OA}}|^2-\bunsuu 14(1-s)\vv{\text{OA}}\cdot \vv{\text{OB}}+\bunsuu 14 \vv{\text{OA}}\cdot \vv{\text{OC}}&=0\\
&\scalebox{0.95}[1]{$\bunsuu 14s \cdot 2+(1-s)\cdot 2^2-2-\bunsuu{1}{16}s\cdot 2^2-\bunsuu 14(1-s)\cdot 2+\bunsuu 14 \cdot 2=0$}\\
&\bunsuu s2+4-4s-2-\bunsuu{s}{4}-\bunsuu 12+\bunsuu s2+\bunsuu 12=0\\
&2-\bunsuu{13}{4}s=0\ \ \therefore\ \ s=\bunsuu{8}{13}
\end{align*}
よって,$\vv{\text{OQ}}=\bunsuu{8}{13}\vv{\text{OP}}+\bunsuu{5}{13}\vv{\text{OB}}$\ \ $\therefore$\ \ \ \text{\bb $\text{PQ:QB}=5:8$}
\end{解答}
\fi

%\vfill
%
%\hfill (30点)

%\advance\K by 1\repeat }
%
%\barabara

\end{document}