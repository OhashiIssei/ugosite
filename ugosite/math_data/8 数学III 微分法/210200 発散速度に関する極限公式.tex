\newif\iffuru
%\furutrue%フルバージョン
\furufalse%解説のみ

\newif\iffigure
%\figuretrue%図表あり
\figurefalse%図表なし

\documentclass[10pt,
b5paper,
%twocolumn,
fleqn,
%landscape, 
%papersize
dvipdfmx,
uplatex
]{jsarticle}

\def\maru#1{\textcircled{\scriptsize#1}}%丸囲み番号

%\renewcommand{\bf}{}
%\usepackage{アプローチ}
%\usepackage{amsthm}
\usepackage{amsmath}
\usepackage{ascmac}
%\usepackage{graphics}
\usepackage{emath}
\usepackage{emathMw}
\usepackage{enumerate}
\usepackage{emathC}
\usepackage{emathEy}
\usepackage{emathP}
\usepackage{emathPp}
\usepackage{emathPl}
\usepackage{emathPk}
\usepackage{emathPh}
\usepackage{emathPs}
\usepackage[g]{esvect}
\usepackage{color}
\usepackage{pxrubrica}%ふりがな
%\usepackage{EMesvect}
%\usepackage[dvipdfmx]{graphicx}%箱ひげ図
%\usepackage{emathSt}%箱ひげ図
%\usepackage{emathG}%箱ひげ図%ヒストグラム
\usepackage{emathPs}
\usepackage{ascmac}%囲み
\usepackage{fancybox}
%\usepackage{fancybx}%囲み
\usepackage[top=12truemm,bottom=12truemm,left=12truemm,right=10truemm]{geometry}
\usepackage{setspace} % 行間
\setstretch{1} % ページ全体の行間を設定
\usepackage{wallpaper}
\usepackage{hako}%センター形式
\usepackage{mathtools}%\abs(絶対値)など
\DeclarePairedDelimiter{\abs}{\lvert}{\rvert}
\usepackage{fancyhdr}%ヘッダの設定
\usepackage{cancel}%消し取り線
\usepackage{ifthen}
%\usepackage{exam}
\usepackage{tcolorbox}
\usepackage{extarrows}%伸縮性のある矢印

%ページレイアウト
%\setlength{\columnseprule}{0.4pt}
%\columnsep=3cm
%%\setlength{\mathindent}{0zw}
\preHEqlabel{$\cdotfill[2em]~$}%houtesikiの点線の長さ
%\linespread{1.3}
%\setstretch{1.2}%行間
\postEqlabel{\hspace{0zw}\null}%式番号の位置
\preEqlabel*{\cdotpfill[2em]~}%式番号の点々の長さ
%\setlength{\columnseprule}{0.4pt}
%\columnsep=1cm
%\setlength{\mathindent}{1zw}%数式の位置


%\pagestyle{headings}
\pagestyle{empty}%ページ番号を消す
% \pagestyle{fancy}
%  \fancyhead{}
%  \fancyhead[RO,RE]{\rightmark}
%%  \fancyhead[LE,LO]{\leftmark}
%  \cfoot{\thepage}
%  \renewcommand{\chaptermark}[1]{\markboth{第\ \thechapter\ 章~#1}{}}
%  \renewcommand{\sectionmark}[1]{\markright{\thesection #1}{}}
 
%\markboth{}{\thesection}
 
%\fancyhead{} % clear all fields
%\fancyhead[CE]{偶数ページ}
%\fancyhead[CO]{奇数ページ}

%定理環境
\usepackage{emathThm}
%\theoremstyle{boxed}
\theorembodyfont{\normalfont}
\newtheorem{Question}{問題}[subsection]
\newtheorem{Q}{}[subsection]
\newtheorem{question}[Question]{}
\newtheorem{quuestion}{}[subsection]

%問題レイアウト
\tcbuselibrary{raster,skins}
\newenvironment{黒tcolorbox}{
\begin{tcolorbox}[enhanced,
frame style={left color=orange!50!white,
right color=black!80!orange},
colback=black!0!white,
drop fuzzy shadow]}{\end{tcolorbox}}
\newcommand{\sub}{\newpage\ \vspace{-4zw}\subsection}
\newcommand{\bqu}{\begin{黒tcolorbox}\begin{question}}
\newcommand{\equ}{\end{question}\end{黒tcolorbox}}
\newcommand{\mondaisettei}{\kaisetukaitoufalse
\renewcommand{\sub}{\subsection}
\renewcommand{\bqu}{\vspace{0.5zw}\begin{question}}
\renewcommand{\equ}{\end{question}\vspace{3zw}\vfill}
}%問題設定
\newcommand{\kaisetutukinosettei}{\kaisetukaitoutrue
\renewcommand{\sub}{\newpage\ \vspace{-4zw}\subsection}
\renewcommand{\bqu}{\begin{黒tcolorbox}\begin{question}}
\renewcommand{\equ}{\end{question}\end{黒tcolorbox}}
}%解答解説の設定

%箇条書きの調整
%\setlength{\itemsep}{5pt}      %2. ブロック間の余白
%\setlength{\parskip}{0pt}      %4. 段落間余白.
%\setlength{\itemindent}{0pt}   %5. 最初のインデント
%\setlength{\labelsep}{5pt}     %6. item と文字の間

%箇条書き省略コマンド
\newcommand{\benu}{\begin{enumerate}}
\newcommand{\eenu}{\end{enumerate}}
\newcommand{\beda}{\begin{edaenumerate}}
\newcommand{\eeda}{\end{edaenumerate}}

\newcommand{\bb}{\bf\boldmath}%全部太字にする
%\newcommand{\bb}{\gtfamily\ebseries\boldmath}%全部極太にする
\newcommand{\doo}{^{\circ}}%角度マーク
\newcommand{\sq}{\textstyle\sqrt}
\newcommand{\ANA}{\hakosenhaba{1pt}\Hako}
\newcommand{\REFANA}{\hakosenhaba{0.3pt}\refHako*}
\newcommand{\C}{\text{C}}
\newcommand{\dsum}{\displaystyle\sum}
\newcommand{\barr}{\left\{\begin{array}{l}}
\newcommand{\earr}{\end{array}\right.}
\newcommand{\cdotss}{\hfill\cdots\cdots}

\usepackage{tabularx}
%\newcolumntype{Y}{&gt;{\centering\arraybackslash}X} %中央揃え
%\includegraphics[width=90mm,bb=9 9 358 434]{./lrep_e1.eps}

%セクション,大問番号のデザイン
\renewcommand{\labelenumi}{(\arabic{enumi})}
%\renewcommand{\labelenumi}{\ \fbox{\protect\makebox[1em][c]{\large{\bfseries\arabic{enumi}}}}\ }
%\renewcommand{\labelenumi}{\textbf{\theenumi}}
%\renewcommand{\theenumiii}{(\alph{enumiii})}
%\renewcommand{\theenumii}{\arabic{enumii}}
%\renewcommand{\thesection}{\Huge  第\arabic{section}章}
\renewcommand{\thesubsection}{\bb 第\arabic{subsection}回
\ }
\renewcommand{\theQuestion}{%\arabic{subsection}-
\large\arabic{Question}.}

%横に縦線
\usepackage{framed}
\makeatletter
\renewenvironment{leftbar}{%
\def\FrameCommand{\vrule width 1pt \hspace{1zw}}
\MakeFramed{\advance\hsize-\width \FrameRestore}}%
{\endMakeFramed}
\makeatother

\newenvironment{leftbbar}{%
\def\FrameCommand{\color{mygray} \vrule width 5pt \hspace{1zw}
\color{black}}%
\MakeFramed {\advance\hsize-\width \FrameRestore}}%
{\endMakeFramed}
\makeatother

%アプローチ
\newenvironment{アプローチ}{
\hspace{-2zw}\underbar{\large \bf Approach}\begin{leftbar}}{\end{leftbar}}

\newenvironment{解答}{
\hspace{-2zw}\phkasen<linethickness=7pt,iro=mygray,kasenUehosei=-3pt>{\bf \large \ 解答\ }\vspace{-1zw}\begin{leftbbar}}{\end{leftbbar}}

\newenvironment{証明}{
\hspace{-2zw}\phkasen<linethickness=7pt,iro=mygray,kasenUehosei=-3pt>{\bf \large \ 証明\ }\vspace{-1zw}\begin{leftbbar}}{\end{leftbbar}}

\newenvironment{解答1}{
\hspace{-2zw}\phkasen<linethickness=7pt,iro=mygray,kasenUehosei=-3pt>{\bf \large \ 解答1\ }\vspace{-1zw}\begin{leftbbar}}{\end{leftbbar}}
\newenvironment{解答2}{
\hspace{-2zw}\phkasen<linethickness=7pt,iro=mygray,kasenUehosei=-3pt>{\bf \large \ 解答2\ }\vspace{-1zw}\begin{leftbbar}}{\end{leftbbar}}
\newenvironment{解答3}{
\hspace{-2zw}\phkasen<linethickness=7pt,iro=mygray,kasenUehosei=-3pt>{\bf \large \ 解答3\ }\vspace{-1zw}\begin{leftbbar}}{\end{leftbbar}}

\newcommand{\kaitoui}{{\bb \color{mygray} $\hookrightarrow$}\phkasen<linethickness=7pt,iro=mygray,kasenUehosei=-3pt>{\bf \ 解答1\ }}
\newcommand{\kaitouii}{{\bb \color{mygray} $\hookrightarrow$}\phkasen<linethickness=7pt,iro=mygray,kasenUehosei=-3pt>{\bf \ 解答2\ }}
\newcommand{\kaitouiii}{{\bb \color{mygray} $\hookrightarrow$}\phkasen<linethickness=7pt,iro=mygray,kasenUehosei=-3pt>{\bf \ 解答3\ }}

%QRコード
\usepackage{qrcode}
\setlength\normallineskiplimit{0pt}

%カラー,色
\usepackage{color}
\definecolor{link}{rgb}{0.63671875,0.99609375,0.99609375}
\definecolor{usumido}{rgb}{0.953125,0.95703125,0.9375}
%\pagecolor{usumido}
\definecolor{mygray}{gray}{0.75}

\newif\ifkaisetukaitou

\newcommand{\kaisetukaitou}{%問題のみ
\mondaisettei
\myfor{1} % ループ実行       
\newpage   
\setcounter{subsection}{0}
\setcounter{Question}{0}
\kaisetutukinosettei
\myfor{1} % ループ実行   
%\TileWallPaper{110mm}{160mm}{方眼紙.pdf} %方眼紙
%\myfor{1} % ループ実行   
}
\begin{document}

{\bb\Large 第1回\ \ $1.1^n$と$n^{10}$のどちらがより速く発散するか?}

%発散速度に関する極限公式とその証明をここにまとめておく.結果だけでなく,証明にも「二項定理」「はさみうちの手法」など多様な数学のテーマが現れるので,ぜひ一度証明を辿ってみてほしい.
%大学入試においては,問題文中でこれらの結果が用意されていることもあるが,誘導に従ってこれを証明してから利用することも求められる.いずれはこれらの証明を,自分で1から再構成できるようになっておくとよい.
%\hfill 大橋\\

数列$\{1.1^n\}$も$\{n^{10}\}$も,無限大に発散することには変わりない.
それでは,どちらの方が速く発散するだろうか?概数を表に示す.
\[\small \begin{array}{c||r|r|r|r}
\phantom{\bunsuu 11}n		&1		&10		&100	&1000	\\\hline
\phantom{\bunsuu 11}1.1^n	&1.1	&2.5937424601	&13780.6123398224&246993291800603000000000000000000000000000\\\hline
\phantom{\bunsuu 11}n^{10}	&1		&10000000000	&100000000000000000000&1000000000000000000000000000000
\end{array}\]
言葉を用意する.正の無限大に発散するような2つの数列$\{a_n\}$,$\{b_n\}$が$\dlim_{n \to \infty} \bunsuu{a_n}{b_n}=0$を満たすとき,$\{a_n\}$は$\{b_n\}$より速く発散する,といい,
\[ nが十分に大きいとき,\ \ \ a_n \ll b_n\]
と表す.これは$\dlim_{n \to \infty} \bunsuu{b_n}{a_n}=\infty$と同値である.
%まずは,数列についてまとめると,次のようになる;

\begin{tcolorbox}[enhanced,
frame style={left color=orange!50!white,
right color=black!80!orange},
colback=black!0!white,
drop fuzzy shadow,
title={\bb 各種数列の発散速度},
coltitle=black]
\bb 1より大きい実数$r$と自然数$k$に対して,\ \ \ $\dlim_{n\to\infty} \bunsuu{n^k}{r^n}=0,\ \ \ \ \ 
\dlim_{n\to\infty} \bunsuu{r^n}{n!}=0$\vspace{0.5zw}\\
つまり,$n\ が十分大きいとき,\ \ \ {n^k}\ll {r^n}\ll{n!}$
\end{tcolorbox}

%これらの数列の各項が正の数であることに注意して逆数を取ると,
%\[\dlim_{n\to\infty} \bunsuu{r^n}{n^k}=\infty,\ \ \ \ \ 
%\dlim_{n\to\infty} \bunsuu{n!}{r^n}=\infty\]
%とも表現できる.これらが意味するのは,
%\[n\ が十分大きいところでは,{r^n}\ は\ {n^k}\ に比べてずいぶん大きく,{n!}\ は\ {r^n}\ に比べてずいぶん大きい\]
%\[\{r^n\}\ は\ \{n^k\}\ に比べてはるかに速く,\{n!\}\ は\ \{r^n\}\ に比べてはるかに速く,無限大に発散する\]
%というようなことである.このことを,
%と表現することもある.
これを表題に当てはめると,答えは意外にも${1.1^n}\gg {n^{10000}}$となる.\\
%証明は$r^n=((r-1)+1)^n$を二項定理によって展開したときに現れる
%${}_n\C_l$が$n$についての$l$次式であることを利用する.

\begin{証明}\vspace{-2.5 zw}\ \ 
$n \geqq  k+1$のとき,$r^n
=((r-1)+1)^n=\dsum_{l=0}^n{}_n\C_l(r-1)^l>{}_n\C_{k+1}(r-1)^{k+1}$より,
\begin{align*}
\bunsuu{n^k}{r^n}
& \leqq \bunsuu{n^k}{{}_n\C_{k+1}(r-1)^{k+1}}\\
&=\bunsuu{(k+1)k(k-1)\cdots 2\cdot 1}{\underbrace{n(n-1)\cdots (n-k)}_{k+1個}}\cdot \bunsuu{ n^k}{(r-1)^{k+1}}\\
&=\bunsuu{(k+1)k(k-1)\cdots 2\cdot 1}{n(1-\frac1n)\cdots (1-\frac{k}{n})}\cdot \bunsuu{1}{(r-1)^{k+1}}
\xlongrightarrow[]{n\to \infty} 0\ \ \ \therefore\ \ \ \dlim_{n\to\infty} \bunsuu{n^k}{r^n}=0
 \end{align*}
 
 一方,$N \leqq r < N+1 $なる自然数$N$をとすると,$n\geqq N+1$のとき,
 \begin{align*}
 \bunsuu{r^n}{n!}
% &=\bunsuu{r\cdot r\cdot r\cdot \cdots\cdots\cdots\cdots\cdots\cdots\cdots \cdot r}{1\cdot 2\cdot 3\cdot \cdots\cdots\cdots\cdots\cdots\cdots\cdots \cdot n}\\
&=\bunsuu{r\cdot r\cdot r\cdot \cdots \cdot r}{1\cdot 2\cdot 3\cdot \cdots \cdot N} \cdot\bunsuu{r\cdot r\cdot r\cdot \cdots \cdot r}{(N+1) \cdot \cdots \cdot n}\\
& \leqq \bunsuu{r\cdot r\cdot r\cdot \cdots \cdot r}{1\cdot 2\cdot 3\cdot \cdots \cdot N} \cdot\bunsuu{r\cdot r\cdot r\cdot \cdots \cdot r}{\underbrace{(N+1) \cdot \cdots \cdot (N+1)}_{(n-N)個}}\\
&=\bunsuu{r\cdot r\cdot r\cdot \cdots \cdot r}{1\cdot 2\cdot 3\cdot \cdots \cdot N} \cdot \left(\bunsuu{r}{(N+1)}\right)^{n-N}\xlongrightarrow[]{n\to \infty} 0\ \ \ \therefore\ \ \ \dlim_{n\to\infty} \bunsuu{r^n}{n!}=0\hspace{5zw}\hfill □
\end{align*}
\end{証明}

\newpage

{\bb\Large 第2回\ $\sq{x}$と$\log x$で,発散速度がより遅いのはどちらか?}

\begin{mawarikomi}{}{
\begin{zahyou}[ul=10mm](0.01,8)(-0.3,3)
\YGraph<linethickness=1pt>{sqrt(X)}
\YGraph<linethickness=1pt>{log(X)}
\YGraph<linethickness=0.5pt>{X}
\end{zahyou}
}
無理関数$y=\sq x$も,対数関数$y=\log x$も,$x\to \infty$のとき,$y \to \infty$であることには変わりない.そしてその発散速度はいづれも$y=x$より遅いことは想像するに容易い.では,より遅いのはどちらだろうか?
\end{mawarikomi}

\vspace{3zw}
%まずは,関数についても重要なものを挙げておく;
\begin{tcolorbox}[enhanced,
frame style={left color=orange!50!white,
right color=black!80!orange},
colback=black!0!white,
drop fuzzy shadow,
title={\bb 各種関数の発散速度},
coltitle=black]
\bb 
正の実数$\alpha$に対して,\ \ \ 
$\dlim_{x\to\infty} \bunsuu{x^\alpha}{e^x}=0,\ \ \ \ \ \dlim_{x\to\infty} \bunsuu{\log x}{x^\alpha}=0$\vspace{0.5zw}\\
つまり,$x\ が十分大きいとき,\ \ \ \log x\ll x^{\alpha}\ll e^x$
\end{tcolorbox}


%これらの数列の各項が正の数であることに注意して逆数を取ると,
%\[\dlim_{x\to\infty} \bunsuu{e^x}{x^\alpha}=\infty,\ \ \ \ \ \dlim_{x\to\infty} \bunsuu{x^\alpha}{\log x}=\infty
%\]
%とも表現できる.これらが意味するのは,
%\[x\ が十分大きいところでは,e^x\ は\ x^{\alpha}\ に比べてずいぶん大きく,x^{\alpha}\ は\ \log x\ に比べてずいぶん大きい\]
%ということである.このことを,

%前回紹介した表現を借りるなら,これは
%\[\]
%%\[x\ が十分に0に近いとき,\ \ \ \busnuu{1}{x^{\alpha}}\gg \log x\]
%ということになる.

\begin{証明}
\vspace{-1.8 zw}
\ \ まず,$n -1\leqq x <n$,$k-1\leqq \alpha<k$なる自然数$n,k$をとれば,\\$x \to \infty$のとき$n\to \infty$となるので,
\[\bunsuu{x^\alpha}{e^x} < \bunsuu{n^k}{e^{n-1}}=\bunsuu{n^k}{e^n}\cdot e \xlongrightarrow[]{x\to \infty} 0\ \ \ \therefore\ \ \ \dlim_{x\to\infty} \bunsuu{x^\alpha}{e^x}=0\]

さらに,$\log x =t$とおくと,$x \to \infty$のとき,$t \to \infty$より,
\[\bunsuu{\log x}{x^{\alpha}}=\bunsuu{t}{(e^{t})^{\alpha}}=\left(\bunsuu{t^{\frac{1}{\alpha}}}{e^{t}}\right)^{\alpha}
\xlongrightarrow[]{x\to \infty} 0^{\alpha}=0\ \ \ \therefore\ \ \ \dlim_{x\to\infty} \bunsuu{\log x}{x^\alpha}=0\]
%\[\dlim_{x\to\infty} \bunsuu{x^\alpha}{e^x}=0
%\ \ \xlongleftrightarrow[]{y=e^x} \ \ \dlim_{y\to\infty} \bunsuu{(\log y)^\alpha}{y}=0
%\iff \dlim_{y\to\infty} \bunsuu{\log y}{y^{\frac{1}{\alpha}}}=0
%\ \ \xlongleftrightarrow[]{\beta=\frac{1}{\alpha}}\ \ \dlim_{y\to\infty} \bunsuu{\log y}{y^{\beta}}=0
%\ \ \xlongleftrightarrow[]{t=\frac1y} \ \ \dlim_{t\to +0} {t^{\beta}}{\log t}=0
%\]
%証明はどちらかだけで十分である.%のは$\dlim_{x\to\infty} \bunsuu{x^\alpha}{e^x}=0$の方である.\\
%よって,どれかを1つを示せば全て証明したことになる.一番手っ取り早いのは,前回の数列の場合に帰着させることである.

\end{証明}
証明の別のアプローチとして,不等式
\[e^x>1+x+\bunsuu{x^2}{2!}+\bunsuu{x^3}{3!}+\cdots+\bunsuu{x^n}{n!}\ (n\ は自然数)\]
を利用する方法も考えられる.

これで,表題の答えは$\sq x=x^{\frac 12} \gg \log x$ということがわかった.驚きなのは$\alpha=0.00001$であっても$x^{0.00001} \gg \log x$ということだ.対数関数が如何に遅いかがわかる.

%,\ \ \ \ \ \dlim_{x\to +0}{x^\alpha} {\log x}=0$

%\begin{証明}
% $n-1 \leqq \alpha < n$となる自然数$n$をとると,\MARU{1}より,$e^x >\bunsuu{x^n}{n!}$であるので,
%\[
%\bunsuu{x^{\alpha}}{e^x} <x^{\alpha}\cdot \bunsuu{n!}{x^n}= \bunsuu{n!}{x^{n-\alpha}} \xlongrightarrow[]{x\to \infty} 0
%\ \ \ \therefore\ \ \ \dlim_{x\to\infty} \bunsuu{x^{\alpha}}{e^x}=0\hfill □\]
%\end{証明}

\iffuru
}\kaisetukaitou
\fi                                                                 

\end{document}