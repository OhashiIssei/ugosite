\newif\iffuru
%\furutrue%フルバージョン
\furufalse%解説のみ

\newif\iffigure%図表
\figuretrue%図表あり
%\figurefalse%図表なし

\documentclass[10pt,
a4paper,
%twocolumn,
fleqn,
%landscape, 
%papersize
dvipdfmx,
uplatex
]{jsarticle}

\def\maru#1{\textcircled{\scriptsize#1}}%丸囲み番号

%\renewcommand{\bf}{}
%\usepackage{アプローチ}
%\usepackage{amsthm}
\usepackage{amsmath}
\usepackage{ascmac}
%\usepackage{graphics}
\usepackage{emath}
\usepackage{emathMw}
\usepackage{enumerate}
\usepackage{emathC}
\usepackage{emathEy}
\usepackage{emathP}
\usepackage{emathPp}
\usepackage{emathPl}
\usepackage{emathPk}
\usepackage{emathPh}
\usepackage{emathPs}
\usepackage[g]{esvect}
\usepackage{color}
\usepackage{pxrubrica}%ふりがな
%\usepackage{EMesvect}
%\usepackage[dvipdfmx]{graphicx}%箱ひげ図
%\usepackage{emathSt}%箱ひげ図
%\usepackage{emathG}%箱ひげ図%ヒストグラム
\usepackage{emathPs}
\usepackage{ascmac}%囲み
\usepackage{fancybox}
%\usepackage{fancybx}%囲み
\usepackage[top=20truemm,bottom=20truemm,left=12truemm,right=10truemm]{geometry}
\usepackage{setspace} % 行間
\setstretch{1} % ページ全体の行間を設定
\usepackage{wallpaper}
\usepackage{hako}%センター形式
\usepackage{mathtools}%\abs(絶対値)など
\DeclarePairedDelimiter{\abs}{\lvert}{\rvert}
\usepackage{fancyhdr}%ヘッダの設定
\usepackage{cancel}%消し取り線
\usepackage{ifthen}
%\usepackage{exam}
\usepackage{tcolorbox}

%ページレイアウト
%\setlength{\columnseprule}{0.4pt}
%\columnsep=3cm
%%\setlength{\mathindent}{0zw}
\preHEqlabel{$\cdotfill[2em]~$}%houtesikiの点線の長さ
%\linespread{1.3}
%\setstretch{1.2}%行間
\postEqlabel{\hspace{0zw}\null}%式番号の位置
\preEqlabel*{\cdotpfill[2em]~}%式番号の点々の長さ
%\setlength{\columnseprule}{0.4pt}
%\columnsep=1cm
%\setlength{\mathindent}{1zw}%数式の位置


%\pagestyle{headings}
\pagestyle{empty}%ページ番号を消す
% \pagestyle{fancy}
%  \fancyhead{}
%  \fancyhead[RO,RE]{\rightmark}
%%  \fancyhead[LE,LO]{\leftmark}
%  \cfoot{\thepage}
%  \renewcommand{\chaptermark}[1]{\markboth{第\ \thechapter\ 章~#1}{}}
%  \renewcommand{\sectionmark}[1]{\markright{\thesection #1}{}}
 
%\markboth{}{\thesection}
 
%\fancyhead{} % clear all fields
%\fancyhead[CE]{偶数ページ}
%\fancyhead[CO]{奇数ページ}

%定理環境
\usepackage{emathThm}
%\theoremstyle{boxed}
\theorembodyfont{\normalfont}
\newtheorem{Question}{問題}[subsection]
\newtheorem{Q}{}[subsection]
\newtheorem{question}[Question]{}
\newtheorem{quuestion}{}[subsection]

%問題レイアウト
\newcommand{\sub}{\newpage\ \vspace{-4zw}\subsection}
\newcommand{\bqu}{\begin{tcolorbox}\vspace{0.5zw}\begin{question}}
\newcommand{\equ}{\end{question}\vspace{0.5zw}\end{tcolorbox}}
\newcommand{\mondaisettei}{\kaisetukaitoufalse
\renewcommand{\sub}{\subsection}
\renewcommand{\bqu}{\vspace{0.5zw}\begin{question}}
\renewcommand{\equ}{\end{question}\vspace{3zw}\vfill}
}%問題設定
\newcommand{\kaisetutukinosettei}{\kaisetukaitoutrue
\renewcommand{\sub}{\newpage\ \vspace{-4zw}\subsection}
\renewcommand{\bqu}{\begin{tcolorbox}\vspace{0.5zw}\begin{question}}
\renewcommand{\equ}{\end{question}\vspace{0.5zw}\end{tcolorbox}}
}%解答解説の設定

%箇条書き省略コマンド
\newcommand{\benu}{\begin{enumerate}}
\newcommand{\eenu}{\end{enumerate}}
\newcommand{\beda}{\vspace{-1zw}\begin{edaenumerate}}
\newcommand{\eeda}{\end{edaenumerate}}
\newcommand{\bb}{\bf\boldmath}%全部太字にする
%\newcommand{\bb}{\gtfamily\ebseries\boldmath}%全部極太にする
\newcommand{\doo}{^{\circ}}%角度マーク
\newcommand{\sq}{\textstyle\sqrt}
\newcommand{\ANA}{\hakosenhaba{1pt}\Hako}
\newcommand{\REFANA}{\hakosenhaba{0.3pt}\refHako*}
\newcommand{\C}{\text{C}}
\newcommand{\dsum}{\displaystyle\sum}
\newcommand{\barr}{\left\{\begin{array}{l}}
\newcommand{\earr}{\end{array}\right.}
\newcommand{\cdotss}{\hfill\cdots\cdots}

\usepackage{tabularx}
%\newcolumntype{Y}{&gt;{\centering\arraybackslash}X} %中央揃え
%\includegraphics[width=90mm,bb=9 9 358 434]{./lrep_e1.eps}

%セクション,大問番号のデザイン
\renewcommand{\labelenumi}{(\arabic{enumi})}
%\renewcommand{\labelenumi}{\ \fbox{\protect\makebox[1em][c]{\large{\bfseries\arabic{enumi}}}}\ }
%\renewcommand{\labelenumi}{\textbf{\theenumi}}
%\renewcommand{\theenumiii}{(\alph{enumiii})}
%\renewcommand{\theenumii}{\arabic{enumii}}
%\renewcommand{\thesection}{\Huge  第\arabic{section}章}
\renewcommand{\thesubsection}{\bb 第\arabic{subsection}回
\ }
\renewcommand{\theQuestion}{%\arabic{subsection}-
\arabic{Question}.}

%横に縦線
\usepackage{framed}

\makeatletter
\renewenvironment{leftbar}{%
\def\FrameCommand{\vrule width 1pt \hspace{1zw}}
\MakeFramed{\advance\hsize-\width \FrameRestore}}%
{\endMakeFramed}
\makeatother


\newenvironment{leftbbar}{%
\def\FrameCommand{\color{mygray} \vrule width 5pt \hspace{1zw}
\color{black}}%
\MakeFramed {\advance\hsize-\width \FrameRestore}}%
{\endMakeFramed}
\makeatother

%アプローチ
\newenvironment{アプローチ}{
\hspace{-2zw}\underbar{\bf Approach}\vspace{-1zw}\begin{leftbar}}{\end{leftbar}}
\newenvironment{アプローチ1}{
\hspace{-2zw}\underbar{\bf Approach1}\vspace{-1zw}\begin{leftbar}}{\end{leftbar}}
\newenvironment{アプローチ2}{
\hspace{-2zw}\underbar{\bf Approach2}\vspace{-1zw}\begin{leftbar}}{\end{leftbar}}
\newenvironment{解答}{
\hspace{-2zw}\phkasen<linethickness=7pt,iro=mygray,kasenUehosei=-3pt>{\bf \large \ 解答\ }\vspace{-1zw}\begin{leftbbar}}{\end{leftbbar}}
\newenvironment{解答1}{\ifkaisetukaitou
\hspace{-2zw}\phkasen<linethickness=7pt,iro=mygray,kasenUehosei=-3pt>{\bf \large \ 解答1\ }\vspace{-1zw}\begin{leftbbar}}{\end{leftbbar}}
\newenvironment{解答2}{\ifkaisetukaitou
\hspace{-2zw}\phkasen<linethickness=7pt,iro=mygray,kasenUehosei=-3pt>{\bf \large \ 解答2\ }\vspace{-1zw}\begin{leftbbar}}{\end{leftbbar}}
\newenvironment{解答3}{\ifkaisetukaitou
\hspace{-2zw}\phkasen<linethickness=7pt,iro=mygray,kasenUehosei=-3pt>{\bf \large \ 解答3\ }\vspace{-1zw}\begin{leftbbar}}{\end{leftbbar}}


\usepackage{qrcode}%QRコード
\setlength\normallineskiplimit{0pt}

%カラー,色
\usepackage{color}
\definecolor{link}{rgb}{0.63671875,0.99609375,0.99609375}
\definecolor{usumido}{rgb}{0.953125,0.95703125,0.9375}
%\pagecolor{usumido}
\definecolor{mygray}{gray}{0.75}

%はやくち解説ボタン
\usepackage[dvipdfmx]{hyperref}%ハイパーリンク
\usepackage{pxjahyper}
\newcommand{\botanA}{\LARGE\color{usumido}l\includegraphics[width=1.7cm,height=0.4cm,bb=0 0 1489 436]{はやくち解説ボタン3.jpeg}l}
%\hypersetup{% hyperrefオプションリスト
%setpagesize=false,
% bookmarksnumbered=true,%
% bookmarksopen=true,%
% colorlinks=true,%
% linkcolor=blue,
% citecolor=red,
%}

\newif\ifkaisetukaitou

\newcommand{\kaisetukaitou}{%問題のみ
\mondaisettei
\myfor{1} % ループ実行       
\newpage   
\setcounter{subsection}{0}
\setcounter{Question}{0}
\kaisetutukinosettei
\myfor{1} % ループ実行   
%\TileWallPaper{110mm}{160mm}{方眼紙.pdf} %方眼紙
%\myfor{1} % ループ実行   
}
\begin{document}

%\kaisetutukinosettei
\mondaisettei

\iffuru
\newcommand{\myfor}[1]{
\fi

{\bb\Large 微分法の基本}

%\bqu{\bb 微分係数の定義}\\
%次の関数の導関数を,導関数の定義に基づいて求めよ.
%\beda
%\item $y=\sin x$
%\item $y=\log x$
%\eeda
%\equ

\bqu{\bb 微分公式の証明(導関数の定義)}\\
次の関数の導関数を,導関数の定義に基づいて求めよ.
\beda<4>
\item $y=\sin x$
%\item $y=\cos x$
\item $y=x^n$
%\item $y=\bunsuu 1x$
%\item $y=\sq{x}$
\item $y=\log x$
\item $y=e^x$
\eeda
\equ

%\bqu{\bb 積/商の微分公式の証明}\\
%次を示せ.
%\beda
%\item $\{f(x)g(x)\}'=f'(x)g(x)+f(x)g'(x)$
%\item $\left\{\bunsuu{f(x)}{g(x)}\right\}'=\bunsuu{f'(x)g(x)-f(x)g'(x)}{\{g(x)\}^2}$
%\eeda
%\equ

%\bqu{\bb 合成関数微分公式の証明}\\
%%$\bunsuu{dy}{dx}=\bunsuu{dx}{dt}\cdot \bunsuu{dt}{dx}$を示せ.%,\ \ \ \ \ すなわち,
%微分公式$\{(f(g(x))\}'=f'(g(x))\cdot g'(x)$
%%\item $\bunsuu{dy}{dx}=\bunsuu{1}{\bunsuu{dx}{dy}}$を示せ.
%%\eeda
%\equ

\bqu{\bb 微分公式の証明(合成関数,積/商の微分利用)}\\
%次の問いに答えよ.
合成関数,積/商の微分公式を用いて,次の導関数を求めよ.ただし,$a$は1でない正の定数,$\alpha$は実数の定数である.
\beda<3>
\item $y=\cos x$
\item $y=\tan x$
\item $y=\log|x|$
\item $y=a^x$\ \ 
%\item $(e^x)'=e^x$を用いて,$y=\log x$を微分せよ.
\item $y=\log_a x$\ 
%\item $(x^{\alpha})'=\alpha x^{\alpha-1}$\ $(\alpha\ は実数)$を示せ.
\item $y=x^{\alpha}$\ 
\eeda
\equ

%\bqu{\bb 微分公式の証明}\\
%%次の問いに答えよ.
%合成関数,積/商の微分公式を用いて,次の導関数を求めよ.
%\benu
%\item $(\sin x)'=\cos x$を用いて$y=\cos x$を微分せよ.
%\item $\tan x=\bunsuu{\sin x}{\cos x}$用いて$y=\tan x$を微分せよ.
%\item $(\log x)'=\bunsuu 1x$を用いて$y=\log|x|$を微分せよ.
%\item $(e^x)'=e^x$を用いて$y=2^x$を微分せよ.
%%\item $(e^x)'=e^x$を用いて,$y=\log x$を微分せよ.
%\item $(\log x)'=\bunsuu 1x$を用いて$y=\log_2 x$を微分せよ.
%%\item $(x^{\alpha})'=\alpha x^{\alpha-1}$\ $(\alpha\ は実数)$を示せ.
%\item $(e^x)'=e^x$を用いて$y=x^{\alpha}$を微分せよ.
%\eenu
%\equ

\bqu{\bb 合成関数の微分練習}\\
次の関数を微分せよ.
\beda
\item $y=\sq{x^2+1}$
\item $y=\log(3x-2)$
\eeda
\equ

%\bqu{\bb 合成関数の微分}\\
%次の関数を微分せよ.
%\beda<3>
%\item $y=\bunsuu1{2x+1}$
%\item $y=\bunsuu1{x^2+1}$
%\item $y=\bunsuu1{\sq{x^2-1}}$
%\item $y=\sq{3-2x}$
%\item $y=\sq{2-x^2}$
%\item $y=\cos2x$
%\item $y=\sin^2 x$
%\item $y=\cos^3 x$
%\item $y=e^{-2x}$
%\item $y=e^{-\bunsuu{x^2}{2}}$
%\item $y=\log(3x)$
%\item $y=(\log x)^2$
%\item $y=\log x(1-x)$
%\item $y=\log (x^2+x+1)$
%\item $y=\log|\cos x|$
%\eeda
%\equ

%\bqu{\bb 微分公式の証明\ (積/商の微分利用)}\\
% $\tan x=\bunsuu{\sin x}{\cos x}$を用いて,関数$y=\tan x$を微分せよ.
%\equ

\bqu{\bb 積/商の微分練習}\\
次の関数を微分せよ.
\beda<3>
\item $y=(x^2+1)e^x$
\item $y=\bunsuu{\sin x}{x}$
\item $y=e^{2x}\cos x$
\eeda
\equ

%\bqu{\bb 対数微分法}\\
%
%\equ

%\bqu{\bb 積/商の微分}\\
%次の関数を微分せよ.
%\beda<3>
%\item $y=(x-2)^4(2x+1)$
%\item $y=\bunsuu{2x}{x-1}$
%\item $y=\bunsuu{x^2+1}{(x-3)^2}$
%\item $y=\bunsuu{2x^2+x+8}{x^2+3}$
%\item $y=x\sq{4-x}$
%\item $y=\bunsuu{(x+1)\sq{x+1}}{x^2}$
%\item $y=\bunsuu 1{\tan x}$
%\item $y=\sin 3x\cos 2x$
%\item $y=\bunsuu{\cos x}{1+\sin x}$
%\item $y=\bunsuu{e^x}{e^x+1}$
%\item $y=\bunsuu{e^x-e^{-x}}{e^x+e^{-x}}$
%\item $y=\bunsuu{\log x}{\sq x}$
%\item $y=x\log(2x)$
%\item $y=\bunsuu{2x-1}{e^x}$
%\item $y=e^{-x}\sin 3x$
%\eeda
%\equ

%\bqu{\bb やや複雑な微分計算}\\
%次の関数を微分せよ.
%\beda<3>
%\item $y=\bunsuu{x\sin x}{1+\cos x}$
%\item $y=\log(x+\sq{x^2+1})$
%\item $y=xe^{-\frac{x^2}{2}}$
%\eeda
%\equ

%\bqu{\bb やや複雑な微分計算}\\
%次の関数を微分せよ.
%\beda<3>
%\item $y=x\sq{2-x^2}$
%\item $y=\bunsuu{x}{\sq{x^2-4}}$
%\item $y=\sq{\bunsuu{1+x}{1-x}}$
%\item $y=\bunsuu{x}{x+\sq{x^2+1}}$
%\item $y=\bunsuu 12\left\{x\sq{x^2+4}+4\log(x+\sq{x^2+4})\right\}$
%\item $y=\bunsuu{(1+x^2)^{\frac 32}}{x}$
%\item $y=\log\bunsuu{1+\sq{1-x^2}}{x}-\sq{1-x^2}$
%\item $y=\sin x \cos^3 x$
%\item $y=\bunsuu{\sin x}{\sq{1-\cos x}}$
%\item $y=\log\left|\tan\bunsuu x2\right|$
%\item $y=e^x \cos^2 x$
%\item $y=\log\bunsuu{1+\cos x}{1-\cos x}$
%\eeda
%\equ

\bqu{\bb 少し進んだ微分計算}\\
次の各々について,$\bunsuu{dy}{dx}$を求めよ.
\beda
\item $x^2+y^2=1$\ \ \ $(xとyで表せ)$
\item $\barr
x=\cos^3 t\\ y=\sin^3 t 
\earr$\ $(tで表せ)$
\eeda
\equ

\bqu{\bb 対数微分法}\\
 関数$y=x^x$\ $(x>0)$を微分せよ.
\equ

%\bqu{\bb そのほかの微分法}\\
%次の各々について,$\bunsuu{dy}{dx}$を求めよ.
%\beda
%\item $x=\sin y\ \left(-\bunsuu{\pi}{2} < y<\bunsuu{\pi}{2}\right)$\ \ $(xで表せ)$
%\item $\barr x=(1+\cos \theta)\cos \theta \\
%(1+\cos \theta)\sin \theta \earr$\ $(\theta で表せ)$
%\item $\bunsuu{x^2}{9}+\bunsuu{y^2}{4}=1\ \ (x,\ y で表せ)$
%\item $y=x^x\ \ (x >0)$
%\eeda
%\equ

\newpage

\bqu{\bb 接線・法線}\\
次の直線の方程式を求めよ.
\benu
\item 曲線$y=\sq{x^2-3}$の点$(2,\ 1)$における接線$l_1$
\item 点$(0,\ 1)$から曲線$y=\log x$に引いた接線$l_2$
\item 曲線$y=\cos x$の,$x=t$における法線$l_3$
\eenu
\equ

\bqu{\bb 増減を調べる}\\
次の関数の増減を調べよ.
\benu
\item $f(x)=\sq x-\log x$\ \ $(x>0)$
\item $g(x)=\cos 3x-3\cos x$\ \ $\left( 0\leqq x \leqq \bunsuu{\pi}{2}\right)$
\eenu
\equ

\bqu{\bb $f'(x)$を用いたグラフ(1)}\\
 関数$y=\bunsuu{x^2}{x-1}$の増減を調べ,そのグラフを描け.また,漸近線についても調べよ.
\equ

\bqu{\bb $f'(x)$を用いたグラフ(2)}\\
次の関数の増減を調べ,そのグラフを描け.
\beda
\item $y=\bunsuu{\log x}{x}$
\item $y=\sin x(1-\cos x)$\ \ $(-\pi \leqq x \leqq \pi)$
\eeda
\equ

\bqu{\bb $f''(x)$まで用いたグラフ}\\
次の関数のグラフの概形を描け.
\[y=\bunsuu{x}{e^x}\]
\equ

\bqu{\bb パラメタ曲線}\\
媒介変数表示された曲線$C$の概形をかけ.
\[\barr x=t-\sin t\\ y=1-\cos t \earr\ \ (0 \leqq t\leqq 2\pi)\]
\equ

%\bqu{\bb 最大・最小}\\
%次の関数の最大・最小を求めよ.
%\beda
%\item $f(\theta)=\bunsuu{\sin\theta}{3-2\cos^2\theta}$\ \ $(0<\theta<\pi)$
%\item $y=x(\sq{12-x^2})^3$\ \ $(1<x<2\sq 3)$
%\eeda
%\equ

\newpage

%\bqu{\bb 分数関数の極値}\\
%関数$f(x)=\bunsuu{4x+3}{x^2-x+1}$の極小値を求めよ.
%\equ
%
%\bqu{\bb 極値,変曲点をもつ条件}\\
%次の各々の問いに答えよ.
%\benu
%\item $f(x)=\sin x+ax$が極値をもたないのは,定数$a$がどんな値のときか.
%\item $f(x)=x^4-4x^3+6ax^2$とする.$y=f(x)$が変曲点をもつのは,$a$がどんな値のときか.
%\eenu
%\equ
%
%\bqu{\bb 極値から係数決定}\\
%関数$f(x)=\cos2x+a\cos x+b$は,$x=\bunsuu{\pi}{3}$で極値1ような定数$a$,$b$の値を求めよ.
%\equ
%
%\bqu{\bb 最大・最小}\\
%関数$f(x)=x+\sq{2x-x^2}$\ $(0\leqq x \leqq 2)$の増減を調べ,その最大値,最小値を求めよ.
%\equ
%
%\bqu{\bb 三角関数の最小値}\\
%$y=2\sin x+\sq 3\sin 2x$の最小値を求めよ.
%\equ
%
%\bqu{\bb $\sin x$,$\cos x$の多項式評価}\\
%$x>0$のとき,次の不等式が成り立つことを示せ.
%\benu
%\item $\cos x >1-\bunsuu{x^2}{2!}$
%\item $\sin x >x-\bunsuu{x^3}{3!}$
%\item $\cos x >1-\bunsuu{x^2}{2!}+\bunsuu{x^4}{4!}$
%\eenu
%ただし,$x>0$のとき,不等式$\sin x<x$が成り立つことは証明せずに用いて良い.
%\equ
%
%\bqu{\bb $e^x$の多項式評価}\\
%$x>0$のとき,次の不等式が成り立つことを証明せよ.
%\beda
%\item $e^x >1+x+\bunsuu{x^2}{2!}$
%\item $e^x >1+x+\bunsuu{x^2}{2!}+\bunsuu{x^3}{3!}$
%\eeda
%\equ
%
%\bqu{\bb $e^x$の多項式評価}\\
%$n$を任意の自然数とする.$x>0$のとき,不等式
%\[e^x >1+x+\bunsuu{x^2}{2!}+\cdots+\bunsuu{x^n}{n!}\]
%が成り立つことを数学的帰納法で証明せよ.
%\equ
%                     

\iffuru
}\kaisetukaitou
\fi

\end{document}